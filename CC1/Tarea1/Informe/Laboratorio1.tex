\documentclass[spanish, fleqn]{article}
\usepackage{babel}
\usepackage[utf8]{inputenc}
\usepackage{amsmath}
\usepackage{amsfonts}
\usepackage{wasysym}
\usepackage[colorlinks, urlcolor=blue]{hyperref}
\usepackage[top = 2.5cm, bottom = 2cm, left = 2cm, right = 2cm]{geometry}
\usepackage{listings}
\usepackage{color}

\definecolor{gris}{rgb}{0.2,0.2,0.2}

\title{ILI 285 \\
       Laboratorio \#1 \\
      }
\author{Alonso Sandoval Acevedo\\asandova@alumnos.inf.utfsm.cl\\201073011-5 \and 
		Hernán Vargas Leighton\\hvargas@alumnos.inf.utfsm.cl\\201073009-3}
\date{\today}

\begin{document}
\maketitle

\thispagestyle{empty}

\section{Descripción del experimento}
	%Descripción del experimento y suposiciones
	El experimento realizado pretende dejar en evidencia los problemas asociados
	a la precisión del computador al realizar cómputo con cantidades muy
	pequeñas o distantes entre sí. Para ello se expondrán cálculos con cifras
	que representan cantidades extremas en la representación de punto
	flotante(double-precision floating-point), o de gran diferencia en órdenes
	de magnitud entre sí, de manera de evidenciar los problemas previstos.
	Veremos también cómo influye en ello la cantidad conocida como $e_{mach}$
	(machine epsilon). 
\section{Desarrollo}
	%Desarrollo y análisis de resultados
	\subsection{Representación punto flotante}
	\begin{enumerate}
		\item
			El resultado entregado de la diferencia $5-2^{-53}$ en Python es 5.
		\begin{enumerate}
			\item
				Lo primero que podríamos pensar es que el computador guarda como
				cero el $2^{-53}$. Sin embargo, lo que realmente ocurre es que
				el computador ''considera'' que $2^{-53}$ comparado con 5 es 
				prácticamente cero.\\
				Esto ocurre debido a la representación en punto flotante doble
				precisión de las cifras, por lo que el problema lo podemos
				notar al observar estas cantidades en la notación punto
				flotante:
				\begin{center}
					$5-2^{-53}$ = $(1.01$ x $2^{2} )$ x $2^{-53}$
				\end{center}
				Sabemos que la mantisa en la representación punto flotante con
				doble precisión ocupa 52 bits, por lo que una diferencia en
				órdenes de magnitud de más de $(16)_{10} = (52)_{2}$ implicaría
				que el ordenador omita la cantidad pequeña, en nuestro caso, si
				vemos la diferencia de los exponentes, el ordenador es tolerante
				hasta una diferencia de 
				$|(2 - 52)| = 50 \rightarrow (2^{50})_{2}, 50$ órdenes de
				magnitud versus el 5. Esto ocurre ya que al guardar el $5$ se 
				utilizan dos decimales de la mantisa, dejando espacio solo para
				$50$ más.
			\item
				$e_{mach}$ representa la diferencia entre el 1 y el menor número
				representable mayor a 1, en notación punto flotante doble
				precisión $e_{mach} = 2^{-53}$.\\
				El error presentado en la pregunta anterior ocurre debido a que
				el resultado real está entre un número y el	siguiente número
				representable, por ello el resultado debe ser aproximado 
				perdiendo así exactitud.\\
				Para que este tipo de errores no ocurran debe suceder que los
				números con los cuales se opera no sean tan distantes entre sí.
				En el ejemplo anterior podemos calcular cual es esta barrera por
				medio de la fórmula $e_{mach} * |5_2|$ es decir:
				\begin{equation*}
					2^{-53} * 1.01 * 2^2 \approx 2^{-51}
				\end{equation*}
				Es decir, para obtener un resultado debemos restarle al 5 
				números con orden de magnitud mayor a $2^{-51}$. En python 
				comprobamos que es posible almacenar el resultado de una
				operación como $5 - 2^{50}$.
		\end{enumerate}
		\item
			Cacular $(5-3$ x $2^{-53})+5$.
		\begin{enumerate}
			\item
				El resultado aritmético exacto es (a través de wolframalpha):\\
				$9.99999999999999966693309261245303787291049957275390625$
			\item
				El resultado aritmético computacional obtenido con Python es:
				10. Identificamos el problema o la diferencia con un cálculo más
				preciso, tal como en el ejercicio anterior, en la significancia
				entre el $3$ y $2^{-53}$ es decir: $ 1.1$ x $2^1$ y $ 2^{-53}$,
				la diferencia en órdenes de magnitud es de $|1 - 52| = 51$, por
				lo que números menores que $2^{-51}$ son 0 con respecto al 3.
			\item
				El cero en punto flotante con doble precisión puede ser
				representado de dos formas, positivo y negativo, con su signo 0
				o 1 respectivamente:
				\begin{center}
					\begin{tabular}{|c|c|c|}
					\hline 
						signo(1 bit) & mantisa(52 bits) & exponente(11 bits) \\ 
					\hline 
						0 & 000....0 & 00...0 \\ 
					\hline 
						1 & 000....0 & 00...0 \\ 
					\hline 
					\end{tabular}
				\end{center}

		\end{enumerate}
	\end{enumerate}
	\subsection{Perdida de significancia}
	\begin{enumerate}
		\item
			Primero obtenemos el límite de la expresión utilizando la regla de
			l'Hôpital. Derivando 3 veces y evaluando en 0 tenemos que el
			límite de la expresión cuando $x \rightarrow 0$ es $\frac{1}{3}$\\
			Ahora sabemos que el resultado de la expresión tenderá a 
			$\frac{1}{3}$ cuando $p \rightarrow \infty$. 
			Ejecutamos el script (ver anexo) con $1 \leq p \leq 20$ y obtenemos
			que la salida es:
			\begin{verbatim}
				01: 0.000341666706845/0.001 = 0.341666706845
				02: 3.34166666782e-07/1e-06 = 0.334166666782
				03: 3.33416849685e-10/1e-09 = 0.333416849685
				04: 3.33510996597e-13/1e-12 = 0.333510996597
				05: 4.4408920985e-16/1e-15 = 0.44408920985
				06: 0.0/1e-18 = 0.0
				07: -2.22044604925e-16/1e-21 = -222044.604925
				08: 0.0/1e-24 = 0.0
				09: 0.0/1e-27 = 0.0
				10: 0.0/1e-30 = 0.0
				11: 0.0/1e-33 = 0.0
				12: 0.0/1e-36 = 0.0
				13: 0.0/1e-39 = 0.0
				14: -2.22044604925e-16/1e-42 = -2.22044604925e+26
				15: -2.22044604925e-16/1e-45 = -2.22044604925e+29
				16: 0.0/1e-48 = 0.0
				17: 0.0/1e-51 = 0.0
				18: 0.0/1e-54 = 0.0
				19: 0.0/1e-57 = 0.0
				20: 0.0/1e-60 = 0.0
			\end{verbatim}
			Como vemos, el script parece funcionar bien hasta $p=4$, pero
			después de eso empiezan a suceder cosas extrañas y termina dando 0
			reiteradas veces.\\
			Para ver bien cual es el problema que está ocurriendo activamos la
			opción verbose e iteramos desde 4 hasta 7 obteniendo como resultado:
			\begin{verbatim}
				04: 3.33510996597e-13/1e-12 = 0.333510996597
				x         : 0.0001
				a = e^x   : 1.0001000050001667
				b = cos(x): 0.999999995
				c = sin(x): 9.999999983333334e-05
				d = x^3   : 1.0000000000000002e-12
				a + b     : 2.0001000000001667
				b - c     : 0.9998999950001667
				(a+b) - c : 2.0000000000003335
				a + (b-c) : 2.0000000000003335
				##################################################
				05: 4.4408920985e-16/1e-15 = 0.44408920985
				x         : 1e-05
				a = e^x   : 1.00001000005
				b = cos(x): 0.99999999995
				c = sin(x): 9.999999999833334e-06
				d = x^3   : 1.0000000000000003e-15
				a + b     : 2.00001
				b - c     : 0.9999899999500002
				(a+b) - c : 2.0000000000000004
				a + (b-c) : 2.0
				##################################################
				06: 0.0/1e-18 = 0.0
				x         : 1e-06
				a = e^x   : 1.0000010000005
				b = cos(x): 0.9999999999995
				c = sin(x): 9.999999999998333e-07
				d = x^3   : 9.999999999999999e-19
				a + b     : 2.000001
				b - c     : 0.9999989999994999
				(a+b) - c : 2.0
				a + (b-c) : 2.0
				##################################################
			\end{verbatim}
			Para $p=4$ notamos que no existen problemas, para $p=5$ vemos que
			si la expresión es evaluada como $a + (b - c)$ ocurrirá un error, 
			pero, como obtenemos un resultado distinto de 0, notamos que el
			computador evalúa $(a + b) - c$.\\
			En el caso $p=6$ vemos que la expresión se hace 0 siempre, es decir,
			el computador incurre en un error de cancelación en $(a+b)-c$ donde
			el resultado debe ser muy cercano a 2, pero no 2, por ello decimos
			que para $1 \leq p \leq 5$ no hay errores de cancelación. 

		\begin{enumerate}
			\item
				El error relativo lo calcularemos según el resultado en la
				representación punto flotante dado por la iteración mostrada en
				el punto anterior, cuando $p = 4$. A pesar de que se 
				logra calcular un valor para $p = 5$ sin incurrir en errores de
				cancelación, este valor tiene un error mayor que el de $p=4$.\\
				Como valor esperado, utilizaremos el resultado entregado por el 
				computo de la expresión en wolframalpha.
				\begin{center}
					$\frac{|0.333510996597 - 0.333334166...|}{|0.333334166...|}
					= 0.00053049$
				\end{center}
				Por la presición de la representación punto flotante usada, el 
				mejor valor a conseguir tendrá un error cercano a $\frac{1}{2} 
				e_{mach}$, sin embargo en nuestro computo insiden otros 
				factores que aumentan el error (lo alejan del mejor caso 
				posible).\\
				En primer lugar, el computo de cada parte de la expresión 
				conlleva un redondeo dado que no se pueden almacenar infinitos
				decimales. Luego, detectamos un error en el paso de decimal, 
				binario, decimal, osea cuando se ingresa el valor, se traspasa 
				máquina y luego se vuelve a recuperar.\\
				Esto se puede mejorar arreglando la expresión, de manera de 
				sopesar estos errores.
			\item
				Computacionalmente el mejor valor que podemos conseguir tiene un 
				error entre 0 (el valor exacto) y $\frac{1}{2} e_{mach}$, pero 
				nuestro algoritmo incurre en errores mucho más grandes pues se
				hacen aproximaciones($e^x, cos(x)$, etc.) y hay un error de 
				cancelación.
		\end{enumerate}
	\end{enumerate}

\section{Conclusiones}
	%Conclusiones
	Existen problemas de cálculo computacional cuando las cifras presentes son 
	extremadamente pequeñas, grandes, o, en nuestro caso de estudio, difieren
	mucho entre si (en ordenes de magnitud), esto ocurre por la representación
	que tienen en la máquina las cifras. El presente laboratorio se centró en
	la representación en punto flotante doble presición, y en particular, en el
	análisis de la cifra $e_{mach}$ la cual nos da cuenta de cuando las cifras
	son comparables entre si al momento de operar, así como del error posible
	al momento de una aproximación. En la representación mencionada existe una
	''tolerancia'' entre cifras del orden de $10^{16}$, la cual es justamente
	$e_{mach} = 2^{-52}$.\\
	De esto derivan los problemas de cancelación (diferencía por debajo de lo
	representable), aproximación, entre otros, con las cuales existen técnicas
	que permiten acomodar las expresiones de manera de evitar omitir cifras
	significativas.

\section{Referencias}
%Referencias
\begin{enumerate}
	\item
		 http://drj11.wordpress.com/2007/07/03/python-poor-printing-of-floating-point/,
		 Acerca de por que no usar la conversión implicita de flotante a string.
\end{enumerate}

\newpage
\section{Anexo}
	%hack para acentos:
	\lstset{literate=
		{á}{{\'a}}1 {é}{{\'e}}1 {í}{{\'i}}1 {ó}{{\'o}}1 {ú}{{\'u}}1
		{Á}{{\'A}}1 {É}{{\'E}}1 {Í}{{\'I}}1 {Ó}{{\'O}}1 {Ú}{{\'U}}1
		{à}{{\`a}}1 {è}{{\'e}}1 {ì}{{\`i}}1 {ò}{{\`o}}1 {ù}{{\`u}}1
		{À}{{\`A}}1 {È}{{\'E}}1 {Ì}{{\`I}}1 {Ò}{{\`O}}1 {Ù}{{\`U}}1
		{ä}{{\"a}}1 {ë}{{\"e}}1 {ï}{{\"i}}1 {ö}{{\"o}}1 {ü}{{\"u}}1
		{Ä}{{\"A}}1 {Ë}{{\"E}}1 {Ï}{{\"I}}1 {Ö}{{\"O}}1 {Ü}{{\"U}}1
		{â}{{\^a}}1 {ê}{{\^e}}1 {î}{{\^i}}1 {ô}{{\^o}}1 {û}{{\^u}}1
		{Â}{{\^A}}1 {Ê}{{\^E}}1 {Î}{{\^I}}1 {Ô}{{\^O}}1 {Û}{{\^U}}1
		{œ}{{\oe}}1 {Œ}{{\OE}}1 {æ}{{\ae}}1 {Æ}{{\AE}}1 {ß}{{\ss}}1
		{ç}{{\c c}}1 {Ç}{{\c C}}1 {ø}{{\o}}1 {å}{{\r a}}1 {Å}{{\r A}}1
		{€}{{\EUR}}1 {£}{{\pounds}}1
	}
	%Anexo con código utilizado
	\lstinputlisting[language=Python,
				 frame=single,
				 showstringspaces=false,
				 commentstyle=\color{gris},
				 title=\lstname,
			 	 tabsize=4]{../Codigos/Pregunta1.py}
	\newpage
	\lstinputlisting[language=Python,
				 frame=single,
				 showstringspaces=false,
				 commentstyle=\color{gris},
				 title=\lstname,
			 	 tabsize=4]{../Codigos/Pregunta2.py}

\vfill\hfill HV/AS/\LaTeXe
\end{document}
