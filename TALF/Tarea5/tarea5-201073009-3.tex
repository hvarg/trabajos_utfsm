\documentclass[spanish, fleqn]{article}
%\usepackage{babel}		Tikz...
\usepackage[spanish, es-noshorthands]{babel}
\usepackage[utf8]{inputenc}
\usepackage{amsmath}
\usepackage{amsfonts}
\usepackage{mathrsfs}  
\usepackage{wasysym}
\usepackage[colorlinks, urlcolor=blue]{hyperref}
\usepackage{fourier}
\usepackage[top = 2.5cm, bottom = 2cm, left = 2cm, right = 2cm]{geometry}

\usepackage{tikz}
\usetikzlibrary{arrows, shapes}

\title{Introducción a la Informática Teórica \\
       Tarea 5 \\
       ``¡norecursivorecursivorecursivamenteenumerablenorecursivo!''
      }
\author{Hernán Vargas \\ 201073009-3}
\date{2 de junio 2014}

\begin{document}
\maketitle
\thispagestyle{empty}

\section*{Respuestas}

\begin{enumerate}
	\item
		Conceptos
		\begin{itemize}
			\item 
%				¿Qué se puede decir de un lenguaje \(\mathcal{L}\) si se sabe
%				que su complemento \(! \mathcal{L}\) \textbf{no} es
%				recursivamente enumerable?
				Sabemos que: $!\mathcal{L}$ es recursivamente enumerable 
				$\Leftrightarrow \mathcal{L} $ es recursivo. De esto podemos
				concluir que si nuestro lenguaje $!\mathcal{L}$ no es
				recursivamente enumerable, entonces el lenguaje $\mathcal{L}$ a
				lo más será recursivamente enumerable.

			\item 
%				¿Qué relación hay entre un lenguaje que es decidible y un
%				lenguaje recursivo?
				Un lenguaje decidible es aquel para el cual existe una máquina
				de Tuning que acepta y se detiene, lo cual también es la
				definición de lenguaje recursivo. En síntesis son lo mismo.
			
			\item
%				¿Por qué se dice que un problema en \(\mathcal{P}\) también
%				está en  \(\mathcal{NP}\)?
				Por definición sabemos que un problema $\mathcal{P}$ es aquel
				que puede ser resuelto en tiempo polinomial por una máquina de
				Tuning determinista. Por otro lado, un problema $\mathcal{NP}$
				puede ser resuelto en tiempo polinomial por una máquina de 
				Tuning no determinista, pero una máquina de Tuning determinista
				es un caso particular de una no determinista, por lo tanto
				podemos decir que un problema $\mathcal{P}$ también está en 
				$\mathcal{NP}$.

			\item
%				Dados un problema \(P\) que se sabe en \(\mathcal{NP}\) y un
%				problema \(\mathcal{NP}\)-completo \(P_c\)
%				¿qué puede concluir de una reducción polinomial de \(P_c\) a
%				\(P\)?.
				Si logramos reducir $P_c$ a $P$ entonces $P$ es un problema 
				$\mathcal{NP}$-completo, ya que $P$ está en $\mathcal{NP}$.
				Ahora todo problema en $\mathcal{NP}$ será reducible a $P$.

			\item
%				¿Cuál es la diferencia entre un lenguaje recursivamente
%				enumerable y uno recursivamente enumerable no recursivo?
				Un lenguaje recursivamente enumerable no recursivo será a lo
				más recursivamente enumerable, es decir la maquina de Tuning que
				lo representa no parará en todos los casos. En contraposición,
				un lenguaje recursivamente enumerable puede ser recursivo, lo
				que indica que su máquina de Tuning se detiene en todos los
				casos.
		\end{itemize}

	\item
%		Se define un \emph{Autómata Linealmente Acotado} como una TM que se
%		detiene en cuanto sale de la entrada original (se detiene si lee un
%		blanco o retrocede antes del comienzo de la cinta). 
%		Demuestre que el lenguaje \(\mathcal{A}_{LA} = \{<M,\omega>:\) M es un
%		\emph{Autómata Linealmente Acotado}  y \(\omega \in \mathcal{L}(M) \}\)
%		es decidible.
		Para demostrar que $\mathcal{A}_{LA}$ es decidible nos basta con
		demostrar que un \emph{Autómata Linealmente Acotado} es recursivo, ya
		que $\mathcal{A}_{LA}$ está compuesto por las palabras $\omega$ que son
		aceptadas por el \emph{Autómata Linealmente Acotado}. Además sabemos que
		ésta máquina de Tuning se detiene cuando sale de la entrada original,
		es decir, se detendrá en todos los casos y por lo tanto es recursiva.
	
	\item
%		El famoso \emph{Halting problem} de Turing consiste en determinar si
%		dada una TM \(M\) y un string \(\omega\), \(M\) se detendrá al procesar
%		como entrada a \(\omega\).
%		En 1936 Alan Turing demostró que este problema es indecidible.
%		Utilizando esta información, ¿Se puede determinar si la TM \(M_b\) se
%		detiene al comenzar con puros blancos grabados en su cinta?
%		\emph{Pista: utilice las famosas ``cajitas''}
		Digamos $M_h$ una máquina que prueba el \emph{Halting problem}, pide un
		string $\omega$ y una máquina de Tuning MT y dice ''Si`` si la máquina
		se detiene, en este caso digamos $\omega$ una cinta que comienza con
		puros blancos y la máquina de Tuning $M_b$,	tenemos:

		%Estilo
		\tikzstyle{caja}=[draw, fill=gray!50, minimum size=1em]
		\tikzstyle{init}=[pin edge={to-,thin,black}]
		%Cajas
		\begin{center}
			\begin{tikzpicture}[auto]
				%Máquinas
				\node [caja, pin={[init]above:$M_b$}] (a) {$M_h$};
				%Caja
				\draw[color=gray,thick](-2, 1.5) rectangle (2,-1);
				%Flecha w
				\node (b) [left of=a,node distance=3cm, coordinate] {a};
				\path[->] (b) edge node {$\omega$} (a);
				%Felcha out
				\node [coordinate] (end) [right of=a, node distance=3cm]{};
				\draw[->] (a) edge node {$Si$} (end);
			\end{tikzpicture}
		\end{center}
		Ahora probar que $M_b$ se detiene sería equivalente a probar el
		\emph{Halting problem}, pero sabemos este último indecidible, por lo
		tanto $M_b$ es indecidible.

\end{enumerate}
\vfill\hfill HV/\LaTeXe
\end{document}
