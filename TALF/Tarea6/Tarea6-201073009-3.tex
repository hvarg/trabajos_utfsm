\documentclass[spanish, fleqn]{article}
%\usepackage{babel}
\usepackage[spanish, es-noshorthands]{babel}
\usepackage[utf8]{inputenc}
\usepackage{amsmath}
\usepackage{amsfonts}
\usepackage{mathrsfs}
\usepackage{wasysym}
\usepackage[colorlinks, urlcolor=blue]{hyperref}
\usepackage{fourier}
\usepackage[top = 2.5cm, bottom = 2cm, left = 2cm, right = 2cm]{geometry}
\usepackage{tikz}
\usetikzlibrary{arrows}

\title{Introducción a la Informática Teórica \\ Tarea 6\\
	   \emph{Hamiltonian Guy} }
\author{Hernán Vargas Leighton \\ 201073009-3}
\date{\today}

\begin{document}
\maketitle
\thispagestyle{empty}

\section*{Respuestas}

\begin{enumerate}
	\item
%		\emph{Vertex Cover}. Dado un grafo \(G = (V, E)\) se debe encontrar el
%		mínimo conjunto \(V' \in V \) de vértices tal que todos los vértices del
%		grafo son adyacentes a éstos. Demuestre que \emph{Vertex covering
%		problem} es $NP$-completo.
%		\emph{Hint: puede reducir el 3-SAT a este hermoso problema}
		Buscamos el menor subconjunto de nodos tal que todas las aristas sean 
		adyacentes a estos. Para ello debemos recorrer cada nodo y verificar
		sus aristas, por lo tanto la evaluación caso a caso será en tiempo
		polinomial. Como buscamos el menor subconjunto debemos siempre recorrer
		el grafo hasta el final, encontrar todos los subconjuntos que son
		\emph{vertex cover}, y elegir de ellos el más pequeño, entonces el
		problema es \texttt{NP}.\\
		Para que el problema sea \texttt{NP}-completo debe ser \texttt{NP} y 
		\texttt{NP}-duro, esta última condición se satisface haciendo la 
		reducción de 3-SAT a nuestro problema:
		\begin{itemize}
			\item
				3-SAT nos dice que el problema de verificar si existen valores
				booleanos tal que una expresión compuesta por clausulas de tres
				literales sea verdadera es \texttt{NP}-completo, es decir:
				$ S_1 \times S_2 \times .. S_n = \texttt{True}$, con $\times$
				una operación booleana (\texttt{or, and, not}) y $S_i$ una
				clausula de la forma $(L_1 \times L_2 \times L_3)$
			\item
				Digamos las clausulas $S_i$ de manera que $L_1, L_2, L_3$ sean
				valores \texttt{True} o \texttt{False} arbitrarios, así $\times$
				puede ser reducido a \texttt{and} u \texttt{or} (el \texttt{not}
				no es necesario pues basta cambiar el valor de $L_i$ por su 
				complemento).
			\item
				Ahora, si la expresión está compuesta por dos \texttt{or}, le
				asignamos un grafo con tres nodos y dos aristas de manera de que
				un nodo sea adyacente a los otros dos. En cualquier otro caso
				asignamos un grafo con tres nodos y tres aristas entonces cada
				nodo estará conectado con los demás. Así, asignamos a cada
				clausula grafos de tres nodos, de alguna de estas formas:
				\begin{center}
					\begin{tikzpicture}[-,>=stealth',shorten >=1pt,auto,node
								distance=1.5cm,thick,main node/.style={circle,
								fill=gray!40,draw,minimum size=10pt}]
						\node[main node] (1) {$L_1$};
						\node[main node] (2) [below right of=1] {$L_2$};
						\node[main node] (3) [above right of=2] {$L_3$};
	
						\node[main node] (4) [right of=3]{$L_1$};
						\node[main node] (5) [below right of=4] {$L_2$};
						\node[main node] (6) [above right of=5] {$L_3$};
						\path[]
							(1) edge node [] {} (2)
							(2) edge node [] {} (3)
	
							(4) edge node [] {} (5)
							(5) edge node [] {} (6)
							(6) edge node [] {} (4);
					\end{tikzpicture}
				\end{center}
			\item
				Cada uno de estos grafos tendrá al menos un nodo con dos
				aristas, llamemos a éste el nodo principal.
			\item 
				Crearemos un nuevo grafo pero ahora con la expresión que evalúa
				las clausulas, para ello cuando las expresiones estén con un
				\texttt{and} pondrán los nodos linealmente mientras que las con
				\texttt{or} unirán solo uno de sus nodos a los \texttt{and}, por
				ejemplo una expresión como $S_1 \land S_2 \land (S_3 \lor S_4
				\lor S_5) \land	S_6$ será:
				\begin{center}
					\begin{tikzpicture}[-,>=stealth',shorten >=1pt,auto,node
								distance=1.5cm,thick,main node/.style={circle,
								fill=gray!40,draw,minimum size=10pt}]
						\node[main node] (1) {$S_1$};
						\node[main node] (2) [above right of=1] {$S_2$};
						\node[main node] (3) [below right of=2] {$S_3$};
						\node[main node] (4) [above right of=3]{$S_4$};
						\node[main node] (5) [below right of=3] {$S_5$};
						\node[main node] (6) [below right of=1] {$S_6$};
						\path[]
							(1) edge node [] {} (2)
							(2) edge node [] {} (3)
							(3) edge node [] {} (6)
							(3) edge node [] {} (4)
							(3) edge node [] {} (5)
							(6) edge node [] {} (1);
					\end{tikzpicture}
				\end{center}
			\item
				Ahora a cada nodo $S_i$ se le agrega una arista y se conecta con
				el nodo principal de los literales de cada clausula. Con este
				proceso se logra transformar una expresión booleana 3-SAT a un
				grafo.
			\item
				Buscar si la expresión booleana satisface el 3-SAT será
				equivalente a solucionar el problema \emph{Vertex Cover}, por lo
				tanto se logra la reducción.
		\end{itemize}

	\item
%		\emph{Set Cover}. Dado un conjunto \(U\) de \(n\) elementos, y \(m\)
%		subconjuntos \(S_i\), ¿se pueden encontrar a lo más \(k\) subconjuntos
%		tal que unidos cubran \(U\)? Reduzca el problema \emph{Vertex Cover}
%		resuelto anteriormente al \emph{Set Cover}.
		Reducción de \emph{Vertex Cover} a \emph{Set Cover}:
		\begin{itemize}
			\item
				Digamos nuestro grafo para el \emph{Vertex Cover} como $G = (V,
				E)$, con $V$ el conjunto de los nodos y $E$ el conjunto de las
				aristas.
			\item
				Digamos los elementos de $V$ como $V_i \forall i \in 1..p$, 
				entonces $V_i$ representará al nodo $i$ de un total de $p$.
			\item
				Digamos los elementos de $E$ como $E_i \forall i \in 1..q$, 
				$E_i$ será la arista $i$ de un total de $q$.
			\item
				Definamos el conjunto U como la unión de los nodos y las aristas
				de nuestro grafo, es decir $ U = V \cup E $. 
			\item
				Los subconjuntos $S_i$ serán elegidos de manera que se contengan
				un nodo y todos los nodos y aristas adyacentes, por ejemplo para
				un nodo $V_1$ que es adyacente a $E_1, E_2, V_2, V_3$ tendremos
				el subconjunto $\{V_1, E_1, E_2, V_2, V_3\}$. Hacer esta
				transcripción implica recorrer los nodos uno por uno y revisar
				sus vecinos, por lo tanto su tiempo es polinomial.
			\item
				Tendremos entonces que el número de elementos $n = p+q$ y el
				número de subconjuntos $m=n$.
			\item
				Ahora encontrar el \emph{Vertex Cover} de $G$ será equivalente a
				encontrar un mínimo $k$ tal que su unión sea $U$, por lo tanto
				se logra la reducción.
		\end{itemize}

	\item
%		\emph{Half-Clique}. Dado un grafo \(G\) con un número par de vértices.
%		¿Existe un \emph{clique} de G que contenga exactamente la mitad de la
%		cantidad de nodos de \(G\)? Demuestre que este es un problema
%		\texttt{NP}-completo.\\
%		\emph{Hint:} Reduzca el problema \emph{Clique} (visto en clases) al
%		\emph{Half-Clique}.
		En primer lugar el algoritmo para resolver el \emph{Half-Clique} será
		polinomial pues nos basta con tomar toda combinación posible de nodos
		que sean la mitad de los nodos totales y verificar si todos están 
		conectados entre si, así el problema es \texttt{NP}.\\
		Para probar que el problema es \texttt{NP}-duro y por ello \texttt{NP}-
		completo basta reducir el problema \emph{Clique} al \emph{Half-Clique}:
		\begin{itemize}
			\item
				\emph{Clique} nos dice que el problema de encontrar un $k$ tal
				que un grafo $G$ tenga un  subconjunto $G'$ con $k$ nodos de
				manera que estos estén todos conectados entre si, es \texttt{NP}
				-completo.
			\item
				Digamos un grafo $G$ con $n$ nodos.
			\item
				Por definición de \emph{Half-Clique} sabemos que $n$ será par,
				entonces $n/2$ es un natural mayor o igual a $1$.
			\item
				Basta con decir $k=n/2$, ahora el problema \emph{Clique} fue
				reducido a un problema \emph{Half-Clique}.
			\item
				Notamos que \emph{Half-Clique} es un caso particular de
				\emph{Clique}.
		\end{itemize}
		Con esto se concluye que \emph{Clique} es \texttt{NP}-completo.

\end{enumerate}
\vfill\hfill HV/\LaTeXe
\end{document}
