\documentclass[spanish, fleqn]{article}
\usepackage{babel}
\usepackage[utf8]{inputenc}
\usepackage{amsmath}
\usepackage{amsfonts}
\usepackage{wasysym}
\usepackage[colorlinks, urlcolor=blue]{hyperref}
\usepackage{fourier}
\usepackage[top = 2.5cm, bottom = 2cm, left = 2cm, right = 2cm]{geometry}

\renewcommand{\thefootnote}{\fnsymbol{footnote}}

\title{Introducción a la Informática Teórica \\
       Tarea 4\\
       ``¡Quiero mi Premio Turing Ahora!''
      }
\author{Hernán Vargas Leighton \\ 201073009-3}
\date{\today}

\begin{document}
\maketitle
\thispagestyle{empty}

\section*{Respuestas}

\begin{enumerate}
	\item
%		Construya la Máquina de Turing que acepte el lenguaje \(\mathcal{L}\) 
%		cuyas cadenas contienen la misma cantidad de ceros que unos. Defina
%		claramente símbolos y estados. Utilice la tabla \ref{tablatm} para
%		describir la máquina.
		La Máquina de Turing que acepta el lenguaje $\mathcal{L} = \{w: 
		\#1 = \#0\}$ será $ M_1 = (Q, \Sigma, \Gamma, \delta, q_0, B, F) $, con:
		\begin{align*}
			Q &= \{q_0,q_1,q_2,q_3,q_4,q_5\} \\
			\Sigma &= \{0, 1 \} \\
			\Gamma &= \{X, Y \} \\
			\delta &= \text{Función de transición.} \\
			q_0 &= \text{Estado inicial.} \\
			B &= \text{Blanco}, B \in \Gamma, B \notin \Sigma \\
			F &= q_5 \in Q
		\end{align*}
		La estrategia consiste en reemplazar el primer símbolo por $Y$ para
		poder determinar donde está el inicio de la cinta y no caer de ella, 
		si esté símbolo fue $1$ se pasa a un estado en el que se busca un $0$
		y se le reemplaza por $X$, si fue $0$ se hace lo contrario, una vez
		reemplazado el segundo símbolo se va a un estado que retrocederá en la
		cinta hasta encontrar la $Y$, ahora se lee un $0$ o un $1$ y se
		reemplaza con $X$ y se pasa a los estados que usamos cuando escribimos
		la $Y$. La maquina aceptará cuando, después de retroceder hasta la $Y$
		avance hasta $B$ sin encontrar $1$ o $0$.
		\begin{table}[h]
			\centering
			\begin{tabular}{|c|c|c|c|c|}
				\hline
					Estado actual & Simbolo leído & Símbolo escrito & 
					Movimiento & Estado siguiente \\
				\hline
				\hline
				$q_0$ & 0 & Y & R & $q_1$ \\
				$q_0$ & 1 & Y & R & $q_2$ \\
				\hline 	
				$q_1$ & 0 & 0 & R & $q_1$ \\
				$q_1$ & X & X & R & $q_1$ \\
				$q_1$ & 1 & X & L & $q_3$ \\
				\hline
				$q_2$ & 1 & 1 & R & $q_2$ \\
				$q_2$ & X & X & R & $q_2$ \\
				$q_2$ & 0 & X & L & $q_3$ \\
				\hline
				$q_3$ & 0 & 0 & L & $q_3$ \\
				$q_3$ & 1 & 1 & L & $q_3$ \\
				$q_3$ & X & X & L & $q_3$ \\
				$q_3$ & Y & Y & R & $q_4$ \\
				\hline
				$q_4$ & X & X & R & $q_4$ \\
				$q_4$ & 0 & X & R & $q_1$ \\
				$q_4$ & 1 & X & R & $q_2$ \\
				$q_4$ & B & B & R & $q_5$ \\
				\hline
			\end{tabular}
			\caption{Tabla de transiciones de $M_1$}
			\label{tablatm}
		\end{table}
		
	\item
%		Construya la Máquina de Turing que realice la multiplicación de dos
%		números enteros utilizando una notación binaria adecuada. Defina
%		claramente símbolos y estados. Utilice la tabla \ref{tablatm} para
%		describir la máquina.
%		\begin{table}[h]
%			\centering
%			\begin{tabular}{c|c|c|c|c}
%				\hline
%					Estado actual & Simbolo leído & Símbolo escrito & 
%					Movimiento & Estado siguiente \\
%				\hline
%				\hline
%				& & & & \\
%			\end{tabular}
%			\caption{Tabla de transiciones de la TM}
%			\label{tablatm}
%		\end{table}
		Para construir una Máquina de Turing podemos basarnos en diferentes
		algoritmos de multiplicación binaria, en este caso me basaré en el 
		``Algoritmo de Booth\footnotemark[1]''. Para ello se pedirá la entrada
		de dos numero binarios separados por espacio y se puede seguir el 
		algoritmo como se ve en la página 
		\url{http://morphett.info/turing/turing.html}.
		\footnotetext[1]{
			\url{http://es.wikipedia.org/wiki/Algoritmo_de_Booth}}

		Por otro lado podemos implementar el algoritmo de multiplicación 
		normal. Para ello digamos una Máquina de Turing $M_2 = (Q, \Sigma, 
		\Gamma, \delta, q_0, B, F)$ con:
		\begin{align*}
			\Sigma &= \{0, 1\}\\
			\Gamma &= \{0, 1, \times, =, +\}
		\end{align*}
		La máquina recibe como entrada un string de la forma $1011\times 1001$.
		En primer lugar pondrá el símbolo = al principio de la cinta y luego la
		recorrerá hasta encontrar el ultimo $0$ o $1$. Si el número es $1$
		copiará lo que está entre = y $\times$ al lado izquierdo de la igualdad
		y luego elimina el número. Si es $0$ pondrá tantos ceros como largo
		tenga el string entre el = y el $\times$ y después eliminará el $0$
		(en ambos casos el ultimo elemento en la cinta). Una vez escrito este
		número se agregará un $+$ al principio de la cinta y tantos $0$'s como
		$+$ hallan al lado izquierdo de la igualdad. Este paso se repite hasta
		que no queden más $0$'s o $1$'s después del $\times$.\\
		Una vez hecho todo este proceso puede descartarse la parte a la derecha
		de la igualdad y pasa a ser un problema de suma de números binarios el
		cual puede ser resuelto por otra máquina de Turing.\\
		Tenemos por ejemplo:
		\begin{align*}
			&1011\times 1001 \\
			=&1011\times 1001 \\
			1011=&1011\times 100 \\
			0+1011=&1011\times 100 \\
			00000+1011=&1011\times 10 \\
			00+00000+1011=&1011\times 10 \\
			000000+00000+1011=&1011\times 1 \\
			000+000000+00000+1011=&1011\times 1 \\
			1011000+000000+00000+1011=&1011\times \\
			1011000+000000+00000+1011=&
		\end{align*}
		La cinta de salida de esta máquina se toma como entrada de una máquina
		que sume números binarios y como resultado se tendrá la multiplicación.
		\\ No se escribe la tabla de estados ni la maquina para sumar ya que
		son muy largos.
		% Cambiar esto si hay tiempo %

	\item
%		Sea M la Máquina de Turing tal que 
%		$$M = \left(\{q_0, q_1, q_2, q_f\}, \{0,1\}, \{0,1,B\},
%		\delta,q_0,B, \{q_f\}\right)$$
%		Describa con sus palabras y de forma clara el lenguaje
%		\(\mathcal{L}(M)\) si la función de transición \(\delta\) se define
%		como:
%		\begin{itemize}
%			\item 
%				\(\delta(q_0, 0)=(q_1, 1, R)\);\(\delta(q_1, 1)=(q_2, 0, L)\)
%			\item
%				\(\delta(q_2, 1)=(q_0, 1, R)\);\(\delta(q_1, B)=(q_f, B, R)\)
%		\end{itemize}
		El lenguaje $\mathcal{L}(M)$ aceptado por la Máquina de Turing $M$ será
		$\mathcal{L}(M)= \{w:w=(01)^*\}$ ya que $q_0$ pasa con $0$ a $q_1$
		moviéndose a la derecha y este a su vez pasa a $q_2$ con $1$ moviéndose
		a la izquierda. En $q_2$ se verifica que tengamos un $1$ en el cursor,
		pero esto siempre será así ya que $q_0$ cambió el $0$ por $1$, ahora
		volvemos a estar en el estado inicial y el cursor se movió a la derecha
		por lo que éste proceso puede repetirse infinitas veces. Si se llega a
		$B$ en el estado $q_0$ se pasa a $q_f$ y se acepta.
	\item
%		Esboce cómo puede usarse la no decibilidad del problema de detectar
%		``Hola, mundo'' para demostrar que el determinar si alguna vez se
%		asigna un valor a la variable \texttt{a} en un programa en C no es
%		decidible.
		Sabemos que el problema de decidir si un programa imprime ``Hola
		mundo'' dada cierta entrada es indecidible. Intentaré reducir el
		problema de ``determinar si se le asigna valor a una variable
		\texttt{a} en un programa en C'' al problema de ``Hola mundo''.
		\begin{enumerate}
			\item
				Primero digamos que el programa en C tiene declarada la
				variable \texttt{a} como un arreglo de \texttt{char}.
			\item
				Digamos que se le asigna el valor \texttt{Hola mundo} a la 
				variable \texttt{a}.
			\item
				Asignar un valor a una variable significa que ésta se imprime
				en la memoria.
			\item
				Ahora el problema se reduce a imprimir \texttt{Hola mundo} que
				sabemos es un problema indecidible.
		\end{enumerate}

\end{enumerate}

\vfill\hfill HV/\LaTeXe
\end{document}
