\documentclass[spanish, fleqn]{article}
\usepackage[top = 2.5cm, bottom = 2cm, left = 2cm, right = 2cm]{geometry}
%\usepackage{babel}
\usepackage[spanish, es-noshorthands]{babel}                                    
\usepackage[utf8]{inputenc}
\usepackage[colorlinks, urlcolor=blue]{hyperref}
\usepackage{amsfonts}
\usepackage{amsmath}
\usepackage{wasysym}
\usepackage{fourier}
\usepackage{graphicx}
\usepackage{tikz}
\usepackage{diagbox}
\usetikzlibrary{automata, positioning}
%\definecolor{navyblue}{RGB}{0,148,222}		%Color

\newcommand{\num}{1}
\renewcommand{\thefootnote}{\fnsymbol{footnote}}
\newcommand{\cycle}[2]{\genfrac{[}{]}{0pt}{}{#1}{#2}}	 % Stirling 1a especie

\title{
	Introducción a la Informática Teórica \\
	Tarea \#\num \\
	``La Maldición del Tercer Ojo de Visnú''
}

\author{
	{Hernán Vargas Leighton}, 
	{201073009-3}
}
\date{2 de abril 2014}

\begin{document}
	\maketitle
	\thispagestyle{empty}
	\section*{Respuestas}
	\begin{enumerate}
%		\item
%			Existe una vieja leyenda en la India relatada por varios sabios
%			matemáticos como Aryabhata y Brahmagupta, que si las secuencias de
%			números del 1 al 5 contenían algunos patrones maléficos	creados por
%			el dios Visnú a través de su tercer ojo, podrían sufrir diversos
%			males aquellos individuos que las construyeran.
%			Para nuestra fortuna, los diabólicos patrones son conocidos:
%				\begin{itemize}
%					\item
%						Quien escriba una secuencia con dos \(4's\)
%						consecutivos tendrá una muerte lenta y dolorosa.
%					\item
%						Impar cantidad de \(5's\) traerá la enfermedad y la
%						hambruna.
%					\item
%						Una tabla antigua contiene la siguiente inscripción:
%						¨2 seguido de 3, en TALF estarás otra vez''.
%				\end{itemize}
%			Sólo aquellos que logren construir un DFA que \emph{evite} conjurar
%			estos místicos patrones podrán salvarse. Recuerde mostrar 
%			claramente alfabeto, estados y transiciones (no queremos que le
%			pase algo si se equivoca).

		\item
			Para satisfacer las tres condiciones, necesitamos un autómata capaz
			de detectar las secuencias \(44\) y \(23\), además, debe ser capaz
			de diferenciar cuando el \emph{string} tiene un número par de 
			\(5's\). Para ello digamos:
				\begin{equation*}
					M = ( Q, \Sigma, \delta, q_{0}, F)
				\end{equation*}
			Con:
				\begin{equation*}
					Q = \{a,b,c,d,e,f,g\} \\
					\Sigma = \{1,2,3,4,5\} \\
					q_{0} = a \\
					F = \{a,f,g\} \\
				\end{equation*}
			Y con \(\delta\): función de transición. Con ello tenemos que la 
			tabla de transiciones será:
			  \begin{center}
				\begin{tabular}{|c|r|r|r|r|r|r|r|}
					\hline
					\backslashbox{entrada}{estado} & 
					\(a\)	& \(b\)	& \(c\)	& \(d\)	& \(e\)	& \(f\)	& \(g\)
					\\\hline
					1	& \(a\)	& \(b\)	& \(b\)	& \(b\)	& \(e\)	& \(a\)	& \(a\)
					\\\hline
					2	& \(g\)	& \(d\)	& \(d\)	& \(b\)	& \(e\)	& \(g\)	& \(g\)
					\\\hline
					3	& \(a\)	& \(b\)	& \(b\)	& \(e\)	& \(e\)	& \(a\)	& \(e\)
					\\\hline
					4	& \(f\)	& \(c\)	& \(e\)	& \(c\)	& \(e\)	& \(e\)	& \(f\)
					\\\hline
					5	& \(b\)	& \(a\)	& \(a\)	& \(a\)	& \(e\)	& \(b\)	& \(b\)
					\\\hline
				\end{tabular}
			  \end{center}
			Notar que si se llega al estado \(e\) ya no podremos salvarnos de
			la maldición de Visnú.\\
			Con la tabla hecha nos basta escribir el autómata determinista:
				\begin{center}
					\begin{tikzpicture}[shorten >=1pt, node distance=5cm,
							on grid, >=stealth,	initial text=Inicio,
	%						every state/.style={draw=navyblue!50, very thick,
	%											fill=navyblue!20},
							bend angle=10]
					%%%%	NODOS	%%%%
						\node[state,initial,accepting]	(a)				{$a$};
						\node[state]			(b)[right=of a]			{$b$};
						\node[state]			(c)[above right=of b]	{$c$};
						\node[state]			(d)[below right=of b]	{$d$};
						\node[state]			(e)[right=of b]			{$e$};
						\node[state,accepting]	(f)[above right=of a]	{$f$};
						\node[state,accepting]	(g)[below right=of a]	{$g$};
					%%%%	CONEXIONES	  %%%%
						\path[->]
							%%%%	a	 %%%%
							(a)	edge[loop above]	node		{1,3}	()
							(a)	edge[bend left]		node[left]	{2}		(g)
							(a)	edge[bend left]		node[left]	{4}		(f)
							(a)	edge[bend left]		node[above]	{5}		(b)
							%%%%	b	 %%%%
							(b)	edge[loop right]	node		{1,3}	()
							(b)	edge[bend left]		node[above]	{2}		(d)
							(b)	edge[bend left]		node[left]	{4}		(c)
							(b)	edge[bend left]		node[below]	{5}		(a)
							%%%%	c	 %%%%
							(c)	edge[bend left]		node[below]	{1,3}	(b)
							(c)	edge[bend left]		node[left]	{2}		(d)
							(c)	edge 				node[right]	{4}		(e)
							(c)	edge[bend right]	node[below]	{5}		(a)
							%%%%	d	 %%%%
							(d)	edge[bend left]		node[left]	{1,2}	(b)
							(d)	edge 				node[right]	{3}		(e)
							(d)	edge[bend left]		node[left]	{4}		(c)
							(d)	edge[bend left]		node[above]	{5}		(a)
							%%%%	e	 %%%%
							(e)	edge[loop right]	node	{1,2,3,4,5}	()
							%%%%	f	 %%%%
							(f)	edge[bend left]		node[left]	{1,3}	(a)
							(f)	edge[bend right]	node[left]	{2}		(g)
							(f)	edge[bend left]		node[below]	{4}		(e)
							(f)	edge[bend left]		node[right]	{5}		(b)
							%%%%	g	 %%%%
							(g)	edge[bend left]		node[left]	{1}		(a)
							(g)	edge[loop below]	node		{2}		()
							(g)	edge[bend right]	node[above]	{3}		(e)
							(g)	edge[bend right]	node[left]	{4}		(f)
							(g)	edge[bend right]	node[right]	{5}		(b);
					\end{tikzpicture}
				\end{center}

				Sobre los estados finales podemos decir los siguiente:
				\begin{itemize}
					\item
						Estado \textbf{\(a\):} Para \emph{strings} terminados 
						en 1, 3 o 5 (par).
					\item
						Estado \textbf{\(f\):} Para \emph{strings} terminados
						en 4.
					\item
						Estado \textbf{\(g\):} Para \emph{strings} terminados
						en 2.
				\end{itemize}
				Cabe destacar que los estados \(c\) y \(d\) son variantes para 
				número de \(5\) impar de los estados \(f\) y \(g\) 
				respectivamente, en este caso no se cumplen las condiciones.\\
				Además el estado \(e\) es al cual se cae cuando ya no es 
				posible evitar la maldición.\\
%		\item
%			Visnú creó poderosos talismanes en forma de cadenas formadas por
%			piedras talladas con las iniciales de los dioses Ammavaru, 	Brahma
%			y Chandra. Estas cadenas, hechas para esparcir la maldición, deben
%			tener cantidad par de símbolos, no poseer la inicial de Chandra dos
%			veces seguidas y al menos una vez Ammavaru y Brahma están juntos en
%			dicho orden. Encuentre las cadenas y expréselas como expresión
%			regular.

		\item
			En primer lugar definiré nuestro alfabeto con las iniciales de 
			Ammavaru, Brahma y Chandra:
			\begin{equation*}
				\Sigma = \{A,B,C\}
			\end{equation*}
			Se debe cumplir con:
			\begin{enumerate}
				\item El \emph{string} debe tener un largo par.
				\item El \emph{string} debe contener al menos un \(AB\).
				\item El \emph{string} no debe tener \(CC\).
			\end{enumerate}
			Para cumplir con estas reglas debemos:
			\begin{enumerate}
				\item 
					Construir la expresión en con base en palabras de
					largo 2.
				\item
					Escribir la expresión regular \(AB\). Como su largo es 2 no
					interfiere con la regla \(a)\).
				\item
					Asegurar que ninguna palabra de largo 2 sea \(CC\). Además
					se debe tener la precaución que una palabra terminada en
					\(C\) no esté junto a una palabra que comience con \(C\) y
					viceversa.
			\end{enumerate}
			Podemos construir una expresion regular que cumpla con estas tres
			reglas definiendo las siguientes palabras:
			\begin{equation*}
				a = (A|B)(A|B) \\
				b = CA \\
				c = CB \\
				d = AC \\
				e = BC \\
				f = CC \\
			\end{equation*}
			Inmediatamente notamos que \(a\) serán las palabras sin problemas 
			\emph{incluida} \(AB\). Por otro lado \(b,c,d,e\) son las que
			debemos tener en cuenta y \(f\) no debe estár incluida.\\
			Entonces:
			\begin{itemize}
				\item
					\((a|b|c)\) denota una palabra que \textbf{no termina en
					\(C\)}.
				\item
					\((d|e)^+a\) denota una palabra que 
					\textbf{no comienza con \(C\), no termina en \(C\) y no 
					tiene \(CC\) intermedias pero si puede tener \(C\)}.
				\item
					\( ( (a|b|c) | (d|e)^+a ) ^* \) está compuesta por las
					anteriores y denota toda palabra de largo par sin \(CC\)
					que \textbf{no termine en C}
				\item
					Para aceptar la opción de que el \emph{string} termine en
					\(C\) (sin quebrar las otras reglas) nos basta con agregar
					\((d|e)\) o \((\epsilon)\) al final de la expresión actual.
					\\Convenientemente escribimos: 
					\( (( (a|b|c) | (d|e)^+a ) ^*)(\epsilon | (d|e)) \)
				\item
					Por último, para asegurar que exista al menos un \(AB\)
					agregamos directamente un \(AB\) y lo \emph{rodeamos} por
					la expresion anterior. Como resultante tenemos:
					\( (( (a|b|c) | (d|e)^+a ) ^*)(\epsilon | (d|e)) 
					AB
					 (( (a|b|c) | (d|e)^+a ) ^*)(\epsilon | (d|e))\)
			\end{itemize}
			Ahora nos basta volver al alfabeto inicial. Para ello tenemos que:
			\begin{equation*}
				(a|b|c) = (A|B|C)(A|B) \\
				(d|e) = (A|B)C 
			\end{equation*}
			Además de las definiciones. Con esto tenemos que la expresión 
			regular que cumple con las tres reglas y por lo tanto denota los 
			poderosos talismanes de Visnú es:
			\begin{equation*}
				\Biggl(
					\Bigl(
						\bigl( A|B|C \bigr)
						\bigl(A|B \bigr)
					\Bigr)
					|
					\Bigl(
						\bigl(
							\left( A|B \right)
						C \bigr)^+ 
						\bigl( A|B \bigr)
						\bigl( A|B \bigr) 
					\Bigr)
				\Biggr)^* 
				\Biggl(
					\epsilon 
					| 
					\bigl( A|B \bigr) C
				\Biggr)
				AB
				\Biggl(
					\Bigl(
						\bigl( A|B|C \bigr)
						\bigl(A|B \bigr)
					\Bigr)
					|
					\Bigl(
						\bigl(
							\left( A|B \right)
						C \bigr)^+ 
						\bigl( A|B \bigr)
						\bigl( A|B \bigr) 
					\Bigr)
				\Biggr)^* 
				\Biggl(
					\epsilon 
					| 
					\bigl( A|B \bigr) C
				\Biggr)
			\end{equation*}
%		\item
%			Físicos de alta energía, con la ayuda de  historiadores, aseguran
%			haber encontrado que los rayos maléficos enviados por el tercer ojo
%			de Visnú serían en realidad rayos de ondas gravitatorias, las 
%			cuales fueron representadas bajo una combinación, sin sentido para
%			nosotros, de símbolos grabados en una tabla: \{0, 1\}. Ayude a este
%			equipo a representar esta valiosa información, sabiendo que los
%			rayos contienen una cantidad impar de \(1's\). Sólo pueden haber
%			\(0's\) seguidos de dos \(1's\) y la cadena debe terminar en 1.

		\item
			Las tres condiciones son:
			\begin{enumerate}
				\item Numero impar de \(1's\).
				\item Si hay \(0's\) serán \((011)'s\).
				\item El \emph{string} siempre termina en \(1\)
			\end{enumerate}
			Gracias a las reglas \(a)\) y \(c)\) podemos simplemente buscar un
			\emph{string} de largo par y agregar un \(1\) al final.\\
			Este nuevo \emph{string} estara denotado por:
			\begin{itemize}
				\item \((011)^*\) si está compuesto solo de \((011)'s\).
				\item \(( 1 (011)^* 1 )^*\) si son solo \((11)'s\) o si 
					\((011)'s\) estan rodeados por un \(1\).
			\end{itemize}
			Notar que éstas reglas cumplen con mantener la cantidad par de 
			\(1's\), además:
			\begin{equation*}
				\bigl( 011 | 1 \left( 011 \right)^* 1 \bigr)^*
			\end{equation*}
			Aceptará cualquier conbinación de \((011)'s\) y \(1's\) con un 
			número par de \(1's\). Nos basta solo agregar un \(1\) al final de
			está expresión regular, pero éste \(1\) puede ser en la forma de
			\((1)\) o \((011)\) es decir: \( (1|011) \).\\
			Con esto tenemos que la expresión regular que representa la 
			información de los rayos del tercer ojo de Visnú es:
			\begin{equation*}
				\bigl( 011 | 1 \left( 011 \right)^* 1 \bigr)^*
				\bigl( 1 | 011 \bigr)
			\end{equation*}
			\\

%		\item
%			Ammavaru, Brahma y Chandra, buscando opacar a Visnú y su maldad,
%			codificaron cadenas sagradas de defensa en lo que actualmente
%			conocemos como una expresión regular. Sea \(\mathcal{L}\) el
%			lenguaje de dichas cadenas tal que
%				\(\mathcal{L}(R) = C^{*}(A|BC^{*})^{*}\).
%			Como usted necesita ver gráficamente las posibles cadenas sagradas,
%			construya el DFA que reconoce dichas cadenas.

		\item
			Digamos: \( \Sigma = \{A,B,C\} \), buscamos un autómata finito 
			determinista \(M = ( Q, \Sigma, \delta, q_{0}, F) \) que cumpla 
			con: \(\mathcal{L}(R) = C^{*}(A|BC^{*})^{*}\).\\
			Sabemos que \(C^*\), \(A|BC^*\) y \( (A|BC^* )^* \) son 
			respectivamente:
			
			\begin{center}
				\begin{tikzpicture}[shorten >=1pt, node distance=2cm,
						on grid, >=stealth,	initial text=Inicio,
		%				every state/.style={draw=navyblue!50, very thick,
		%									fill=navyblue!20},
						bend angle=10]
				%%%%	NODOS	%%%%
					%%	C^*	 %%
					\node[state,initial,accepting]	(aa)				{$a$};
					%%	A|BC*  %%
					\node[xshift=1cm, state, initial](ba)[right=of aa]	{$a$};
					\node[state, accepting]		(bb)[above right=of ba]	{$b$};
					\node[state, accepting]		(bc)[below right=of ba]	{$c$};
					%%	(A|BC*)*  %%
					\node[xshift=3cm, state, initial](ca)[right=of ba]	{$a$};
					\node[state, accepting]		(cb)[above right=of ca]	{$b$};
					\node[state, accepting]		(cc)[below right=of ca]	{$c$};
				%%%%	CONEXIONES	  %%%%
					\path[->]
						%%%%	a*	 %%%%
						(aa)	edge[loop above]	node		{$C$}	()
						%%%%	b*	 %%%%
						(ba)	edge[bend left]		node[above left]{$A$} (bb)
						(ba)	edge[bend right]	node[below left]{$B$} (bc)
						(bc)	edge[loop right]	node		{$C$}	()
						%%%%	g*	 %%%%
						(ca)	edge[bend left]		node[above left]{$A$} (cb)
						(ca)	edge[bend right]	node[below left]{$B$} (cc)
						(cb)	edge[loop right]	node		{$A$}	()
						(cb)	edge[bend right]	node[left]	{$B$}	(cc)
						(cc)	edge[loop right]	node		{$C$}	()
						(cc)	edge[loop below]	node		{$B$}	()
						(cc)	edge[bend right]	node[right]	{$A$}	(cb);
				\end{tikzpicture}
			\end{center}

			Con esto nos basta hacer la unión y agregar las \emph{entradas}
			faltantes para obtener el autómata finito determinista resultante:

			\begin{center}
				\begin{tikzpicture}[shorten >=1pt, node distance=3cm,
						on grid, >=stealth,	initial text=Inicio,
		%				every state/.style={draw=navyblue!50, very thick,
		%									fill=navyblue!20},
						bend angle=10]
				%%%%	NODOS	%%%%
					%%	C*(A|BC*)*  %%
					\node[state, accepting, initial]	(a)				{$a$};
					\node[state, accepting]		(b)[above right=of a]	{$b$};
					\node[state, accepting]		(c)[below right=of a]	{$c$};
					\node[state]				(d)[right=of b]	{$d$};
				%%%%	CONEXIONES	  %%%%
					\path[->]
						(a)	edge[loop above]	node		{$C$}			()
						(a)	edge[bend left]		node[above left]	{$A$} (b)
						(a)	edge[bend right]	node[below left]	{$B$} (c)
						(b)	edge[loop above]	node		{$A$}		()
						(b)	edge[bend right]	node[left]	{$B$}		(c)
						(b)	edge				node[above]	{$C$}		(d)
						(c)	edge[loop right]	node		{$C$}		()
						(c)	edge[loop below]	node		{$B$}		()
						(c)	edge[bend right]	node[right]	{$A$}		(b)
						(d)	edge[loop right]	node		{$A,B,C$}	();
				\end{tikzpicture}
			\end{center}
			
			Sobre los estados finales podemos decir los siguiente:
			\begin{itemize}
				\item
					Estado \textbf{\(a\):} Para \emph{strings} compuestos
					exclusivamente de \(C's\).
				\item
					Estado \textbf{\(b\):} Para \emph{strings} terminados en 
					\(A\).
				\item
					Estado \textbf{\(c\):} Para \emph{strings} terminados en
					\(B\) o \(C\).
			\end{itemize}
			Cabe destacar que el estado \(d\) es alcanzado cuando la cadena no
			es sagrada, es decir cuando el \emph{string} no calza con la
			expresión regular \(\mathcal{L}\)\\
			Ahora podemos definir \(M = ( Q, \Sigma, \delta, q_{0}, F) \) con:
			\begin{equation*}
				Q = \{a,b,c,d\} \\
				\Sigma = \{A,B,C\} \\
				q_{0} = a \\
				F = \{a,b,c\} \\
			\end{equation*}
			Y con \(\delta\): función de transición. Con ello tenemos que la 
			tabla de transiciones será:
			  \begin{center}
				\begin{tabular}{|c|r|r|r|r|}
					\hline
					\backslashbox{entrada}{estado} & 
					\(a\)	& \(b\)	& \(c\)	& \(d\)
					\\\hline
					A	& \(b\)	& \(b\)	& \(b\)	& \(d\)	
					\\\hline
					B	& \(c\)	& \(c\)	& \(c\)	& \(d\)	
					\\\hline
					C	& \(a\)	& \(d\)	& \(c\)	& \(d\)	
					\\\hline
				\end{tabular}
			  \end{center}

	\end{enumerate}

\vfill\hfill HV/\LaTeXe
\end{document}
