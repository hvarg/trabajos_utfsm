\documentclass[spanish, fleqn]{article}
\usepackage[english]{babel}
\usepackage[utf8]{inputenc}
\usepackage{amsmath}
\usepackage{amsfonts}
%\usepackage{wasysym}
%\usepackage{mathrsfs}
\usepackage[colorlinks, urlcolor=blue]{hyperref}
\usepackage[top = 2.5cm, bottom = 2cm, left = 2.5cm, right = 2.5cm]{geometry}
\usepackage{fancyhdr, graphicx}
\usepackage{caption}
\usepackage{changepage}
\usepackage{wrapfig}
\usepackage{titling}

\renewcommand{\headrulewidth}{0pt}
\fancyhead[L]{\vbox{\includegraphics[height=2cm]{Logo_INFO.png} \vspace{0.3cm}}}
\fancyhead[C]{
    \vbox{\textsc{\large Universidad Técnica Federico Santa María.}\\[0.14cm]
          \textsc{\LARGE Departamento de Informática.} \\[0.14cm]
          \textsc{\Large ICI 309 - Seminario de Memoria.}
          \noindent\makebox[\linewidth]{\rule{18cm}{0.5pt}} }}
\fancyhead[R]{\vbox{\includegraphics[height=2cm]{Logo_UTFSM.png}\vspace{0.3cm}}}

\title{Análisis de Examen de Título.} 
\author{Hernán Vargas Leighton -- 201073009-3 \\ hernan.vargas@alumnos.usm.cl}
\date{\today}

\let\b\textbf
\let\e\emph
\newcommand{\x}{$\times$}
\begin{document}
\maketitle
\thispagestyle{empty} \thispagestyle{fancy}
\section{Datos principales}
\begin{table}[h!]\centering
    \begin{tabular}{|l|l|}
    \hline
    \b{Nombre del memorista}        & Jorge Maldonado Soto \\ \hline
    \b{Título al que opta}          & Ingeniero Civil Informático \\ \hline
    \b{Fecha}                       & Lunes 16 de Noviembre 2015, 16.00 Hrs 
                                      \\ \hline
    \b{Lugar}                       & Auditorio \e{Claudio Matamoros}, F106 
                                      UTFSM Casa Central, Valparaíso\\ \hline
    \b{Tema}                        & \begin{tabular}{@{}l@{}}
     Animación Facial 3D a partir del seguimiento de un rostro real utilizando\\
     el dispositivo Kinect 
                                      \end{tabular} \\ \hline
    \b{Profesores de la comisión}   & Hubert Hoffmann N. (\b{Guía}) y 
                                      Javier Cañas (\b{Correferente })\\ \hline
    \b{Duración}                    & 50 minutos (30 minutos exposición + 
                                                  20 minutos defensa)\\ \hline
    %\b{Nota final}                 & \\ \hline
    \b{Número de asistentes}        & 27 Personas\\ \hline
    \end{tabular}
\end{table}

\section{Análisis de forma de la presentación}
    Se nota la experiencia del memorista a la hora de presentar en público. Usa
    correctamente su cuerpo y voz para mostrar la importancia de las ideas que
    está planteando, toma el tiempo necesario para explicar correctamente cada
    uno de sus diagramas y ejemplifica las partes más complicadas.

    El orden de la presentación es claramente establecido al comienzo de la
    misma y se cumple a cabalidad, se introducen y desarrollan correctamente los
    temas a tratar y se ejecutan a tiempo. Se finaliza tanto con conclusiones
    generales como especificas y se enmarca el trabajo futuro en esta área.

    El apoyo audio visual es muy preciso, muestra variadas imágenes de como se
    ha desarrollado la tecnología en el ámbito de su presentación y como se
    utiliza actualmente. Presenta diagramas e imágenes de su proyecto y muestra
    resultados finales en un vídeo generado por su aplicación.

    Al responder preguntas muestra sus dudas y no ataca directamente el tema
    preguntado, por lo mismo demora mucho en llegar a la respuesta adecuada y
    generalmente plantea sus opiniones pero no explica las razones de ellas.

\section{Análisis de fondo de la presentación}
    La presentación comienza con un análisis histórico de la tecnología
    utilizada para la animación digital y los grandes exponentes de la misma
    (pixar, disney, etc). El análisis es exhaustivo, pero no se centra en la
    tecnología en sí, si no que aborda temas como el éxito de las compañías y 
    los gastos privados por lo que se pierde parte de la importancia de su
    proyecto y al final le terminan preguntando mucho más sobre los costos y
    alcances de la animación digital en nuestro país que por las ideas
    especificas de su proyecto.

    Al explicar el funcionamiento de su proyecto nos habla del \e{framework} de
    \e{kinect} y como por medio de un \e{script} en \e{python} logra modelar los
    datos para el programa de modelamiento y animación 3D \e{blender}. Esta
    explicación es precisa logra que se entienda por la generalidad del público.
    En la ronda de preguntas explica más exhaustivamente el funcionamiento y por
    qué eligió esas tecnologías contra algunas opciones más estudiadas en la
    universidad (\e{openGL}).

\newpage \newgeometry{left=2.5cm, right=2.5cm, top=2.5cm, bottom=2.5cm}
\section{Análisis usando la rúbrica del MIT}
\begin{table}[h!]\centering\begin{tabular}{|l|c|c|c|c|c|} \hline
    \b{Presentation quality} &\b{Poor} &\b{Fair} &\b{Good} &\b{Exc.}&\b{NA}
                                                                     \\ \hline
    Main objective of presentation is clearly stated       & & & \x & &\\ \hline
    Presenter maintains good eye contact with the audience & & & & \x &\\ \hline
    Presenter uses voice effectively                       & & & & \x & \\ \hline
    Presenter is poised and professional                   & & & & \x & \\ \hline
    Transitions to the next presenter are smooth and effective 
                                                           & & & & & \x \\ \hline
    \b{Technical content} & & & & & \\ \hline
    Technical content is accurate and significant          & & & & \x & \\ \hline
    Technical content shows sufficient development         & & & \x & & \\ \hline
    Main points are emphasized and relationship between ideas is clear
                                                           & & & & \x & \\ \hline
    Ideas are supported with sufficient details and clear drawings
                                                           & & & & \x & \\ \hline
    Graphics and demonstrations are effectively designed and used
                                                           & & & & \x & \\ \hline
    Alternatives are presented with a rationale for those selected
                                                           & & & \x & & \\ \hline
    Key issues are addressed                               & & & & \x & \\ \hline
    Questions are answered accurately and concisely        & & & \x & & \\ \hline
\end{tabular}\end{table}

\section{Comentarios}
    Elegí esta tesis debido a que sólo conocía a Jorge por su labor en la
    Federación de estudiantes UTFSM y quería ver su lado más profesional, por
    lo que el tema de su memoria en un principio me pareció más opcional pero
    luego de escucharla noté como muchas de los conocimientos utilizados para
    realizar su proyecto son trasversales en el área de la ingeniería.

    Uno de los factores que más me impresionó fue lo precisas que fueron las
    preguntas de los profesores. Atacaron directamente los puntos que tenían
    falencias por lo que hicieron dudar al memorista, lamentablemente estos
    puntos eran más que nada sobre como se vendería la tecnología y aspectos más
    que nada comerciales mientras que a mi me interesaba más la parte técnica.

    Por otro lado se notó que los profesores habían tenido reuniones previas a
    la presentación con el memorista por que algunas preguntas eran sobre
    decisiones de elección de tecnologías o del análisis interno de las mismas.
    Estos temas no habían sido explicados completamente, supongo que para no
    hacer la presentación innecesariamente larga, pero gracias a las preguntas
    se lograron incluir entre los tópicos tratados.

%TIEMPO SCT
\begin{flushright}
    \begin{flalign*}
        &&\text{\textbf{Tiempo SCT}:
              \emph{Planificación}}               &= 0:15 \\
            &&\emph{Búsqueda de información}      &= 0:20 \\
            &&\emph{Análisis}                     &= 1:30 \\
            &&\emph{Desarrollo}                   &= 2:20 \\
            &&\emph{Edición}                      &= 0:20 \\
            &&\text{Total}                        &= 4:45
    \end{flalign*}
\end{flushright}

\vfill\hfill  HVL / \LaTeXe

\end{document}
