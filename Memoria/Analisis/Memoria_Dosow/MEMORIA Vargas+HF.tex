\documentclass[spanish, fleqn]{article}
\usepackage[english]{babel}
\usepackage[utf8]{inputenc}
\usepackage{amsmath}
\usepackage{amsfonts}
%\usepackage{wasysym}
%\usepackage{mathrsfs}
\usepackage[colorlinks, urlcolor=blue]{hyperref}
\usepackage[top = 2.5cm, bottom = 2cm, left = 2.5cm, right = 2.5cm]{geometry}
\usepackage{fancyhdr, graphicx}
\usepackage{caption}
\usepackage{changepage}
\usepackage{wrapfig}
\usepackage{titling}

\renewcommand{\headrulewidth}{0pt}
\fancyhead[L]{\vbox{\includegraphics[height=2cm]{figures/di} \vspace{0.3cm}}}
\fancyhead[C]{
    \vbox{\textsc{\large Universidad Técnica Federico Santa María.}\\[0.14cm]
          \textsc{\LARGE Departamento de Informática.} \\[0.14cm]
          \textsc{\Large ICI 309 - Seminario de Memoria.}
          \noindent\makebox[\linewidth]{\rule{18cm}{0.5pt}} }}
\fancyhead[R]{\vbox{\includegraphics[height=2cm]{figures/utfsm}\vspace{0.3cm}}}

\title{Análisis de Memoria.} 
\author{Hernán Vargas Leighton -- 201073009-3 \\ hernan.vargas@alumnos.usm.cl}
\date{\today}

\let\b\textbf
\let\e\emph
\newcommand{\x}{$\times$}
\newcommand{\tbr}[2][c]{%
      \begin{tabular}[#1]{@{}c@{}}#2\end{tabular}}
\newcommand{\tbrc}[2][c]{%
      \begin{tabular}[#1]{@{}l@{}}#2\end{tabular}}
\begin{document}
\maketitle
\thispagestyle{empty} \thispagestyle{fancy}
\section{Datos principales}
\begin{table}[h!]\centering
    \begin{tabular}{|l|l|}
    \hline
    \b{Nombre del estudiante}      & Yonathan Helmuth Dossow Acuña  \\ \hline
    \b{Título al que opta}         & Ingeniero Civil en Informática \\ \hline
    \b{Fecha}                      & Junio del 2014                 \\ \hline
    \b{Tema}                       & Propuesta de migración a IPv6 para la UTFSM
                                                                     \\ \hline
    \b{Profesores de la comisión}  & 
        \begin{tabular}{@{}ll@{}}
            \b{Guía:} & Javier Cañas \\
            \b{Correferente:} & Horst von Brand
        \end{tabular} \\ \hline
    \b{Número de páginas}          &
        \tbrc{\b{Documento:} 118\\ \b{Escrito:} 59 \\ \b{Anexos:} 39}\\ \hline
    \b{Capítulos}                  &
        \begin{tabular}{@{}l@{}}
            1.- Introducción \\
            2.- Estado del arte \\
            3.- Implementación de IPv6 \\
            4.- Pruebas y resultados \\
            5.- Conclusiones 
        \end{tabular} \\ \hline
    \end{tabular}
\end{table}

\section{Análisis de forma del escrito}
    La memoria analizada está correctamente seccionada de manera de facilitar su
    lectura. Comienza con un índice general seguido por el de las figuras y por
    último el de los anexos lo que facilita enormemente la labor de encontrar
    información especifica dentro del documento.

    La versión analizada por mi fue la digital en formato \texttt{pdf} en ella
    pude notar que el documento y gran parte de las figuras fueron hechas en
    \texttt{latex} y por ello están vectorizadas, lo que significa que puedo
    hacer \e{zoom} sin perder calidad, esto es fundamental a la hora de analizar
    diagramas de red como los presentados, en un libro físico podemos perder
    parte de la información de una red muy grande por las limitaciones de
    espacio, acá leer todas las direcciones con facilidad gracias a esta
    característica. 

    Por otro lado, al utilizar este tipo de documento el formato de letra,
    imágenes, tablas y demás es consistente a través de todo el documento.
    Además se utiliza un formato ideal para la impresión ya que genera margenes
    diferentes para las hojas que están a la derecha y a la izquierda,
    impidiendo que los margenes se pierdan por la unión de las paginas, por
    último la letra \e{serif} que utiliza es muy clara y evita distracciones y
    es a la que estamos acostumbrados a leer cuando vemos documentos técnicos.

    Se utiliza correctamente las referencias, citas y bibliografía en general.
    Cuando es necesario se utilizan notas al pie de pagina para evitar
    confusiones y cada diagrama tiene una etiqueta representativa y un número
    utilizado para referirse a ella.

\newpage\newgeometry{left=2.5cm, right=2.5cm, top=2.5cm, bottom=2.5cm}
\section{Análisis de fondo del escrito}
    En esta sección se analiza el fondo del documento detallado por capitulo del
    mismo, se da una pequeña descripción y se presentan comentarios generales.
    \subsection{Introducción}
    Se plantean los objetivos del escrito, se identifica el problema y se
    enuncia como será tratado. 

    Es un capitulo para entender las razones por las cuales se debería leer el
    documento y su por carácter general es de especial importancia para aquellos
    que no están tan familiarizados con el tema pues les indica en palabras
    simples de que trata.

    \subsection{Estado del arte}
    Se introducen los tópicos a tratar en el escrito, se define \texttt{IPv4}
    e \texttt{IPv6} y se explican sus características y funcionamiento. Por otra
    parte se analizan los problemas actuales y algunos métodos utilizados
    actualmente para atacarlos, se contextualiza la situación para su
    implementación dentro de la UTFSM y se analizan las aplicaciones que están
    sujetas a los posibles cambios.

    En este capitulo podemos ver todos los avances que ya existen en el tema,
    muy útil para informarse de las tecnologías actuales y las limitantes
    conocidas. Especialmente relevante para aquellos que tienen conocimientos
    técnicos del área pero no se especializan en el tópico de este documento por
    que genera un punto de comparación con otros conocimientos y es una buena
    forma de sumergirse en los capítulos siguientes

    \subsection{Implementación de IPv6}
    El capitulo más largo pues es el que presenta todo el desarrollo técnico
    realizado por el memorista. Se enuncian diferentes propuestas de soluciones
    para los mismos problemas enumerando las ventajas de cada una y como deben
    ser implementadas. 

    Esta es la parte medular de la memoria, presenta mucha información
    especifica y relacionada al problema que se está tratando. El análisis de
    las posibles soluciones es completo y detallado y explica por medio de
    diagramas como funcionan algunas partes del mismo, lo que hace muy fácil el
    entendimiento. Especialmente útil para comprender como funcionan los
    servicios de \texttt{Dual Stack} y como trabajar con ellos.

    \subsection{Pruebas y resultados}
    En este capitulo se hacen comparativas y gráficas para poder notar los
    cambios entre \texttt{IPv4} e \texttt{IPv6}. 

    Desde un inicio se ve que esta sección es la que presenta mayor cantidad de
    diagramas ya que son muy útiles a la hora de compara comportamientos y
    gracias a su buen uso se puede expresar mucha información en poco espacio lo
    que permite al lector poder sacar sus propias conclusiones y no quedarse
    solamente con lo planteado por el autor.
 
    \subsection{Conclusiones}
    Se concluyen los elementos claves vistos a lo largo de la memoria. Podemos
    ver diferentes tipos de conclusiones: las prácticas que hablan directamente
    de lo que se hizo y sus resultados; las generales que abarcan el tema
    completamente y las que guían el trabajo futuro, para que otros estudiantes
    puedan comenzar su trabajo con base en estos puntos.

    \subsection{Anexos}
    Si bien esta sección no es un capitulo, considero que es muy importante pues
    gran parte del trabajo realizado se presenta en forma de configuraciones de
    programas y código en general, por lo cual poder analizarlo en profundidad
    contribuye mucho al entendimiento de como utilizar las tecnologías de
    \texttt{IPv6} correctamente, en especial para otros estudiantes o personas
    que buscan solución a problemas específicos.

\newpage 
\section{Análisis usando la rúbrica del MIT}
\begin{table}[h!]\centering\begin{tabular}{|l|c|c|c|c|c|} \hline
    &\tbr{\b{Needs}\\\b{Improvement}} &\b{Acceptable} &\b{Good} 
    &\tbr{\b{Very}\\\b{Good}} &\tbr{\b{Not}\\\b{Applicable}} \\ \hline
    \tbrc{Main objective is easily identified and\\supported by the content}
                                                           & & & & \x &\\ \hline
    \tbrc{Content, Structure, and language are\\ geared to the intended audience}
                                                           & & & & \x &\\ \hline
    \tbrc{Main points are emphasized and the\\relationship between ideas is clear}
                                                           & & & & \x &\\ \hline
    \tbrc{Arguments area clearly supported and\\in sufficient detail}
                                                           & & & \x & &\\ \hline
    Content is properly footnoted                          & & & & \x &\\ \hline
    Conclusions are valid and reasonable                   & & & & \x&\\ \hline
    \tbrc{Significance and implications area clearly\\addressed}
                                                           & & & & \x &\\ \hline
    Writing is clear, organized, and coherent              & & & & \x &\\ \hline
    \tbrc{A sufficient number of appropriate\\sources are cited}
                                                           & & & & \x &\\ \hline
    \tbrc{Attention is given to grammar\\punctuation, spelling and formatting}
                                                           & & & & \x &\\ \hline
\end{tabular}\end{table}

\section{Reflexión y comentarios}
    Hace algún tiempo tomé el ramo "Taller de redes computacionales" dictado
    por Javier Cañas, mismo profesor que fue el guía de Dossow en esta memoria.
    En el desarrollo de un proyecto para dicho ramo decidimos hacer una
    implementación de los protocolos \texttt{IPv4} e \texttt{IPv6} en
    \texttt{sockets} en \texttt{c}, ante lo cual el profesor nos recomendó leer
    la memora analizada en este documento. Es una memoria fácil de leer de un
    tema en el cual he trabajado y me interesa, por ello la considero útil ya
    que en su tiempo me ayudo bastante con otro ramo y así decidí analizar
    la misma acá.

    Creo que la importancia de la memoria radica en lo expresado anteriormente,
    más allá de ser un requisito para titularse es un documento que culminará
    nuestro paso por la universidad y por ello debemos cuidar que sea algo útil
    para las generaciones venideras, de esta manera nuestro último trabajo como
    estudiantes se transformará en parte del aprendizaje de los más jóvenes e
    influirá en su desarrollo como profesional.
    
%TIEMPO SCT
\begin{flushright}
    \begin{flalign*}
        &&\text{\textbf{Tiempo SCT}:
              \emph{Planificación}}               &= 0:20 \\
            &&\emph{Búsqueda de información}      &= 0:30 \\
            &&\emph{Análisis}                     &= 4:30 \\
            &&\emph{Desarrollo}                   &= 3:00 \\
            &&\emph{Edición}                      &= 0:25 \\
            &&\text{Total}                        &= 8:45
    \end{flalign*}
\end{flushright}

\vfill\hfill  HVL / \LaTeXe

\end{document}
