\documentclass[spanish, fleqn]{article}
\usepackage[english]{babel}
\usepackage[utf8]{inputenc}
\usepackage{amsmath}
\usepackage{amsfonts}
\usepackage[colorlinks, urlcolor=blue]{hyperref}
\usepackage[top = 2.5cm, bottom = 2cm, left = 2.5cm, right = 2.5cm]{geometry}
\usepackage{fancyhdr, graphicx}
\usepackage{caption}
\usepackage{changepage}
\usepackage{wrapfig}
\usepackage{titling}

\renewcommand{\headrulewidth}{0pt}
\fancyhead[L]{\vbox{\includegraphics[height=2cm]{figures/di} \vspace{0.3cm}}}
\fancyhead[C]{
    \vbox{\textsc{\large Universidad Técnica Federico Santa María}\\[0.14cm]
          \textsc{\LARGE Departamento de Informática} \\[0.14cm]
          \textsc{\Large ICI-309 -- Seminario de Memoria}
\noindent\makebox[\linewidth]{\rule{18cm}{0.5pt}} }}
\fancyhead[R]{\vbox{\includegraphics[height=2cm]{figures/utfsm}\vspace{0.3cm}}}

\title{Fichas Bibliográficas} 
\author{Hernán Vargas Leighton -- 201073009-3 \\ hernan.vargas@alumnos.usm.cl}
\date{\today}

%17:20

\begin{document}
\maketitle
\thispagestyle{empty}
\thispagestyle{fancy}

\section{Bibliografía revisada}
\subsection{Linked data-the story so far\cite{bizer2009linked}}
\subsubsection*{Descripción:}
Este paper presenta las bases para un estudio de la web de datos enlazados
(\emph{linked data} en ingles), comienza planteando la problemática, hace un
recorrido histórico e introduce las tecnologías de la web semántica junto a
diversas herramientas relacionadas.

Uno de los principales hitos para el desarrollo de la web como la conocemos fue
la creación de buscadores e indices que facilitan enormemente la navegación para
los usuarios, pero estos sistemas de búsqueda y relación son deficientes a la
hora de ser tratados por una maquina. Esta es la problemática que intenta
resolver la web semántica: generar enlaces entre los datos de la web para que
estos puedan ser recuperados por consultas hechas por un ordenador.

El documento introduce el concepto de URI (\emph{Uniform Resource Identifiers}),
RDF (\emph{Resource Description Framework}) y los \emph{Linked data principles}
que son reglas para enlazar datos en la web enunciadas por 
Berners-Lee\cite{berners2011linked} como siguen:
\begin{itemize}
    \item Usar URIs para nombrar las cosas del mundo.
    \item Las URIs deben ser HTTP para que la gente pueda ver el contenido.
    \item Proveer información útil, utilizar los estandares (RDF, SPARQL).
    \item Incluir enlaces hacia otras URIs para poder navegar entre ellas.
\end{itemize}

Se continua con una explicación de como se crea y que compone un triple RDF 
($\langle sujeto, predicado, objeto \rangle$), los vocabularios básicos para
describir el mundo y sus relaciones (RDFSchema y OWL) e información sobre el
proyecto \emph{linked open data project} y algunos de sus más importantes
participantes como son la DBpedia, datos gubernamentales y varios datasets que
también son parte del proyecto Bio2RDF.

Finalmente se dan consejos para publicar datos en la web (creación del modelo,
incluir metadata, etc) se enumeran algunas herramientas de publicación (D2R
Server, Virtuoso Universal Server, etc) y se muestran algunas aplicaciones para
visualizar estos datos (navegadores, buscadores, indices y aplicaciones
especializadas como la BDpedia mobile).

\subsubsection*{Utilidad:}
Este paper presenta los conceptos básicos para cualquier investigación en el
campo de la web semántica y por ello facilita la comprensión de los tópicos a
tratar en el desarrollo de la memoria, específicamente lo relacionado a las
tecnologías de la web semántica como son RDF, RDFS, OWL, SPARQL, etc.

\newpage \newgeometry{left=2.5cm, right=2.5cm, top=2.5cm, bottom=2.5cm}

\subsection{Bio2RDF: towards a mashup to build bioinformatics knowledge systems\cite{belleau2008bio2rdf}}
\subsubsection*{Descripción:}
Este documento introduce el proyecto Bio2RDF, plantea la existencia de variados
datasets biológicos y la necesidad de hacer un \emph{mashup} de ellos pues para
cada dataset el dominio y las relaciones difieren.

Se señalan las tecnologías utilizadas para hacer esta integración (los
\emph{``rdfizers''}, OWL, sesame, etc), el modelo de obtención de datos, la
creación de un dataset común y la generación de la vista por parte del usuario.
También se introducen los datasets participantes (Kegg, PDB, MGI, HGNC, etc) y
se presenta un caso de uso a modo de ejemplo (Parkinson).

El proyecto Bio2RDF genera una antología común para todos los datasets que lo
componen y normaliza las URIs de la siguiente manera: 
\texttt{https://bio2rdf.org/<namespace>:<identifier>} por ejemplo, para una
entidad del dataset pubmet tenemos \url{http://bio2rdf.org/pubmed:12728276}

\subsubsection*{Utilidad:}
La memoria utiliza como data principal las consultas SPARQL hechas a este
proyecto por lo que saber sus bases y funcionamiento se hace imperativo.

\subsection{Bio2rdf release 2: Improved coverage, interoperability and provenance of life science linked data\cite{callahan2013bio2rdf}}
\subsubsection*{Descripción:}
Este paper es un estudio de la revisión hecha al proyecto Bio2RDF para su
\emph{release} 2. Enumera las problemáticas encontradas en la versión previa y
las mejoras hechas para esta versión.

Entre las características de la versión 2 se encuentran la capacidad de acceder
a un \emph{endpoint} SPARQL, descargar el dataset completo, mejoras en el
\emph{rdfizer}, la generación automática de metadata (sobre la generación de
datos), entre otras.
Además se generaron métricas del proyecto que pueden ser consultadas con SPARQL 
sobre \texttt{http://bio2rdf.org/bio2rdf-ctd-statistics}. El paper incluye
ejemplos de la utilización de las mismas.

\subsubsection*{Utilidad:}
Al igual que el documento anterior, la información del proyecto Bio2RDF es
importante para poder entender el análisis que se realizará en la memoria. Se
incorporan métricas y metadata que puede ser consultada facilitando el posterior
análisis. 

\subsection{Graph structure in the web\cite{broder2000graph}}
\subsubsection*{Descripción:}
Este estudio analiza la web como un grafo. Para generar los grafos utiliza 
\emph{web crawling algorithms} sobre datos de altavista (cerca de 200 millones
de páginas web y 1.500 millones de enlaces entre ellas). Una vez creado el
modelo de grafo se procede a analizar los resultados y recopilar estadísticas
como el \emph{incoming degree}, el \emph{outgoing degree}, etc.

Entre los resultados notables se obtiene que los enlaces entre páginas web
siguen una distribución de probabilidad según una ley potencial (\emph{power
law}), hecho que será una de las bases de estudios posteriores.

\subsubsection*{Utilidad:}
Determinar como se comportan los datos en la web ``normal'' nos presenta algunas
ideas de como deberían comportarse los datos enlazados de un dataset RDF. Los
resultados de este paper se contrastarán con el siguiente.

\subsection{Link Analysis of Life Science Linked Data\cite{hu2015link}}
\subsubsection*{Descripción:}
Documento principal para el desarrollo de la memoria. En este paper se estudia
el proyecto Bio2RDF como un grafo y con ello se generan estadísticas (como las 
de ~\cite{broder2000graph}). Se introducen varios términos importantes para el
análisis y se enfoca el mismo en tres perspectivas:
\begin{itemize}
    \item 
        \textbf{Dataset link analysis:} Se estudia el grafo de enlaces entre
        los datasets que componen el proyecto (los nodos son los datasets y los
        arcos son la cantidad de enlaces entre ellos). Se obtienen como 
        estadísticas el componente más conectado, la distancia media más larga,
        la cantidad de entidades por dataset, etc.

        Entre los resultados notables tenemos que se produce el fenómeno de
        mundo pequeño (\emph{small world phenomenom}) lo que denota una gran
        conectividad y por otro lado que las entidades están normalmente
        distribuidas entre los datasets.
    \item
        \textbf{Entity link analysis:} Se estudian las relaciones entre las
        entidades de los datasets. Como la relación más común es la llamada
        \texttt{x-relation} (cerca del $76\%$ del total) se genera un grafo
        utilizando ésta como arco. Se investiga el grado de distribución,
        la simetría, la transitividad, entre otros.

        Se concluye que tanto el \emph{incoming degree} como el \emph{outgoing
        degree} no siguen una distribución según una ley potencial (a diferencia
        de los resultados de ~\cite{broder2000graph}). Además la simetría se da
        generalmente entre pares del mismo dataset y la transitividad se
        debilita a medida que se hacen más largos los caminos a recorrer por la
        misma; se considera que estos resultados son debido a la divergencia
        entre los modelos de cada dataset.

    \item
        \textbf{Term link analysis:} Para este estudio se utilizó como nodos los
        términos y como arcos si existe un \emph{mapping} entre dos de ellos.

        Se obtuvo como resultado que la mayoría de los \emph{mappings} pueden
        ser encontrados por comparación lingüística (\emph{string comparison})
        y se generó estadística sobre los términos más utilizados: Para
        \emph{Class} sería \emph{Resource} y \emph{Gene}, para \emph{Object
        properties} tenemos \emph{x-uniprot} y \emph{x-ncbigene} y para
        \emph{Data properties} se obtuvo \emph{synonym} y \emph{definition}.
\end{itemize}

\subsubsection*{Utilidad:}
Este es uno de los documentos más importantes de la memoria pues resume varios
tópicos fundamentales al tratar un conjunto de datasets RDF como un grafo. Los
resultados obtenidos en este trabajo son el punto de partida de la investigación
sobre las consultas SPARQL hechas por los usuarios al proyecto Bio2RDF y se
espera que los resultados de la memoria sean complementarios a este estudio y
determinen como se utilizan los enlaces analizados aquí.

Por otro lado este estudio nos permite notar que existen diferencias 
sustanciales entre un modelo de grafos RDF y uno generado por \emph{web
crawling} sobre la web normal. Como sabemos que generalmente las páginas web
enlazan la información de manera que sea accesible por los usuarios se hace
interesante investigar como utilizan los mismos la base de datos RDF para
recuperar la información del proyecto Bio2RDF.


%Los resultados obtenidos en este paper serán una de las principales fuentes de
%datos para el desarrollo de la memoria y puede ser consultada en:
%\url{http://es.nju.edu.cn/bio2rdf-analysis/}

%\section{Bibliografía recomendada}
%\subsection{On graph features of semantic web schemas}
%\subsection{Power-law distributions in empirical data}
%\subsection{Relatedness between vocabularies on the Web of data: A taxonomy and an empirical study}
%\section{SameAs networks and beyond: analyzing deployment status and implications of owl: sameAs in linked data}

\renewcommand{\refname}{\selectfont Referencias.} % Hack para nombre
\bibliographystyle{ieeetr}
\bibliography{bibliografia}

%TIEMPO SCT
\begin{flushright}
    \begin{flalign*}
        &&\text{\textbf{Tiempo SCT}: 
                \emph{Planificación}}       &= 0:20 \\
              &&\emph{Creción de informe}   &= 3:40 \\
              &&\emph{Edición}              &= 0:40 \\
              &&\text{\textbf{Total}}       &= 4:40 \\
              &&\text{Lectura de papers}    &\simeq 8:00
    \end{flalign*}
\end{flushright}

\end{document}
