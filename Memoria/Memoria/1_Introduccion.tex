%!TEX root = main.tex

\chapter{Introducción}

\section{Identificación del problema}
El proyecto Bio2RDF (en su versión 3) incorpora la información de 35 datasets
RDF biológicos, cerca de 11.000 millones de triples que conforman la red de data
enlazada más grande de esta ciencia.

Debido al volumen de datos que maneja este proyecto es interesante determinar
cual es la información más consultada por parte de los usuarios y verificar que
ésta tenga el soporte adecuado por parte del modelo. Así, se hace indispensable
contar con métricas del uso de los datos que incorpora el proyecto. 

Si bien en trabajos anteriores como el de Hu et al.\cite{hu2015link} se han
determinado parámetros como el grado de distribución o la simetría y la
transitividad de los enlaces entre los datos, no se ha hecho un análisis de
centralidad de este proyecto.

Por otro lado, tampoco existe un estudio que determine cual es el subconjunto de
datos que realmente son consultados por los usuarios y por ello no es posible
verificar que parte del dataset de Bio2RDF es la más importante.

\section{Objetivos}
