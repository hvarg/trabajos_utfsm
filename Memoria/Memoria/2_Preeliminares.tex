%!TEX root = main.tex

\chapter{Estado del arte}
El proyecto Bio2RDF utiliza las tecnologías de la web semántica para su
implementación y funcionamiento. Así, el análisis que se hará en esta memoria
toma estos conocimientos como base y generará a través de ellos estadísticas de
centralidad con las cuales se determinará cuales son los datos más importantes
del \emph{dataset}.

Para poder analizar el problema correctamente es necesario el conocimiento de
las tecnologías que se enuncian en este capítulo. En la sección \ref{ea:ws} se
describe qué es la web semántica y las tecnologías que la componen.
En la sección \ref{ea:bio} se hace una revisión al proyecto Bio2RDF 
%y en la
%sección \ref{ea:cent} se introduce el concepto de centralidad junto a sus
%métricas y algoritmos.
%TODO

\section{Web Semántica}\label{ea:ws}
La web semántica es un conjunto de actividades propuestas por la \emph{World
Wide Web Consortium} (desde ahora W3C) con el objetivo de generar tecnologías
para la publicación de datos en la web de tal manera que sean procesables por
las maquinas. 
Se basa en la idea de añadir metadatos semánticos y ontológicos que describan el
contenido y la relación entre los datos publicados. De esta manera se logra
mejorar la interoperatividad de internet pues los programas podrán acceder a los
datos en un lenguaje formal, procesar su contenido, razonar en base a este y
combinarlo para resolver problemas cotidianos automáticamente.

Es posible rastrear los orígenes de esta idea hasta una propuesta temprana de la
\emph{world wide web} en 1986 ~\cite{berners1989proposal} y el subsecuente
trabajo de Berners-Lee et al.\cite{berners1992world} donde se prevé la
necesidad de una evolución desde objetos legible por las personas a información
semántica orientada a las máquinas.

Para lograr los objetivos de la web semántica se han generado múltiples
metalenguajes y estándares de representación como son XML, XML Schema, RDF,
RDF Schema, OWL y SPARQL. Estas tecnologías se utilizan actualmente para generar
datos enlazados (\emph{Linked Data}), que buscan enlazar información arbitraria
en la web generando así una ``red de las cosas del mundo, descrita por los datos
en la Web'' \cite{berners2011linked}.

\begin{figure}[htpb]
  \centering
  \includegraphics[width=.6\textwidth]{figures/Semantic_web_stack.png}
  \caption{Pila de tecnologías de la web semántica.}
  \vspace{-.25cm}
  \caption*{Creado por Marobi1\cite{wikimg:swstack}.}
  \label{fig:swstack}
\end{figure}

En las siguientes secciones se describirán más detalladamente algunos de los
conceptos presentados en la figura \ref{fig:swstack} comenzando desde la base,
con las tecnologías del hipertexto (URI y XML), para luego pasar a las
tecnologías de la web semántica más estandarizadas (RDF, RDFS y OWL) y terminar
con el lenguaje de consulta SPARQL.

\subsection{URI}
Una URI (del las siglas del ingles \emph{Uniform Resource Identifier}) es una
cadena de caracteres utilizada para identificar un recurso inequívocamente.
Si bien en un principio solo acepta un subconjunto caracteres ASCII (el alfabeto
en ingles y los números) y código porciento (por ejemplo '\%26' es '\&') el
estándar fue extendido el 2005 a IRI (\emph{Internationalized resource
identifier}) que puede contener caracteres Unicode/ISO 10646, incluyendo chino,
japones, koreano, entre otros\cite{gangemi2006bourne}. En este documento no se
diferenciará entre URI e IRI.

La sintaxis actual de una URI fue definida por Berners-Lee et al. el
2005 ~\cite{berners2004uniform} como sigue:\\
\texttt{
  scheme:{[//{[user:password@]}host{[:port]}]}{[/]}path{[?query]}{[\#fragment]}
}\\
donde:
\begin{itemize}
  \item
    El \bf{esquema} (\emph{scheme}) que consiste en una secuencia de caracteres
    sensibles a las mayúsculas que comienza con una letra y es seguido por
    cualquier combinación de letras, puntos (\tt{.}), guiones (\tt{-}) y signos
    más (\tt{+}), terminando con dos puntos (\tt{:}). 
    Ejemplos populares son \tt{http}, \tt{ftp} y \tt{mailto}.
  \item
    Las dos barras (\tt{//}) son requeridos para algunos esquemas. Cuando la
    componente de autoridad no está presente la ruta no puede comenzar con dos
    barras.
  \item
    La \bf{autoridad} se divide como sigue:
    \begin{itemize}
      \item
        Una \bf{autentificación} opcional compuesta por el usuario
        (\emph{user}), dos puntos y una contraseña (\emph{password}) terminando
        en un arroba (\tt{@}).
      \item
        Un anfitrión (\emph{host}) que puede ser un nombre registrado un una
        dirección IP.
      \item
        Un número de \bf{puerto} opcional (\emph{port}) separado del anfitrión
        por dos puntos.
    \end{itemize}
  \item
    Una \bf{ruta} (\emph{path}) que es una secuencia de segmentos separados
    por una barra (\tt{/}) que comienza con una de las mismas. Generalmente
    representa la localización de un archivo en un sistema de archivos
    jerárquico pero no es necesario.
  \item
    Una \bf{consulta} opcional (\emph{query}) que comienza con un signo de
    interrogación (\tt{?}) y, a pesar de no tener un sintaxis claramente
    definida, usualmente presenta pares atributo-valor.
  \item
    Un \bf{fragmento} opcional (\emph{fragment}) que comienza con un numeral
    (\tt{\#}) seguido por un identificador secundario al recurso principal.
\end{itemize}

\begin{figure}[htpb]
  $$
    \underbrace{\text{http}}_{\text{esquema}}\text{://}
    \overbrace{
      \overbrace{
        \underbrace{\text{user:password}}_{\text{autentificación}}\text{@}
        \underbrace{\text{example.com}}_{\text{anfitrión}}\text{:}
        \underbrace{\text{80}}_{\text{puerto}}
      }^{\text{autoridad}}
      \overbrace{\text{/path/data}}^{\text{ruta}}
    }^{\text{parte jerárquica}}\text{?}
    \underbrace{\text{key=value}}_{\text{consulta}}\text{\#}
    \underbrace{\text{section1}}_{\text{fragmento}}
  $$
  \caption{Ejemplo de una URI.}
  \label{fig:uriex}
\end{figure}

Un ejemplo de lo anterior sería la figura \ref{fig:uriex} donde podemos notar
que una URL (\emph{Uniform Resource Locator}) es una URI, pero no debemos
confundir los términos pues una URL además de identificar un recurso web permite
obtener una representación del mismo generalmente en formato HTML vía HTTP. Es
decir, una URL es una URI que apunta a un recurso físico en la web, mientras que
una URI no necesariamente debe apuntar a una localización que exista realmente.

\subsection{XML}
XML es el acrónimo del ingles \emph{Extensible Markup Language}. Es un
lenguaje de marcas desarrollado por el W3C para almacenar datos de manera
legible tanto por personas como por máquinas. 
El diseño de XML busca la simplicidad, generalidad y usabilidad a través de
internet\cite{paoli2004extensible}. 

Un documento XML puede comenzar con un identificador declarando alguna
información sobre el documento en sí. Luego el cuerpo del fichero debe contener
solo un elemento raíz y dentro de éste se pueden escribir múltiples elementos y
atributos.

\begin{figure}[htpb]
  \centering
  \begin{tabular}{c}
    \lstinputlisting[
      basicstyle=\ttfamily\scriptsize,
      language=xml]{Codigo/example-xml.xml}
  \end{tabular}
  \caption{Ejemplo de XML.}
  \vspace{-.25cm}
  \label{fig:xmlex}
\end{figure}

En la figura \ref{fig:xmlex} vemos la libertad que nos da XML a la hora de
representar cualquier tipo de información, pero esta libertad hace que
interpretar la validez de un documento sea complicado, pues la información
contenida en este puede ser sensible a la aplicación y el lenguaje definido con
esta tecnología. En respuesta a esta problemática el W3C desarrolló \emph{XML
Schema} que cobra vital importancia para la web semántica.


\subsection{XSD}
XSD (del ingles \emph{XML Schema Definition}) es una recomendación del W3C que
especifica formalmente la estructura y restricciones de los contenidos de un
fichero XML de manera precisa más allá de las normas sintácticas del propio
lenguaje XML.

A diferencia de otros lenguajes de esquemas para XML, XSD especifica también
tipos de datos y sus restricciones, logrando un estándar a la hora de
representar datos\cite{biron2004xml}. 
Presenta 19 tipos de datos básicos: \tt{anyURI},
\tt{base64Binary}, \tt{boolean}, \tt{date}, \tt{dateTime}, \tt{decimal},
\tt{double}, \tt{duration}, \tt{float}, \tt{hexBinary}, \tt{gDay}, \tt{gMonth},
\tt{gMonthDay}, \tt{gYear}, \tt{gYearMonth}, \tt{NOTATION}, \tt{QName},
\tt{string} y \tt{time}. Además permite la creación de nuevos tipos de datos por
medio de tres mecanismos:
\begin{itemize}
  \item \bf{Restricción:} Reduce los valores que puede tomar un dato.
  \item \bf{Lista:} Permite una secuencia de valores.
  \item \bf{Unión:} Permite la elección de valores de diferentes tipos.
\end{itemize}

Gracias a estos factores un fichero XML Schema puede representar un modelo de
datos completo y robusto, con relaciones entre las entidades y asignaciones de 
tipos de datos básicos.
Esta característica es fundamental para la web semántica pues todos los tipos de
datos de XSD son compatibles con RDF.

\begin{figure}[htpb]
  \centering
  \begin{tabular}{c}
    \lstinputlisting[
      basicstyle=\ttfamily\scriptsize,
      language=xml]{Codigo/example-xmls.xml}
  \end{tabular}
  \caption{Ejemplo de XMLS.}
  \vspace{-.25cm}
  \caption*{Definiendo una película con XSD.}
  \label{fig:xsdex}
\end{figure}

\subsection{RDF}
RDF (del inglés \emph{Resource Description Framework}) es una familia de
especificaciones del W3C diseñada como un modelo de datos para metadatos.
Fue adoptado como una recomendación del W3C en 1999, mientras que la
especificación 1.0 fue publicada el 2004 y la 1.1 el
2014\cite{bikakis2013semantic}.

El modelo de datos RDF se basa en la idea de hacer declaraciones sobre 
recursos web (URIs) en forma de expresiones $\langle sujeto, predicado, objeto
\rangle$ que son llamados triples RDF.
El $sujeto$ indica el recurso mientras que el $predicado$ denota la relación
con el $objeto$.
Así podemos decir que un triple RDF sigue la clásica notación 
\tt{entidad-atributo-valor} de los modelos orientados a objetos, permitiendo
además, gracias a su simpleza, modelar conceptos abstractos.

Llamaremos vocabulario a la definición de conceptos y relaciones (términos)
utilizados para describir y representar un área de conocimiento.
Otro concepto a tener en cuenta son las ontologías, aunque no existe una clara 
división entre éstas y los vocabularios, generalmente se les considera más
complejas y formales.
En la web semántica una colección de triples RDF puede denotar un vocabulario o
ontología.

Un conjunto de triples RDF será representado naturalmente por un grafo dirigido.
Esta característica faculta la tecnología para ser parte fundamental de la web 
semántica pues permite relacionar información de diferentes fuentes sin mayor
problema y representarla en un esquema fácilmente identificable.

RDF es un modelo abstracto con varios formatos de serialización, por lo que la
codificación de un triple varía dependiendo el tipo de archivo en el que se
guarde. En esta memoria se trabajará con triples codificados en RDF/XML y
Turtle\cite{beckett2014turtle}.

RDF/XML fue la primera codificación estándar para serializar RDF (en un archivo
XML) y si bien es potente, es difícil de leer por las personas. En base a esto
utilizaremos Turtle pues su simpleza lo hace ideal para el entendimiento
de los triples. Algunas de las reglas para escribir un archivo Turtle 
(\tt{.ttl}) son las siguientes:
\begin{itemize}
  \item Toda sentencia termina con un punto (\tt{.}).
  \item La sucesión de tres URIs es un triple.
  \item 
    Si una linea termina en punto y coma (\tt{;}) la siguiente linea mantiene el
    $sujeto$ y solo es necesario escribir las URIs del $predicado$ y el
    $objeto$.
  \item 
    Se pueden definir prefijos con \tt{@prefix rdf: <some\_URI> .}. Esta
    característica es particularmente útil para agregar vocabularios ya
    definidos.
\end{itemize}

Podemos encontrar una descripción completa de las características y sintaxis de
Turtle en~\cite{beckett2014turtle} y de RDF/XML en ~\cite{beckett2004rdf}.

El vocabulario incluido en la especificación RDF es muy básico y por ello fue
extendido a \emph{RDF Schema}, por lo que la gran mayoría de los \emph{dataset}
RDF contienen ambos vocabularios.

En la figura \ref{fig:triples} podemos ver un ejemplo básico de la utilización 
de Turtle para la generación de triples (figura \ref{fig:triples:ttl}), el
mismo ejemplo escrito en XML/RDF (figura \ref{fig:triples:rdf}) y como estos
pueden ser visualizados como un grafo dirigido (\ref{fig:triples:grafo}).
\begin{figure}[htpb]
  \centering
  \begin{subfigure}[b]{\textwidth}
    \centering
    \begin{tabular}{c}
      \lstinputlisting[basicstyle=\ttfamily\scriptsize]{
        Codigo/example-turtle.ttl}
    \end{tabular}
    \caption{Ejemplo en Turtle.}
    \label{fig:triples:ttl}
  \end{subfigure}
  \\[0.5cm]
  \begin{subfigure}[b]{\textwidth}
    \centering
    \begin{tabular}{c}
      \lstinputlisting[basicstyle=\ttfamily\scriptsize]{
        Codigo/example-xml-rdf.rdf}
    \end{tabular}
    \caption{Mismo ejemplo en XML/RDF.}
    \label{fig:triples:rdf}
  \end{subfigure}
  \\[0.5cm]
  \begin{subfigure}[]{\textwidth}
    \centering
    \includegraphics[width=\textwidth]{figures/example-rdf_graph-1.png}
    \includegraphics[width=.5\textwidth]{figures/example-rdf_graph-2.png}
    \caption{Grafo generado.}
    \label{fig:triples:grafo}
  \end{subfigure}
  \caption{Ejemplo de triples RDF.}\label{fig:triples}
\end{figure}

\subsection{RDFS}
RDFS (de las siglas del ingles \emph{Resource Description Framework Schema},
también llamado RDF Schema) es un vocabulario que extiende RDF proveyendo un
set de clases y propiedades que mejoran la creación de modelos como son:
\tt{Class} para declarar clases, \tt{subClassOf} para denotar herencia,
\tt{range} y \tt{domain} para el rango y dominio de cierta propiedad
(\tt{rdf:Property}), entre otras.

RDF fue presentado en 1998 e introducido finalmente como recomendación del W3C
el 2004\cite{bikakis2013semantic}.

RDFS provee el vocabulario necesario para la construcción de cualquier ontología
más avanzada y por ello es la base de otros vocabularios del como OWL y SKOS.

La especificación completa del vocabulario puede encontrarse en ~\cite{brickley2014rdfs}.
%en https://en.wikipedia.org/wiki/RDF_Schema#RDFS_entailment hay un ejemplo.

\subsection{OWL}
OWL (del ingles \emph{Web Ontology Language}) es una familia de lenguajes para
la creación de ontologías complejas.
Agrega lógica computacional para que las relaciones
hechas con este lenguaje puedan ser procesadas con el fin de verificar la
consistencia de la información o generar información implicita.

La versión actual de OWL se conoce como ``OWL 2'' y fue publicada el 2009 como 
una revisión y extensión de la versión inicial publicada el
2004\cite{bikakis2013semantic}. Generalmente cuando se hablar de ``OWL'' nos
referimos a la versión del 2004.

Actualmente OWL tiene tres variantes (sublenguajes) enumerados del más simple al
más complejo como sigue\cite{mcguiness2004owl}:
\begin{enumerate}
  \item \bf{OWL Lite}:
    Se ideó pensando en dar soporte a restricciones simples y jerarquía básica,
    por ejemplo la cardinalidad de los números. 
    Se esperaba que fuera más facil generar y mantener herramientas para OWL
    Lite que para sus variantes más avanzadas, pero debido a que casi todas las
    características de OWL DL pueden ser implementadas como una combinación de
    las de OWL Lite, los desarrolladores han probado que no es 
    así\cite{grau2008owl}. Actualmente OWL Lite no es ampliamente utilizado.
  \item \bf{OWL DL}:
    Fue diseñado para proveer la máxima expresividad posible sin perder la
    decidibilidad y completitud computacional. Posee el vocabulario completo de
    OWL pero solo puede ser utilizado cumpliendo ciertas restricciones.
  \item \bf{OWL Full}:
    Fue diseñado para mantener la compatibilidad con RDFS, permite todo el
    vocabulario sin restricciones pero es indecidible por lo que un programa de
    razonamiento no puede asegurar la completitud de los resultados.
\end{enumerate}
Así, toda ontología o conclusión valida en OWL Lite tambien lo será en OWL DL y
estás a su vez lo serán en OWL Full.

OWL 2 tiene tres perfiles dependiendo de la función que cumple.
\begin{enumerate}
  \item \bf{OWL 2 EL}: 
    Es el fragmento del lenguaje decidible en tiempo polinomial, diseñado para
    trabajar con grandes volúmenes de propiedades y clases.
  \item \bf{OWL 2 QL}:
    Fue diseñado para facilitar el acceso a \emph{datasets} con un gran número
    de instancias donde las consultas son más importantes que el razonamiento.
  \item \bf{OWL 2 RL}:
    Está optimizado para el análisis de reglas lógicas, en aplicaciones que
    requieren un razonamiento escalable sin perder la expresividad del lenguaje.
\end{enumerate}
Se pueden ver las características completas de los perfiles de OWL 2 en
~\cite{motik2009owlprofiles}.

\subsection{SPARQL}

\section{Proyecto Bio2RDF}\label{ea:bio}
%\section{Métricas de centralidad}\label{ea:cent}

%\section{Algoritmos de acceso eficiente a grafos}
%\subsection{Algoritmos de búsqueda en grafos}
%\subsection{Vecinos más cercanos}
%\subsection{Relaciones entre nodos de grafos}
