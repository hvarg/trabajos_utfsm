\documentclass[letter, 10pt]{article}
\usepackage[utf8]{inputenc}
\usepackage[spanish]{babel}
\usepackage{amsfonts}
\usepackage{amsmath}
\usepackage[dvips]{graphicx}
\usepackage{url}
\usepackage[top=3cm,bottom=3cm,left=3.5cm,right=3.5cm,footskip=1.5cm,headheight=1.5cm,headsep=.5cm,textheight=3cm]{geometry}
\usepackage[]{algorithm2e}

\begin{document}
\title{
    Inteligencia Artificial \\ 
    \begin{Large}
        Informe Final: Bus Evacuation Problem
    \end{Large}
}
\author{Hernán Vargas \\ 201073009-3}
\date{\today}
\maketitle


%--------------------No borrar esta sección--------------------------------%
\section*{Evaluación}

\begin{tabular}{ll}
    Mejoras 1ra Entrega (10 \%): &  \underline{\hspace{2cm}}\\
    Código Fuente (10 \%): &  \underline{\hspace{2cm}}\\
    Representación (15 \%):  & \underline{\hspace{2cm}} \\
    Descripción del algoritmo (20 \%):  & \underline{\hspace{2cm}} \\
    Experimentos (10 \%):  & \underline{\hspace{2cm}} \\
    Resultados (10 \%):  & \underline{\hspace{2cm}} \\
    Conclusiones (20 \%): &  \underline{\hspace{2cm}}\\
    Bibliografía (5 \%): & \underline{\hspace{2cm}}\\
    &  \\
    \textbf{Nota Final (100)}:   & \underline{\hspace{2cm}}
\end{tabular}
%---------------------------------------------------------------------------%

\begin{abstract}
    %Resumen del informe en no más de 10 líneas.
    El Bus Evacuation Problem busca obtener la ruta de buses que minimiza el
    tiempo de evacuación de una zona afectada por alguna catástrofe teniendo
    buses y refugios con capacidades acotadas. En este documento se analiza la
    definición del problema y sus alcances, se plantea el escenario actual de
    investigación y se muestra el modelo matemático estándar, 
    con el objetivo de entender el problema para luego poder modelarlo con
    heurísticas basadas en algoritmos voraces  y comparar sus resultados con las
    técnicas actuales.
\end{abstract}

\section{Introducción}\label{sec:intro}
    %Una explicación breve del contenido del informe. Es decir, detalla:
    %Propósito, Estructura del Documento, Descripción (muy breve) del Problema y
    %Motivación.
    En este informe se analizarán algunas de las heurísticas disponibles
    actualmente para enfrentar el ``Bus Evacuation Problem'',  (\emph{problema
    de evacuación en buses} en español) viendo cuales son sus diferencias, 
    costos y ventajas asociadas, para luego explicar y implementar una
    resolución por medio de algoritmos voraces.

    El ``Bus Evacuation Problem'', consiste en la
    minimización del tiempo de evacuación en buses de cierta zona afectada por
    una catástrofe, sea natural o no. Suponemos que las personas a evacuar se
    juntarán en puntos de encuentro determinados y esperarán los buses que pasan
    a recogerlos para llevarlos a refugios con capacidades acotadas.
    
    Consideramos que es un problema
    interesante de estudiar pues está muy relacionado a la realidad que vemos
    en Chile el cual últimamente ha sido golpeado en reiteradas ocasiones
    por este tipo de fenómenos, así la resolución del problema ayudará
    considerablemente a la planificación en situaciones de catástrofe ya que
    cada minuto perdido puede poner en riesgo vidas humanas. Además, debido a
    que el problema no está enfocado en el desastre sino que en la necesidad de
    evacuación, puede ser ampliamente utilizado para cualquier situación que
    amerite desplazar gente desde puntos de encuentro hasta zonas determinadas.

    En la sección \ref{sec:def} se definirá el problema y los términos claves
    que debemos manejar para el correcto entendimiento del mismo, en la
    sección \ref{sec:art} se hará un resumen de las técnicas utilizadas
    actualmente para su resolución, para continuar con el modelo
    matemático estándar en la sección \ref{sec:mod}.

    En la sección \ref{sec:repr} se mostrará la representación computacional del
    modelo que será resuelto por el algoritmo mostrado en la sección
    \ref{sec:alg}. Se realizarán experimentos que serán relatados en la sección
    \ref{sec:exp} cuyos resultados serán presentados en la sección \ref{sec:res}
    para finalizar con conclusiones atingentes (sección \ref{sec:conc}).

    El objetivo de este documento es proponer las bases para una 
    investigación y resolución del problema por métodos voraces, los cuales no
    necesitan una solución inicial (pues son constructivos) y, aunque no nos dan
    las mejores soluciones, generan soluciones factibles rápidamente lo cual es
    fundamental en este tipo de situaciones.

\section{Definición del Problema}\label{sec:def}
    %Explicación del problema que se va a estudiar, en que consiste, cuales son
    %sus variables, restricciones y objetivos de manera general. Variantes más
    %conocidas que existen.
    El ``Bus Evacuation Problem'' (desde ahora BEP) busca evacuar a un número
    determinado de personas en la zona de riesgo por medio de buses. La
    sigueinte descripción es obtenida de la lectura y interpretación de 
    \cite{bish2011planning} y \cite{goerigk2013branch}:
    
    El problema asume que las personas se reunirán en puntos de encuentro
    sabidos de antemano y al momento de comenzar la evacuación no llegará más
    gente ni se moverá entre ellos. Por otro lado tenemos una
    cantidad de buses sabida con capacidades limitadas que pasarán a buscar a
    las personas a los puntos de encuentro y las llevarán a refugios
    determinados con capacidades establecidas de antemano.
    Así el objetivo de nuestro problema es encontrar la ruta de buses por la
    cual el tiempo que demora el último bus en terminar su recorrido es mínimo y
    de esta manera el tiempo total de la evacuación será mínimo.

    Para poder entender a cabalidad los alcances del problema se establecen los
    siguientes parámetros:
    \begin{enumerate}
        \item \textbf{De las personas:}
            \begin{itemize}
                \item Número de personas totales a ser evacuadas.
                \item 
                    Número de personas en cada punto
                    de encuentro al momento de comenzar la eva- cuación.
            \end{itemize}
        \item \textbf{De los puntos de encuentro:}
            \begin{itemize}
                \item Número de puntos de encuentro disponibles.
                \item Ubicación de cada punto de encuentro.
            \end{itemize}
        \item \textbf{De los refugios:}
            \begin{itemize}
                \item Número de refugios disponibles.
                \item Capacidad máxima de cada refugio en personas.
                \item Ubicación de cada refugio.
            \end{itemize}
        \item \textbf{De los buses:}
            \begin{itemize}
                \item Número de buses disponibles para la evacuación.
                \item Capacidad máxima de cada bus en personas.
                \item Ubicación inicial de cada bus (Estación de partida).
            \end{itemize}
    \end{enumerate}
    Para manejar las ubicaciones geográficas de los puntos de encuentro,
    refugios y buses, por simplicidad, se utilizan matrices de distancias como
    las siguientes:
    \begin{itemize}
        \item
            Distancias entre los puntos de encuentro y los refugios:
            $$D_{pr}\,\, /\,\, p \in [1..P], \, r \in [1..R]$$
        \item
            Distancias entre las estaciones de partidas de los buses y los
            puntos de encuentro:
            $$D^{\text{PB}}_{pb} \,\, /\,\, p \in [1..P], \, b \in [1..B]$$
    \end{itemize}

    Con estos parámetros podemos plantear el problema restringido básicamente
    con:
    \begin{itemize}
        \item Se deben evacuar todas las personas.
        \item No se debe exceder la capacidad de cada bus en cada viaje.
        \item No se debe exceder la capacidad máxima de cada refugio.
    \end{itemize}

    Como nuestro objetivo es minimizar el tiempo total de la evacuación y
    teniendo en cuenta que la distancia recorrida es proporcional al tiempo
    empleado en recorrerla (suponiendo una velocidad similar entre los buses)
    podemos simplemente minimizar la máxima distancia total recorrida por los
    buses, ya que estos se mueven al mismo tiempo. Además, para que el estudio
    sea útil debemos ser capaces de rescatar los datos relacionados al
    desplazamiento de cada uno de los buses para así generar un recorrido 
    óptimo.

    Finalmente las variables utilizadas para modelar y resolver el problema
    (obteniendo las rutas de los buses en el proceso) serán las siguientes:
    \begin{itemize}
        \item
            Una variable binaria para representar a cierto bus, en uno de sus
            viajes desde un punto de encuentro a un refugio.
        \item
            Una variable que medirá el tiempo que demora un bus en ir y en
            volver para cierto viaje.
        \item
            Una variable que mostrará el tiempo máximo de viaje de todos los
            buses.
    \end{itemize}
    La representación de estas variables y su relación con los parámetros y
    restricciones del problema son explicados con mayor profundidad en la
    sección \ref{sec:mod}.

    Con esta definición del problema efectivamente podemos generar un plan de
    acción para la evacuación de una zona en una situación que lo amerite. Aún
    así, la simplicidad del modelo deja en manifiesto varios problemas que
    ocurren en la realidad. La siguiente lista ejemplifica algunos de los
    problemas que podemos interpretar de la definición actual:
    \begin{itemize}
        \item 
            No se considera la llegada de las personas a los puntos de
            reunión. Nuestra definición actualmente espera que todas las
            personas a evacuar ya estén en los puntos de reunión y que no se
            muevan entre ellos.
        \item
            No se consideran los tiempos que requieren las personas en ingresar
            a los buses. Nuestras definiciones asumen que todas las personas
            demorarán lo mismo a la hora del abordaje, cuando en la realidad
            esto depende de a quien estemos evacuando, por ejemplo, evacuar un
            asilo de ancianos puede ser mucho más lento que evacuar una
            universidad.
        \item
            No se consideran los tiempos de conducción ni el combustible del
            vehículo. Por definición, esperamos que no deban hacerse paradas
            para el cambio de combustible o chofer.
        \item
            Se espera que los refugios tengan mayor capacidad que las personas
            evacuadas, cuando en la realidad, aunque hayan más personas
            evacuadas que capacidad de los refugios, aún así deben evacuarse
            todas.
    \end{itemize}

    Son situaciones como las descritas anteriormente las que dan pie a diversas
    variaciones y modificaciones que se le pueden hacer al problema para
    adecuarlo mejor a la realidad.

    Entre las variantes del problema se encuentra el ``Robust Bus Evacuation
    Problem''\cite{goerigk2012robust} que viene a atacar el primer punto
    mencionado anteriormente. Establece un número de evacuados incierto,
    determinado por un modelo probabilístico.
    Otro ejemplo de variación puede ser el presentado por Yue Liu y Jie
    Yu\cite{liu2011emergency} que si bien agrega los tiempos de reunión de las
    personas ignora la existencia de refugios y solo busca sacarlos de la zona
    afectada por la catástrofe.
    Más ejemplos de variaciones para diferentes casos de estudio pueden ser
    encontrados en la sección \ref{sec:art}.
    
\section{Estado del Arte}\label{sec:art}
    %Lo más importante que se ha hecho hasta ahora con relación al problema.
    %Debería responder preguntas como las siguientes ?`cuando surge?, ?`qué
    %métodos se han usado para resolverlo?, ?`cuales son los mejores algoritmos
    %que se han creado hasta la fecha?, ?`qué representaciones han tenido los
    %mejores resultados?, ?`cuál es la tendencia actual?, tipos de movimientos,
    %heurísticas, métodos completos, tendencias, etc... Puede incluir gráficos
    %comparativos, o explicativos.\\ La información que describen en este punto
    %se basa en los estudios realizados con antelación respecto al tema. Dichos
    %estudios se citan de manera que quien lea su estudio pueda también acceder
    %a las referencias que usted revisó. Las citas se realizan mediante el
    %comando \verb+\cite{ }+.  Por ejemplo, para hacer referencia al artículo de
    %algoritmos híbridos para problemas de satisfacción de restricciones
    %~\cite{Prosser93Hybrid}.

    Si bien los problemas del área ``Evacuation Planing'' y ``vehicle routing
    problem'' has sido ampliamente estudiados para situaciones particulares 
    como los huracanes Katrina y
    Rita\cite{kiefer2006incrementalism}\cite{litman2006lessons} o para lugares
    particulares como Washington DC\cite{liu2008corridor} o New 
    Orleans\cite{wolshon2002planning} estos estudios plantean una evacuación
    principalmente por parte de los vehículos particulares. El 2011 Naghawi y
    Wolshon\cite{naghawi2011performance}\cite{naghawi2011operation} registran 
    una simulación del rendimiento de una evacuación con tránsito multi-modal
    de la cual se concluye que la correcta utilización de buses puede
    incrementar la cantidad de gente evacuada y minimizar los tiempos. 

    En el ámbito de la evacuación efectuada en buses tenemos la investigación
    de Perkins et al.\cite{perkins2001modeling} que aborda el transporte de
    personas incapacitadas en una catástrofe repentina. En ésta se asume que los
    buses parten en un garaje y optimiza sus tiempos de partida de manera que el
    tiempo total del viaje sea minimizado. El problema modelado por Perkins et
    al. esta limitado por una serie de rutas estáticas que cada bus debe seguir
    para salir de la zona afectada, además, este problema no considera la
    capacidad máxima de los buses y refugios.

    El 2007 Sayyady\cite{sayyady2007optimizing} presenta un modelo basado en el
    ``Minimum-cost flow problem'' en el cual asume que la gente ya se encuentra
    en los puntos de reunión y solo basta que los buses los pasen a buscar para
    sacarlos de la zona de peligro. Este modelo considera la utilización de un
    algoritmo de búsqueda tabú para determinar las rutas óptimas de lo buses.
    Este estudio y el anterior se limitan a un viaje por bus, es decir, una
    vez el bus sale de la zona de peligro no vuelve a ingresar para buscar más
    gente. En publicaciones más 
    recientes\cite{sayyady2010optimizing}\cite{tunc2011optimizing} también se
    han desarrollado algoritmos de programación lineal entera mixta para
     encontrar las rutas óptimas para la evacuación.

     Siguiendo la misma linea Margerus et al.\cite{margulis2006hurricane} 
     generó un modelo de programación binaria para la asignación de buses a
     puntos de reunión y refugios con el objetivo de maximizar la tasa de
     evacuación para un determinado tiempo. Este estudio espera que los buses
     se encuentren en su punto de reunión designado y una vez hecho un viaje
     vuelvan a él.

     El 2009 He et al.\cite{he2009optimal} desarrollaron un modelo de
     optimización estocástico basado en el ``Location Routing Problem'' el
     cual establece un plan de evacuación para cierto número de refugios
     generando la ubicación de estos, el número de buses necesarios y sus
     respectivas rutas con el objetivo de minimizar el tiempo total de la
     operación. Aún así, la asunción de que los buses se encontrarán en los
     refugios al acontecer una catástrofe en un lugar que sabemos de antemano 
     no se adecua a la realidad.

     El 2011, el problema fue tratado por Yue Liu y Jie
     Yu\cite{liu2011emergency} con una formulación que contempla una
     optimización de dos niveles, en primer lugar: guiar a las personas desde
     edificios y zonas públicas hasta los puntos de reunión (paraderos de bus,
     estaciones de metro, etc) y en segundo lugar generar las rutas óptimas para
     que estos buses logren evacuar a toda la gente. La evacuación en este caso
     se da por completada cuando todas las personas salen de la zona de peligro.
     Este modelo considera tanto los tiempos de transporte como los de llegada
     al punto de reunión, además se espera saber de antemano la cantidad de
     gente evacuada y la capacidad del bus, por otro lado, el modelo limita a
     los buses a siempre volver al mismo punto donde empezaron.

    El mismo año Douglas R. Bish\cite{bish2011planning} nombra y plantea el
    problema de la forma en la cual enunciamos en este informe. En este estudio
    se caracteriza el problema por la aparición de refugios con capacidades
    acotadas, lo cual implica que además de sacar a la gente de la zona de
    conflicto es necesario llevarlos a un lugar con la capacidad de recibirlos.
    Del modelo matemático y heurísticas planteadas por Bish nacen las posteriores
    investigaciones que no modifican el modelo en sí, sino que intentan
    adecuarlo más a la realidad.

    Actualmente el problema ha sido tratado ampliamente por Marc Goerigk. Su
    estudio del 2012 junto a Bob Grün titulado ``The Robust Bus
    Evacuation Problem''\cite{goerigk2012robust} nos plantea una situación más
    realista que la propuesta en un principio por Bish. Ahora el número de
    personas a evacuar en principio no es conocido, en vez de ello tenemos un
    escenario probabilístico y nuestra decisión determinará si se debe esperar a
    que lleguen más personas o partir con las se encuentran actualmente en el
    punto de reunión. El estudio presenta un sistema de programación lineal
    entera mixta y se discuten diversas soluciones a este. En particular, se
    presenta una búsqueda tabú para encontrar soluciones aceptables.

    Una investigación más reciente de de Goerigk et al.\cite{goerigk2013branch}
    plantea la búsqueda de soluciones por medio del método de ramificación y
    acotamiento discutiendo diversas formas de abordar el problema para
    determinar las mejores heurísticas que lo solucionan. De estos últimos
    estudios podemos notar que la tendencia actual es agregar factores que
    afectan la situación real al modelo para lograr mejores planes de acción y
    tiempos más acotados.

\section{Modelo Matemático}\label{sec:mod}
    %Uno o más modelos matemáticos para el problema.
    A continuación se presenta la formulación matemática estándar para el BEP
    obtenido desde \cite{goerigk2013branch} con los siguiente parámetros:
    \begin{itemize}
        \item[\textbf{INPUT:}]
            Un número $B$ de buses, $P$ puntos de encuentro y $R$ refugios. Una
            matriz de distancias entre los puntos de encuentro y los refugios
            $D_{ij}\,\forall i\in [P], j\in [R]$. Un vector con la cantidad de
            gente en cada punto de encuentro $L_i\, \forall i\in [P]$
            suministrada en múltiplos de la capacidad del bus (cada unidad en
            $L_i$ representa un bus completo) y otro vector con
            la capacidad de cada refugio $U_j\, \forall j\in  [R]$. Además se
            necesita una matriz con las distancias iniciales entre los buses y
            los puntos de encuentro; por simplicidad asumiremos que todos los
            buses se encuentran en el mismo lugar así se defina el vector de
            distancias entre los puntos de encuentro y los buses como: $D^B_i\,
            \forall i\in [P]$.
        \item[\textbf{FIND:}]
            Nuestro objetivo es obtener el recorrido que minimiza el tiempo de
            viaje del bus que demora más respetando las capacidades de los buses
            y los refugios sin dejar a nadie en la zona de peligro.
    \end{itemize}

    Llamaremos además un \emph{tour} a la tupla $(i, j)$ con $i\in [P]$ y 
    $j\in [R]$.

    Para representar los viajes de los buses entre los distintos puntos de
    reunión y refugios utilizaremos la variable binaria $x^{bv}_{ij}$ que
    representa al bus $b$ viajando desde el punto de reunión $i$ hasta el
    refugio $j$ en su viaje número $v$. Así definimos un número máximo de 
    viajes que cada bus puede efectuar como $V^{max}$ para mantener el problema
    en los limites de la programación lineal.

    Por otro lado, para medir el tiempo que demorará el bus $b$ en sus viajes 
    $j$ utilizaremos las variables $t^{bv}_{ir}$ y $t^{bv}_{volver}$ para ir y
    volver respectivamente. Por último la variable $t_{max}$ denotará el máximo
    tiempo de viaje entre todos los buses.

    Así el problema queda planteado como:
    \newcommand{\qweq}{\sum_{i\in [P]} \sum_{j\in [R]}}
    \newcommand{\asda}{x^{bv}_{ij}}
    \begin{align}
        F.O:\quad&\min t_{max} \nonumber \\
        S.T:\quad& t_{max} \geq \sum_{v\in [V]}(t^{bv}_{ir} + t^{bv}_{volver})
        + \qweq D_i^Bx^{b1}_{ij} && \forall b\in [B]
        \label{r1} \\
        & t^{bv}_{ir} = \qweq D_{ij}\asda && \forall b\in [B], v\in [V]
        \label{r2} \\
        & t^{bv}_{vuelta} \geq D_{ij}( \sum_{k\in [P]}x^{bv}_{kj} + \sum_{k\in
        [R]} x^{bv+1}_{il} - 1) && \forall b\in [B], v\in [V-1], i\in [P],
         j\in [R] \label{r3} \\
        & \qweq \asda \leq 1 && \forall b\in [b], v\in [V] \label{r4}\\
        & \qweq \asda \geq \qweq x^{b,v+1}_{ij} && \forall b\in [B], v\in[V-1]
        \label{r5} \\
        & \sum_{j\in[R]}\sum_{b\in[B]}\sum_{v\in[V]} \asda \geq L_i && \forall
        i\in[P] \label{r6}\\
        & \sum_{i\in[P]}\sum_{b\in[B]}\sum_{v\in[V]} \asda \leq U_j && \forall
        j\in[R] \label{r7}\\
        &\asda \in \mathbb{B}&& \forall i\in[P],j\in[R],b\in[B],v\in[V]
        \label{r8} \\
        & t^{bv}_{ir}, t^{bv}_{volver}\in \mathbb{R}&&\forall b\in[B],v\in[V]
        \label{r9} \\
        & t^{max} \in \mathbb{R} \label{r10}
    \end{align}

    La restricción (\ref{r1}) establece como tiempo máximo al mayor de los 
    tiempos de cada uno de los buses. (\ref{r2}) calcula el tiempo que demora
    un bus en ir desde un punto de reunión hasta un refugio mientras que
    (\ref{r3}) calcula el tiempo que demora en volver. Por su parte, la
    restricción (\ref{r4}) verifica que cada bus solo haga un viaje entre el
    punto de reunión y el refugio a la vez, mientras que la restricción
    (\ref{r5}) conecta los tours de los buses e impide que estos se detengan
    en algún lugar. Las restricciones (\ref{r6}) y (\ref{r7}) verifican que las
    personas de cada punto de encuentro sean evacuadas y se respete la capacidad
    de los refugios. Las restricciones (\ref{r8}), (\ref{r9}) y (\ref{r10}) son
    la naturaleza de las variables asociadas.

\section{Representación}\label{sec:repr}
    %Representación matemática y estructura de datos que se usa (arreglos,
    %matrices, etc.), por qué se usa, la relación entre la representación
    %matemática y la estructura.
    Para la representación computacional del problema modelado en la sección
    \ref{sec:mod} se utilizaron las siguientes variables:
    \begin{itemize}
        \item (\texttt{int}) \texttt{nbus} guarda el número de buses $B$.
        \item (\texttt{int}) \texttt{cbus} guarda la capacidad de los buses.
        \item (\texttt{int}) \texttt{nest} guarda el número de estaciones.
        \item (\texttt{int[]}) \texttt{cest} 
            es un arreglo con la cantidad de buses en cada estación, así
            \texttt{cest[$x$]} será la cantidad de buses en la estación $x$.
        \item (\texttt{int}) \texttt{npep}
            guarda el número puntos de encuentro $P$.
        \item (\texttt{int}) \texttt{tpep} 
            guarda el total de personas en los puntos de encuentro.
        \item (\texttt{int[]}) \texttt{cpep}
            es un arreglo con la cantidad de personas en cada punto de
            encuentro, así \texttt{cpep[$y$]} será la cantidad de personas en el
            punto de ecuentro $y$. Representa $L_i$ del modelo matemático.
        \item (\texttt{int}) \texttt{nref} guarda el número de refugios $R$.
        \item (\texttt{int}) \texttt{tref}
            guarda la capacidad total de personas de los refugios.
        \item (\texttt{int[]}) \texttt{cref}
            es un arreglo con la capacidad de cada refugio, así
            \texttt{cref[$z$]} será la capacidad del refugio $z$. Representa
            $U_j$ en el modelo matemático.
        \item (\texttt{int[][]}) \texttt{dest\_pep}
            es un arreglo bidimensional con las distancias entre estaciones y
            puntos de encuentro, $D^{B}_{ik}$ en el modelo matemático. Así
            \texttt{dest\_pep[$x$][$y$]} será la distancia entre la estación $x$
            y el punto de encuentro $y$.
        \item (\texttt{int[][]}) \texttt{dpep\_ref}
            es un arreglo bidimensional con las distancias entre los puntos de
            encuentro y los refugios, $D_{ij}$ en el modelo matemático. Así
            \texttt{dpep\_ref[$y$][$z$]} será la distancia entre el punto de
            encuentro $y$ y el refugio $z$.
    \end{itemize}
    Además se creó la siguiente estructura para guardar los \emph{tours}
    realizados por los buses:
    \begin{verbatim}
    struct route{
        int     time;
        short   where;
        bool    in_depot;
        char    *rec;
}
    \end{verbatim}
    En ella \texttt{time} es el tiempo gastado por el bus, \texttt{where} es
    donde se encuentra el bus, si \texttt{in\_depot} es \texttt{true}
    representará el indice de una estación, en otro caso será el indice de un
    refugio. Por ultimo el puntero \texttt{rec} apunta a un \texttt{string} que
    guarda la información de los \emph{tours} como tuplas 
    $(i,j),\,i\in[P],j\in[R]$.

    Con esta estructura se crea la variable \texttt{tour[]} en la cual se
    guarda el tour de cada bus. Así \texttt{tour[$b$]} guardará la información
    del tour del bus $b$.

    Por simplicidad se utilizó el tipo de datos $int$, pero
    dependiendo de las condiciones del problema se puede cambiar a un $unsigned
    short$ para utilizar menos memoria o un $unsigned int$ para ampliar el
    rango, pues sabemos que estos parámetros no pueden ser negativos.

\section{Descripción del algoritmo}\label{sec:alg}
    %Cómo fue implementando, interesa la implementación más que el algoritmo
    %genérico, es decir, si se tiene que implementar SA, lo que se espera es que
    %se explique en pseudo código la estructura general y en párrafo explicativo
    %cada parte como fue implementada para su caso particular, si se utilizan
    %operadores se debe explicar por que se utilizó ese operador, si fuera el
    %caso de una técnica completa, si se utiliza recursión o no, etc. En este
    %punto no se espera que se incluya código, eso va aparte.
    Para la resolución del problema se utilizó un algoritmo voraz de elaboración
    propia tomando como base la creación de \texttt{tours} en tiempo de
    ejecución que se describe en UB4\cite{goerigk2013branch}. El algoritmo
    hace lo que se lista a continuación:
    \begin{enumerate}
        \item
            Primero se selecciona el bus con menor tiempo actual, si hay un
            empate se elige al azar. Como al inicio todos tienen tiempo cero
            siempre se elegirá al azar la primera iteración.
        \item
            Se selecciona el punto de encuentro más cercano (con personas a
            rescatar) a la estación o
            refugio en el cual está el bus escogido en el punto anterior. Si hay
            empate en las distancias se elige al azar entre los afectados.
        \item
            Se selecciona el refugio con capacidad más cercano al punto de 
            encuentro elegido
            en el punto anterior. Si hay empate se elige al azar entre los
            afectados.
        \item
            Se actualiza la información, es decir, resta la capacidad del bus al
            punto de encuentro y al refugio afectados, se suma el tiempo de
            viaje al tour del bus y se modifica su posición.
        \item
            Finalmente se resta la capacidad del bus al total de personas a
            rescatar, si este número no queda en cero se repite el proceso.
    \end{enumerate}

    El proceso es descrito en el siguiente pseudo código utilizando las
    variables enunciadas en la sección anterior.\\

    \begin{algorithm}[H]
        \For{bus in buses}{
            \texttt{tour[bus].time  = 0}\\
            \texttt{tour[bus].where = estacion\_disponible()}\\
            \texttt{tour[bus].rec = ``''}\\
        }
        \While{\texttt{tpep} $ > 0$}{
            \texttt{bus = bus\_mejor\_tiempo()}\\
            \texttt{pep = pep\_mas\_cercano(bus)}\\
            \texttt{ref = ref\_mas\_cercano(pep)}\\
            \texttt{tour[bus].time += dist\_bus\_pep(bus, pep) +
            dist\_pep\_ref(pep, ref)}\\
            \texttt{tour[bus].where = ref}\\
            \texttt{tour[bus].rec += }``(pep,ref)''\\
            \texttt{cpep[pep] -= cbus}\\
            \texttt{cref[ref] -= cbus}\\
            \texttt{tpep -= cbus}\\
        }
        \KwResult{Solución factible en tour[*].rec}
    \end{algorithm}

    Algunas de consideraciones a tener en cuenta sobre funcionamiento del
    algoritmo son las siguientes:
    \begin{itemize}
        \item
            Los buses parten en una estación pero nunca más vuelven a
            ellas. En cada viaje después del primero el bus se moverá de un
            refugio a otro recogiendo la gente en los puntos de encuentro.
        \item
            Como cada iteración contempla tanto el viaje de ida al punto de
            encuentro como el viaje hacia el refugio más cercano, el número de
            iteraciones será igual al número de viajes efectuados. Por ello
            podemos calcular las iteraciones del problema de antemano como:
            \texttt{tpep/cbus}.
        \item
            El algoritmo al ser constructivo será muy rápido pues solo tomo
            decisiones teniendo en cuenta cual es la mejor situación en cada
            paso. Se recomienda utilizar la solución generada como solución
            inicial en algún algoritmo que mejore soluciones.
    \end{itemize}

\section{Experimentos}\label{sec:exp}
    %Se necesita saber como experimentaron, como definieron parámetros, como los
    %fueron modificando, cuales problemas se trataron, instancias, por que
    %ocuparon esos problemas.
    Para la experimentación se utilizaron las instancias de prueba
    suministradas y algunas de generación propia.

    Como partes fundamentales del algoritmo utilizan números aleatorios para
    decidir que hacer, cada ejecución puede dar un resultado diferente
    dependiendo de la semilla elegida para generar el \texttt{random}. Por ello
    se ejecutó el programa múltiples veces registrando como variaron los
    tiempos y resultados de cada instancia.

    Después de utilizar las instancias de prueba otorgadas se optó por crear
    nuevas instancias que llevarán los parámetros a limites más alto por dos
    motivos principalmente:
    \begin{enumerate}
        \item 
            Analizar el comportamiento del algoritmo en situaciones más
            cercanas a la realidad donde el número de evacuados sea de miles de
            personas.
            Por ello se crearon múltiples instancias con más de $10000$
            personas cada una.
        \item
            Probar las capacidades del algoritmo variando los parámetros de
            diferentes maneras que no son muy posibles en la realidad. Algunas
            de las situaciones analizadas fueron:
            \begin{itemize}
                \item
                    Cantidad similar de puntos de encuentro y refugios.
                \item
                    Más refugios que puntos de encuentro.
                \item
                    Muy pocos buses para mucha gente.
                \item
                    Más estaciones que puntos de encuentro o refugios.
            \end{itemize}
    \end{enumerate}

    Para todas las instancias creadas se utilizó una cantidad de buses de $24$
    personas lo que nos permite remitirnos al ``peor de los casos''.

    Por simplicidad para la generación de las nuevas instancias de prueba se
    creo un simple \texttt{script} en \texttt{python} que genera los archivos
    necesarios teniendo en cuenta lo siguiente:
    \begin{itemize}
        \item
            El total de gente es un múltiplo de la capacidad de los buses. De
            igual manera lo son la gente en los puntos de encuentro y la 
            capacidad de los refugios.
        \item
            El total de gente es igual a la suma de la gente en los puntos de
            encuentro.
        \item
            Los refugios tiene una capacidad total cercana al $115\%$ de las
            personas a evacuar.
        \item
            No puede haber más estaciones que buses.
        \item
            Las distancias son elegidas al azar entre dos números. En 
            nuestro caso, por simplicidad, entre $1$ y $9$.
    \end{itemize}

    Las $10$ instancias creadas con este \texttt{script} y las suministradas son
    analizadas en profundidad en la sección \ref{sec:res} mediante 
    la ejecución de cada prueba 5 veces para así poder notar variaciones en el
    tiempo de ejecución y los resultados obtenidos.

\section{Resultados}\label{sec:res}
    %Que fue lo que se logró con la experimentación, incluir tablas y
    %parámetros, gráficos si fuera posible, lo más explicativo posible.

    Las siguientes tablas muestran algunos datos de interés y los resultados de
    las repetidas ejecuciones del algoritmo voraz sobre las instancias
    suministradas y las de elaboración propia.

    Con:
    \begin{itemize}
        \item \textbf{B:} Número de buses.
        \item \textbf{cB:} Capacidad de los buses.
        \item \textbf{E:} Número de estaciones.
        \item \textbf{P:} Número de puntos de encuentro.
        \item \textbf{R:} Número de refugios
        \item \textbf{T:} Total de personas.
        \item el \textbf{peso} del archivo en bytes.
        \item 
            los \textbf{resultados} de cada ejecución del algoritmo sobre la
            instancia seleccionada separados por una coma.
        \item y el \textbf{tiempo} como la media de las 5 ejecuciones medidas
            con \texttt{/usr/bin/time}.
    \end{itemize}

    \begin{table}[h!]
        \centering
        \caption{Resultados instancias suministradas.}
        \begin{tabular}{|l|c|c|c|c|c|c|c|c|c|c|}
            \hline
            \multicolumn{1}{|c|}{{\bf Instancia}} & {\bf Peso} & {\bf B} &
                    {\bf cB} & {\bf E} & {\bf P} & {\bf R} & {\bf T} & 
                    {\bf Resultados}   & {\bf Iter} & {\bf Time} \\ \hline
            1-4-2-4    & 105  & 4  & 20 & 1 & 4  & 2  & 200  & 
                    23, 23, 20, 20, 23 & 10 & 0,0046 \\ \hline
            1-5-3-6    & 130  & 6  & 20 & 1 & 5  & 3  & 140  & 
                    13, 13, 13, 13, 13 & 7  & 0.0054 \\ \hline
            2-12-3-6   & 286  & 6  & 20 & 2 & 12 & 3  & 560  &
                    42, 37, 43, 28, 41 & 28 & 0.0062 \\ \hline
            2-22-4-10  & 583  & 10 & 20 & 2 & 22 & 4  & 1040 &
                    37, 28, 29, 27, 33 & 52 & 0.0078 \\ \hline
            2-32-5-18  & 826  & 18 & 20 & 2 & 32 & 5  & 1360 &
                    27, 24, 30, 28, 33 & 68 & 0.0060 \\ \hline
            2-9-7-5    & 315  & 5  & 36 & 2 & 9  & 7  & 579  &
                    20, 20, 17, 20, 21 & 16 & 0.0042 \\ \hline
            3-11-10-7  & 481  & 7  & 36 & 3 & 11 & 10 & 684  &
                    20, 16, 18, 24, 21 & 19 & 0.0062 \\ \hline
            5-25-12-15 & 1.3K & 15 & 36 & 5 & 25 & 12 & 1764 &
                    24, 23, 23, 26, 23 & 49 & 0.0058 \\ \hline
            8-40-20-20 & 2.9K & 20 & 36 & 8 & 40 & 20 & 2700 &
                    21, 24, 20, 21, 23 & 75 & 0.0062 \\ \hline
        \end{tabular}
        \label{tabla1}
    \end{table}

    \begin{table}[h!]
        \centering
        \caption{Resultados instancias propias.}
        \begin{tabular}{|l|c|c|c|c|c|c|c|c|c|c|}
            \hline
            \multicolumn{1}{|c|}{{\bf Instancia}} & {\bf Peso} & {\bf B} &
                    {\bf cB} & {\bf E} & {\bf P} & {\bf R} & {\bf T} & 
                    {\bf Resultados}   & {\bf Iter} & {\bf Time} \\ \hline
            10-1000-500-50 & 1006K & 50 & 24 & 10 & 1000 & 500 & 36000 &
                    68, 67, 63, 68, 66      & 1500 & 0,2198 \\ \hline
            10-100-50-30   & 13K   & 30 & 24 & 10 & 100  & 50  & 36000 &
                    114, 118, 118, 116, 119 & 1500 & 0,0248 \\ \hline
            10-110-50-30   & 15K   & 30 & 24 & 10 & 110  & 50  & 48000 &
                    147, 147, 143, 145, 145 & 2000 & 0,0308 \\ \hline
            10-200-110-18  & 49K   & 18 & 24 & 10 & 200  & 110 & 6720  &
                    39, 38, 40, 39, 36      & 280  & 0,0140 \\ \hline
            12-300-240-30  & 151K  & 30 & 24 & 12 & 300  & 240 & 10800 &
                    34, 34, 37, 35, 37      & 450  & 0,0312 \\ \hline
            15-280-220-40  & 132K  & 40 & 24 & 15 & 280  & 220 & 6720  &
                    18, 18, 19, 21, 20      & 280  & 0,0240 \\ \hline
            15-350-20-40   & 27K   & 40 & 24 & 15 & 350  & 20  & 12000 &
                    34, 36, 37, 34, 37      & 500  & 0,0172 \\ \hline
            6-180-60-30    & 25K   & 30 & 24 & 6  & 180  & 60  & 14400 &
                    44, 49, 48, 50, 46      & 600  & 0,0176 \\ \hline
            6-50-300-20    & 32K   & 20 & 24 & 6  & 50   & 300 & 12000 &
                    61, 59, 62, 57, 62      & 500  & 0,0194 \\ \hline
            6-650-300-30   & 395K  & 30 & 24 & 6  & 650  & 300 & 19200 &
                    61, 58, 60, 62, 57      & 800  & 0,0660 \\ \hline
        \end{tabular}
        \label{tabla2}
    \end{table}

    Los resultados de las instancias suministradas se muestran en la tabla
    \ref{tabla1} mientras que la tabla \ref{tabla2} muestra el comportamiento
    del algoritmo en las instancias de prueba propias.

    Del análisis general de estos resultados obtenemos las siguientes
    conclusiones:
    \begin{itemize}
        \item
            Como se esperaba; el número de iteraciones corresponde a el total de
            personas dividido en la capacidad del bus. Este número es constante
            pues el algoritmo es constructivo y el proceso para
            generar una solución siempre hace las mismas operaciones y por lo
            tanto toma aproximadamente el mismo tiempo.
            Aún así, al recurrir al azar, los resultados pueden variar
            drásticamente como evidencia la instancia \texttt{2-12-3-6} donde
            generalmente se obtienen resultados cercanos al $42$ pero en una
            ocasión se obtuvo un $28$ lo que representa una mejora considerable
            en la solución de dicho problema.
        \item
            El tiempo de ejecución del programa es proporcional al tamaño del
            archivo de entrada. Esto sumado al punto anterior evidencia que, si
            bien el proceso de obtener una solución y sus respectivas
            iteraciones tienen influencia en el tiempo de ejecución, la mayor
            perdida de tiempo se hace a la hora de efectuar la lectura del
            archivo de entrada. Una prueba de esta situación es la comparación
            entre la instancia \texttt{10-1000-500-50} y la
            \texttt{10-100-50-30} donde ambas tienen la misma cantidad de
            iteraciones ($2000$) pero sus tiempos de ejecución varían
            considerablemente por el tamaño del archivo ($1006$K en $0,2198$
            contra $13$K en $0,0248$).

            Razones evidentes de esta situación son la poca
            optimización que presenta el código con respecto a la lectura del
            archivo (hecha fácil de leer a costa de eficiencia) y la lentitud de
            acceso a archivos en disco duro. Además, por las razones mencionadas
            anteriormente, podemos suponer que estas instancias utilizarán más
            memoria \texttt{ram}. La velocidad del algoritmo y la cantidad de
            memoria utilizada pueden ser mejorados seleccionando un
            \texttt{buffer} de menor tamaño ($256$ \texttt{bytes} pueden ejecutar
            todas las instancias de prueba suministradas). 
        \item
            Los parámetros que más afectan los resultados son la cantidad de
            buses y su capacidad en relación con la cantidad de gente total a
            evacuar. Esta conclusión es evidente pues si tenemos más buses con
            mayor capacidad disponibles para ir a rescatar a la gente
            paralelamente serán mejores los resultados. Varias instancias
            evidencian esta situación, por ejemplo las instancias
            \texttt{15-350-20-40} y \texttt{6-50-300-20} tiene $12000$ personas
            a evacuar, pero los resultados de la segunda casi doblan los de la
            primera pues se tienen la mitad de los buses. 
        \item 
            En las instancias evaluadas no se nota una relación real entre la
            cantidad de refugios y la de puntos de encuentro con los resultados
            de la prueba. Esta situación ocurre por dos razones principales:
            \begin{enumerate}
                \item
                    Las instancias de prueba fueron generadas aleatoriamente con
                    un rango de distancias relativamente bajo y bien
                    distribuido. Por ello, la distancia entre las estaciones,
                    puntos de encuentro y refugios no hacen una contribución muy
                    evidente en la variación de los resultados generales. Es de
                    suponer que en la realidad las distancias cumplan un papel
                    diferenciador entre una solución \emph{mala} y una
                    \emph{buena}, aún así, el algoritmo voraz entregará en todos
                    los casos una solución factible.
                \item
                    La capacidad de los refugios generados siempre es mayor que
                    el total de personas a evacuar por un $15\%$ y sigue una
                    distribución aleatoria bien distribuida. Así, siempre habrá
                    un refugio relativamente cercano con capacidad disponible
                    por lo que situaciones donde buses tengan que ir a un
                    refugio lejano son poco probables.
            \end{enumerate}
    \end{itemize}

\section{Conclusiones}\label{sec:conc}
    %De acuerdo a la introducción que se hizo, entregar afirmaciones RELEVANTES
    %basadas en los experimentos y sus resultados.
    El ``Bus evacuation problem'' se diferencia de los la mayoría de los
    problemas de ``Evacuation planing'' por dos puntos fundamentales:
    \begin{enumerate}
        \item 
            Considera la evacuación de las personas utilizando el servicio
            público, así genera una ruta optima de buses para sacar a los
            afectados del peligro dejando de lado la acción particular para
            generar un plan de acción centralizado. Algunos ejemplos de estudios
            que utilizan la acción particular son: 
            \cite{kiefer2006incrementalism} \cite{litman2006lessons}
            \cite{liu2008corridor} \cite{wolshon2002planning}
        \item
            Plantea la existencia de refugios y sus respectivas capacidades,
            elemento que es ignorado por la mayoría de los modelos previos que
            solo buscaban sacar a la gente del peligro, pero no se preocupan de
            donde llevarlos. Algunos ejemplos de modelos que
            no consideran refugios son: \cite{perkins2001modeling}
            \cite{sayyady2007optimizing} \cite{sayyady2010optimizing}
            \cite{tunc2011optimizing}
    \end{enumerate}
    Estos dos puntos hacen del BEP una excelente opción a la hora de planificar
    una eventual evacuación, debido a que ataca dos de los problemas recurrentes
    en cualquier catástrofe.

    Aún así, la definición del problema (sección \ref{sec:def}) ignora algunos
    aspectos que la situación real debe afrontar de todos modos como son los
    cambios de combustible y conductor y la posibilidad de que la gente no se
    encuentre en los puntos de reunión o que se muevan entre ellos.

    Por otra parte, aunque el problema está acotado a un área geográfica 
    especifica y en teoría conocida, las
    dimensiones del mismo hacen imposible su tratamiento por métodos
    convencionales y se hace necesaria la creación y modificación de heurísticas
    para su solución en tiempos acotados. Las investigaciones actuales han hecho
    grandes avances por medio de la modificación de los algoritmos para
    ``Vehicle routing problem'' o técnicas como la búsqueda tabú, pero siempre
    hay espacio para la mejora.

    Si bien, cuando los planes de evacuación son hechos con anterioridad, podemos
    esperar obtener las mejores soluciones posibles, en un caso como éste, que
    puede ocurrir en cualquier momento y en el cual cualquier retraso puede
    significar poner en peligro vidas, es necesario que se encuentren soluciones
    factibles rápidamente aunque estas no sean las más eficientes. En este
    contexto podemos dividir la búsqueda de soluciones en dos frentes: las
    heurísticas que mejoran soluciones factibles y las que generan las mismas.
    Este documento se centra en estas últimas mediante la creación de un
    algoritmo voraz generador de soluciones que posteriormente pueden ser
    mejoradas por algunas de las heurísticas de la primera categoría.

    En cuanto a la creación de algoritmos voraces para la generación de estas
    soluciones el trabajo de Goerigk et al. ``Branch and bound algorithms for 
    the bus evacuation problem'\cite{goerigk2013branch} presenta cuatro de ellos
    de los cuales los primeros tres necesitan una lista de \emph{rutas} posibles
    generadas con anticipación para los buses mientras que el cuarto las genera
    en tiempo de ejecución. 
    Nuestro algoritmo se basa en esta última idea y mediante el calculo de las
    distancias mínimas en cada iteración obtiene una solución factible
    rápidamente.

    De los resultados de la experimentación podemos concluir que la velocidad
    esperada de este tipo de algoritmos es su gran ventaja a la hora de buscar
    soluciones. Problemas con más de $10000$ variables son resueltos en menos de
    un segundo lo que nos hace presumir que el tiempo de ejecución no será una
    limitante. Por otro lado, los efectos indeseados la variación de las 
    soluciones entregadas debido a la aleatoriedad de algunos pasos pueden ser 
    abolidos utilizando la misma velocidad que ganamos: nos basta con ejecutar
    el programa varias veces y seleccionar el resultado con menor tiempo de
    evacuación. Aún así, la solución siempre es factible pero nunca podemos
    asegurar que sea la más eficiente por lo que se recomienda utilizarla como
    solución inicial en un proceso que la mejore si hay tiempo suficiente para
    ello.

    La planificación de la evacuación en una catástrofe es un proceso delicado
    que generalmente requiere la coordinación, anticipación y información
    suficiente para actuar rápidamente cuando ésta ocurra. La resolución
    del BEP necesita los tres elementos, pero los algoritmos voraces pueden
    brindarnos soluciones al instante si tenemos la información necesaria. Por
    ello su estudio y mejoramiento se hace importante no solo como soluciones
    iniciales de cualquier otra heurística sino que también para actuar cuando
    no se tiene el tiempo de buscar mejores soluciones ya que las entregadas por
    nuestro algoritmo son lo suficientemente buenas como para ser replicadas en
    la realidad.

\section{Bibliografía}\label{sec:bib}
    %Indicando toda la información necesaria de acuerdo al tipo de documento
    %revisado. Las referencias deben ser citadas en el documento.

\bibliographystyle{plain}
\bibliography{Referencias}
\end{document} 
