\documentclass[journal, 10pt]{IEEEtran}
\usepackage[utf8]{inputenc}
\usepackage[spanish]{babel}
\usepackage{amsfonts}
\usepackage{amsmath}
\usepackage{graphicx}
\usepackage{url}
\usepackage{tikz}
\def\IEEEkeywordsname{Palabras Claves}

\makeatletter
\long\def\@makecaption#1#2{\ifx\@captype\@IEEEtablestring%
\footnotesize\begin{center}{\normalfont\footnotesize #1}\\
{\normalfont\footnotesize\scshape #2}\end{center}%
\@IEEEtablecaptionsepspace
\else
\@IEEEfigurecaptionsepspace
\setbox\@tempboxa\hbox{\normalfont\footnotesize {#1.}~~ #2}%
\ifdim \wd\@tempboxa >\hsize%
\setbox\@tempboxa\hbox{\normalfont\footnotesize {#1.}~~ }%
\parbox[t]{\hsize}{\normalfont\footnotesize \noindent\unhbox\@tempboxa#2}%
\else
\hbox to\hsize{\normalfont\footnotesize\hfil\box\@tempboxa\hfil}\fi\fi}
\makeatother

%Para insertar texto plano
\usepackage{fancyvrb}
% redefine \VerbatimInput
\RecustomVerbatimCommand{\VerbatimInput}{VerbatimInput}%
{fontsize=\footnotesize,
 %
 frame=lines,  % top and bottom rule only
 framesep=2em, % separation between frame and text
 rulecolor=\color{Gray},
 %
 label=\fbox{\color{Black}data.txt},
 labelposition=topline,
 %
 commandchars=\|\(\), % escape character and argument delimiters for
					  % commands within the verbatim
 commentchar=*		% comment character
}

\begin{document}
\title{\textit{Problema de corte en una y dos dimensiones.}}
\author{
	Alex Escárate, Alonso Sandoval, Hernán Vargas.\\ 
	Universidad Técnica Federico Santa María \\
	Avenida España 1680, Valparaíso, Chile \\
	alex.escarate@alumnos.usm.cl, alonso.sandovala@alumnos.usm.cl,
	hernan.vargas@alumnos.usm.cl. \\ 
}

\maketitle
\begin{abstract}
	\boldmath El presente documento plantea las características básicas y 
	resolución por programación lineal del llamado problema de corte, tanto 
	en una como dos dimensiones. Establece la formulación matemática y las
	consideraciones a tener en cuenta para la resolución del problema, además
	ejemplifica el corte en dos dimensiones para el rubro de la creación 
	de muebles y extiende dicha problemática con límites y costos variables
	en la utilización de los patrones.
\end{abstract}

\begin{IEEEkeywords}
	Problema de corte, Problema de la guillotina, Programación lineal,
	Corte en una dimensión, Corte en dos dimensiones, Patrones de corte fijos,
	Patrones de corte con costo, Patrones limitados.
\end{IEEEkeywords}

\section{Introducción}
	El problema de corte, más conocido como ``The cutting stock problem'', es
	un problema clásico de optimización que surge de la necesidad de la 
	industria de mano-factura de mejorar sus procesos productivos. 
	Comúnmente lo que se busca es minimizar el uso de materia prima
	para la generación de productos más pequeños, por ejemplo, una plancha 
	de metal de área $A \times L$ para la producción de diferentes planchas de 
	menor tamaño con áreas: $a_1 \times l_1$, $a_2 \times l_2$, etc.
	
	Este problema ha sido estudiado por personas desde el área de la 
	programación hasta gente de negocios, en donde diversas metodologías se han
	desarrollado en la búsqueda de la optimización de los recursos disponibles. 
	Para las empresas actuales, el desarrollo de estos métodos significa un 
	importante impacto en la disminución de costos, lo que las vuelve más 
	competitivas, factor fundamental en el sistema económico moderno.
	
	El problema en sí tiene muchas variantes, todas dependiendo del contexto
	de la industria donde se aplica. La formulación más general consta de
	un ``stock'' disponible de materiales a cortar para la producción de 
	pedidos más pequeños como mencionábamos anteriormente. La forma
	en que cortamos la materia prima la denominamos ``patrones'', las
	cuales determinan las unidades de cada pedido a generar a partir del
	material del ``stock''.
	
	La función objetivo varía dependiendo de las necesidades de la
	empresa. La más común es minimizar los residuos que se generan de
	los patrones de corte dado que estos no son perfectos.
	Otro enfoque podría ser minimizar los costos asociados a la maquinaria
	que genera los patrones de cortes, es decir, considerar el coste de
	generar un patrón como variable, así se requeriría elegir la combinación
	de actividades (que realizan el corte de acuerdo a un patrón) más 
	económica. También se pueden considerar las dimensiones de la materia
	prima a cortar, considerando sólo el largo(1D), el largo y el ancho(2D), 
	e incluso volúmenes(3D). Acorde a este último criterio el problema
	adquiere mayor complejidad a medida que aumentamos las dimensiones.
	
	Los artículos más utilizados en la investigación y construcción de este 
	documento son los escritos de Gilmore y 
	Gomory\cite{Gilmore:1961}\cite{Gilmore:1965}, la tesis presentada por 
	Rosa Delgadillo en el 2007\cite{Delgadillo:2007} y el paper de Tzu-Yi 
	Yu et al. del 2014\cite{Tzu-Yi Yu:2014}.
	
	A continuación realizaremos un barrido histórico por las diferentes
	soluciones al problema, planteando al final el modelo matemático
	que contemplaremos en este proyecto.

\section{Estado del arte} \label{sec:arte}
	El problema del corte es \texttt{NP-Completo} lo que además de implicar que
	no existen algoritmos que puedan resolver el problema en tiempo polinomial,
	tiene como consecuencia que podamos hacer la reducción de éste a uno de los
	21 problemas de Karp\cite{Karp:1972}: El problema de la mochila, razón por
	la cual, toda investigación de dicho problema también nos es útil a la hora
	de analizar el problema de corte, es más, podemos tratar indistintamente
	tanto los problemas de la mochila (corte en 1D), los de corte (2D) y los de
	almacenamiento o empaquetamiento (corte en 3D) que afectan constantemente a
	la industria, puesto que sus soluciones serán prácticamente las mismas.
	
	Debido a los factores mencionados anteriormente, el problema del corte ha
	sido estudiado y documentado ampliamente desde los años 60. Los escritos más
	importantes de dicha época son sin duda los de P.C Gilmore y R.E Gomory
	que en 1961 publican una resolución del problema de corte en una dimensión 
	por métodos de programación lineal, además nos dan una descripción general
	de la situación en dos dimensiones y el problema de la 
	guillotina\cite{Gilmore:1961}. 
	En 1963 extienden su obra y la contextualizan en la industria del
	papel\cite{Gilmore:1963} y en 1965 amplían su solución analizando los 
	problemas en dos o más dimensiones como una colección de problemas
	unidimensionales\cite{Gilmore:1965} reducibles al problema de la mochila 
	cuya solución ya habían descrito.
	
	Después de sus trabajos el problema de corte en dos dimensiones ha sido
	contextualizado a diferentes industrias con necesidades específicas. En
	general, el problema puede ser clasificado dependiendo de sus patrones de
	corte, así podemos desglosar en problemas con guillotina o sin ella. El
	problema de la guillotina radica en que en ciertos rubros para minimizar los
	gastos es necesario que los cortes a efectuar sean ininterrumpidos y rectos
	desde un lado de la placa hasta el otro, luego el problema de la guillotina
	queda definido por el número mínimo de cortes necesarios para crear un 
	patrón, la figura~\ref{fig:patrones} es un ejemplo de ello.
	\begin{figure}[t]
	\centering
	\begin{tikzpicture}[scale = 0.80]
		% sin guillotina:
		\node at (2,3.5) {Sin guillotina:};
		\draw (0, 0) -- (4, 0) -- (4, 3) -- (0, 3) -- cycle ;
		\draw (0, 2) -- (3, 2);
		\draw (1, 0) -- (1, 2);
		\draw (1, 1) -- (4, 1);
		\draw (3, 1) -- (3, 3);

		% n-guillotina:
		\node at (8,3.5) {Guillotina de $n$ fases:};
		\draw (6, 0) -- (10, 0) -- (10, 3) -- (6, 3) -- cycle;
		\draw (6, 1.3) -- (10, 1.3);
		\draw (7.5, 0) -- (7.5, 1.3);
		\draw (8.5, 0) -- (8.5, 1.3);
		\draw (8, 1.3) -- (8, 3);
		\draw (6, 0.8) -- (7.5, 0.8);
		\draw (8.5, 0.8) -- (10, 0.8);
		\draw (8.5, 0.1) -- (10, 0.1);
		\draw (7.5, 0.4) -- (8.5, 0.4);
		\draw (7.5, 0.4) -- (8.5, 0.4);
		\draw (8, 1.5) -- (10, 1.5);
		\draw (9.4, 1.5) -- (9.4, 3);
		\draw (9.1, 1.5) -- (9.1, 1.3);
		\fill[gray!60]  (7.5, 0) -- (7.5, 0.4) -- (8.5, 0.4) -- 
						(8.5, 0.1) -- (10, 0.1) -- (10, 0) -- cycle;
		\fill[gray!60]  (9.4, 3) -- (10, 3) -- (10, 1.3) -- (9.1 ,1.3) --
						(9.1, 1.5) -- (9.4, 1.5) -- cycle;

		% 2- guillotina:
		\node at (2,8.5) {Guillotina de 2 fases:};
		\draw (0, 5) -- (4, 5) -- (4, 8) -- (0, 8) -- cycle ;
		\draw (2, 5) -- (2, 8);
		\draw (2, 6.5) -- (4, 6.5);
		\draw (0, 7.3) -- (2, 7.3);
		\fill[gray!60] (0, 8) -- (0, 7.3) -- (2, 7.3) -- (2, 8) -- cycle;

		% 3- guillotina:
		\node at (8,8.5) {Guillotina de 3 fases:};
		\draw (6, 5) -- (10, 5) -- (10, 8) -- (6, 8) -- cycle ;
		\draw (6, 6.5) -- (10, 6.5);
		\draw (6.6, 7.2) -- (10, 7.2);
		\draw (6.6, 8) -- (6.6, 6.5);
		\draw (6, 5.8) -- (9.2, 5.8);
		\draw (9.2, 5) -- (9.2, 6.5);
		\draw (9.8, 5) -- (9.8, 6.5);
		\fill[gray!60] (9.8, 5) -- (9.8, 6.5) -- (10, 6.5) -- (10, 5) -- cycle;
	\end{tikzpicture}
	\caption{Ejemplo de patrones para cortes con guillotina.}
	\label{fig:patrones}
\end{figure}

	
	En 1985 Beasley presenta un modelo de programación dinámica que, por medio
	de la recursión, logra obtener múltiples soluciones óptimas para problemas
	de corte por guillotina\cite{Beasley:1985}, aunque solo es posible en
	pequeña escala. Evitando los problemas de la recursión, Christofides y
	Whitlock\cite{Chris:1977} en 1977 solucionan el mismo problema con arboles
	de búsqueda, método que mejorará su eficiencia por el trabajo de
	Viswanathan y Bagchi en 1993\cite{Viswa:1993} y el de Hifi y Ouafi en 
	1997\cite{Hifi:1997} que, aplicando Best-First, empeoran las soluciones, 
	pero mejoran los tiempos.

	Para la creación de patrones, en 1983 Wang presenta dos métodos
	combinacionales\cite{Wang:1983} los cuales son mejorados posteriormente 
	por Vasko\cite{Vasco:1989} en 1989 y por Oliveria y 
	Ferreira\cite{Olib:1990} en 1990. Dichos métodos aún son ampliamente
	utilizados.

	El 2000, Valdes et al.\cite{Valdes:2000} presentan un algoritmo de búsqueda
	tabú\cite{Golver:1997} para solucionar problemas de corte generales.
	Posteriormente utilizan un algoritmo GRASP\cite{Feo:1989} para mejorar sus
	soluciones óptimas y disminuir la perdida de material en patrones con corte
	de guillotina de más de dos fases. Si bien, este último algoritmo muestra un
	mejor desempeño también requiere mayor tiempo computacional.

	Tanto los algoritmos de programación lineal como los dinámicos deben buscar
	entre todas las combinaciones la mejor solución posible, debido a la carga
	computacional que supone dicha tarea, es de esperar que la mejora constante
	que tienen los procesadores ayude significativamente en la resolución. 
	Actualmente es posible obtener soluciones en tiempos aceptables para una 
	gran cantidad de situaciones con los algoritmos descritos anteriormente,
	pero, cuando la cantidad de patrones posibles incrementa, el tiempo 
	computacional también lo hace, y no polinomialmente, por ello algunos
	algoritmos sacrifican la exactitud de sus soluciones en busca de tiempos
	más plausibles para la industria.
	
	En este contexto los algoritmos genéticos\cite{Holland:1975}, que utilizan
	la noción de la supervivencia del más fuerte y los algoritmos de recorrido
	simulado\cite{Kirk:1983}\cite{Cerny:1985}, que deciden la solución óptima 
	basados en probabilidades, han sido los pioneros en la investigación actual
	del problema del corte. 
	
	Un ejemplo de algoritmo de recorrido simulado para la creación de patrones
	sin guillotina es el de Lai y Chan\cite{Lai:1997}. Faina utiliza el mismo
	tipo de algoritmo pero quita la restricción de guillotina en su trabajo de
	1999\cite{Faina:1999}. Una variante de solución utilizando arboles
	binarios en conjunto a un algoritmo de camino simulado fue propuesto por 
	Parada et al. en 1996\cite{Parada:1996}. Por otro lado, los algoritmos 
	genéticos no se quedan atrás en la investigación, varios autores enfocan
	su trabajo buscando soluciones en ellos como por ejemplo: Jakobs en 
	1993\cite{Jackobs:1993}, Wu en 2006\cite{Wu:2006} o Goncalves en 
	2007\cite{Gon:2007}, entre otros.
	
	En general, la estructura de un problema de corte puede ser declarada según
	Dyckhoff\cite{Dyckhoff:1990}, de la siguiente manera:
	\begin{enumerate}
		\item Existen dos grupos de parámetros básicos, cuyos elementos son
		cuerpos geométricos en una o más direcciones:
		\begin{itemize}
			\item Un conjunto de órdenes que determinan cierta cantidad de 
			piezas con tamaños, formas y cantidades específicas.
			\item Un stock grande de bloques de material de tamaño fijo del 
			cual deben cortarse las piezas determinadas por la demanda.
		\end{itemize}
		\item Existe un proceso que determina los patrones por los cuales se 
		pueden cortar los bloques de material de manera de obtener una o más
		piezas demandadas. Puede cumplir alguna de las siguientes condiciones:
		\begin{itemize}
			\item Especificaciones del material a cortar.
			\item Tecnología de corte utilizada.
			\item Existencia de objetivos específicos a alcanzar.
		\end{itemize}
	\end{enumerate}

	Dependiendo de las características de la producción, el proceso de creación
	de patrones es, en general, mucho más complejo que la asignación óptima de 
	ellos. Al utilizarse algoritmos combinatoriales sujetos a las condiciones
	de la maquinaria el problema de hallar todos los patrones de corte posibles
	requiere mucho tiempo computacional. En el modelo matemático dejaremos de 
	lado la creación de patrones y nos enfocaremos directamente en la resolución
	del problema del corte, es decir: la optimización de productos generados 
	sujetos a los a un stock y patrones de corte suministrados.

\section{Modelo Matemático o LP}
	El enfoque principal de la literatura al cutting stock problem es para los
	casos de 1 y 2 dimensiones, siendo el caso 1D el punto de partida. Es por 
	esto que en primer lugar explicaremos el modelo matemático para el caso 
	de una dimensión, generalizando con el modelo para dos dimensiones. La
	siguiente formulación está basada en el estudio algorítmico del problema de 
	corte, realizado en la tesis de Delgadillo~\cite{Delgadillo:2007}
	
\subsection{Formulación Estándar, una dimensión}
	Supongamos que tenemos un stock de barras de largo $L$ a cortar en trozos 
	más pequeños de largo $l_i$, para satisfacer ciertas demandas $N_i$.
	Consideremos además que tenemos un conjunto de patrones $j$, con
	$j=1,2,...$, es decir, el $j$-ésimo patrón representa una forma de cortar un
	trozo de barra de largo $L$ para formar barras más pequeñas de largo $l_i$.
	Luego, nuestras variables serán los $x_j$ que representan la cantidad de
	veces que utilizaremos el patrón $j$.
	
	Así, tenemos la formulación más simple para el caso de una dimensión:
	\begin{align*}
		\text{Min  } &\sum\limits_{j} x_j\\
		\text{s.a.  } &\sum\limits_{j} a_{ij}x_j \geq N_i, \text{ i=1,...,m}\\
			&x_j \geq 0
	\end{align*}
	
	Gilmore y Gomory\cite{Gilmore:1961}, expresaron este modelo de forma 
	matricial quedando:
	
	\begin{align*}
		\text{Min  } &\vec{x}\\
		\text{s.a.  } &A\vec{x} \geq \vec{n}\\
			&\vec{x} \geq 0
	\end{align*}
	
	Donde:
	\begin{itemize}
		\item
			$\vec{x}$ representa un vector columna de las variables $x_j$, que
			representan la cantidad de veces que se utiliza el patrón $j$.
		\item 
			$A$ representa la matriz de $m$ filas, donde cada columna 
			$[a_1,a_2,...,a_m]^T$, representa un patrón de corte. El elemento
			$a_1$, por ejemplo, representa la cantidad de elementos de largo
			$l_1$ que se generan.
		\item
			$\vec{n}$ representa el vector columna de las demandas $N_i$
	\end{itemize}
	
	Podemos notar que el modelo en una dimensión puede ser reducido al
	planteamiento del problema de la mochila con la siguiente idea: Supongamos
	que nuestras barras de largo $L$ se transforman en una mochila que soporta
	un peso $P$, luego las secciones que podemos cortar de la barra (de largo
	$l_i$) serán equivalentes a los objetos, de peso $p_i$, que podemos llevar
	en nuestra mochila. El problema de corte además incluye "mochilas infinitas"
	y su solución implica la minimización de estas para transportar toda nuestra
	demanda y nuestros patrones serán las configuraciones utilizadas en dicha
	misión.
	
\subsection{Formulación Estándar, dos dimensiones}
	Para el caso 2D consideremos un stock de láminas de área $L \times W$,
	para formar otras de menor tamaño de área $l_i \times w_i$ (las cuales 
	llamaremos piezas) con el fin de satisfacer cierta demanda de $N_i$.
	
	Como dato adicional necesitamos los patrones de corte que nuestra maquinaria
	es capaz realizar para la creación de las láminas demandadas. Estos patrones 
	están determinados por la industria que quiere solucionar el problema. A
	veces es más costoso hacer el corte que el material desperdiciado por él,
	por lo que se optará por efectuar los mínimos cortes posibles aunque por
	ello pierda más material. En otras ocasiones ocurre todo lo contrario, el
	material es mucho más caro que la realización del corte, en estos casos
	generalmente, además de resolver el problema de corte, se debe resolver el
	problema de generación de todos los patrones posibles de manera de
	desperdiciar lo mínimo posible de material aunque se deban hacer muchos
	portes por cada lámina.

	El problema genérico busca minimizar la cantidad de láminas del ``stock''
	utilizadas y satisfacer la demanda de piezas.
	
	El modelo es similar al caso de una dimensión:
	
	\begin{align*}
		\text{Min  } &\vec{x}\\
		\text{s.a.  } &A\vec{x} \geq \vec{n}\\
			&\vec{x} \geq 0
	\end{align*}
	
	Donde:
	\begin{itemize}
		\item
			$\vec{x}$ representa un vector columna de las variables $x_j$, que
			indican la cantidad de veces que se utiliza el patrón $j$.
		\item 
			$A$ representa la matriz de $m$ filas, donde cada columna 
			$[a_1,a_2,...,a_m]^T$, representa un patrón de corte rectangular. 
			El elemento $a_1$, por ejemplo, representa las láminas de 
			área $l_1 \times w_i$ que se generan.
		\item
			$\vec{n}$ representa el vector columna de las demandas $N_i$.
	\end{itemize}
	
	El problema de corte tiene variantes dependiendo del contexto como
	mencionábamos anteriormente, para nuestro caso nos enfocaremos en el ``The
	assortment problem'' (Problema de selección), es cual busca minimizar el
	número de láminas del stock a utilizar.

	Básicamente el planteamiento a este problema es el modelo que hemos
	presentado, como extensión mostraremos los casos en que la maquinaria sólo
	puede realizar ciertos tipos de cortes por lo que tomaremos el tema de la
	incidencia de los patrones de corte en la solución óptima.
	
\subsection{Extensión}
	El modelo de extensión es el siguiente:
	\begin{align*}
		\text{Min  } &\vec{c}\vec{x}\\
		\text{s.a.  } &A\vec{x} \geq \vec{n}\\
			&\vec{x} \leq \vec{v}\\
			&\vec{x} \geq 0
	\end{align*}
	
	Donde:
	\begin{itemize}
		\item
			$\vec{x}$ representa un vector columna de las variables $x_j$, que
			indican la cantidad de veces que se utiliza el patrón $j$.
		\item
			$\vec{c}$ representa un vector columna con los costos $c_j$
			asociados a cada uno de los patrones de corte $x_j$.
		\item 
			$A$ representa la matriz de $m$ filas, donde cada columna 
			$[a_1,a_2,...,a_m]^T$, representa un patrón de corte rectangular. 
			El elemento $a_1$, por ejemplo, representa las láminas de 
			área $l_1 \times w_i$ que se generan.
		\item
			$\vec{n}$ representa el vector columna de las demandas $N_i$.
		\item
			$\vec{v}$: representa un vector con los límites de veces  $v_j$ en
			las que un patrón $x_j$ puede ser ejecutado.
	\end{itemize}
	Básicamente, la extensión consiste en un nuevo parámetro a considerar y una
	nueva restricción, es decir, nuestro problema ahora considera dos factores
	importantes en la industria: tiempo y costos. 
	
	Los costos $c_j$ representan que tanto le cuesta a la maquinaria realizar el
	patrón $a_j$, esto puede ser considerado en términos de energía, mano de
	obra necesaria, etc. Independiente de ello, este factor cambia las
	preferencias que antes teníamos, y la función objetivo ahora quiere
	minimizar los costos asociados a la generación de elementos más pequeños, en
	vez de reducir gastos por la utilización de materia prima.
	
	La restricción $\vec{x} \leq \vec{v}$ introduce otro factor asociado al
	tiempo que puede demorar la maquinaria en realizar un corte, por lo tanto,
	hay que tener en cuenta la cantidad de veces máxima que podemos ``ejecutar''
	el corte de un patrón, sobre todo cuando hay un tiempo límite en la que un
	pedido de producción debe ser satisfecho.
	
	Como vemos, el modelo es más restrictivo, pero se acerca más a una situación
	real con respecto al planteamiento estándar, normalmente querremos
	acercarnos lo más posible a la situación por lo que lo normal es que se
	agreguen este tipo de extensiones.
\section{Experimentación}
	\subsection{Entorno (Hardware y Software)}
		\emph{Detalles entorno de experimentación:}
		
		\begin{itemize}
		\item 
			Software utilizado: LINGO v 10.0, 2006
		\footnote{Proporcionado para el curso vía moodle}
		
		\item 
			Hadware: 
			\begin{itemize}
				\item 
					CPU: Intel Pentium G620 @ 2.60GHz
				\item 
					RAM: 4,00 GB DD3 @ 532 MHz
			\end{itemize}
		
		\item 
			Configuración del software: Por defecto.
		
		\end{itemize}
		
		\emph{Problema experimental (Parámetros del modelo)}
		
		Para la experimentación propondremos un problema simplificado para
		el modelo estándar y agregaremos detalles para el modelo extendido:
		
		\emph{Armando muebles para computadoras}
		
		Supongamos que una empresa desea armar muebles para computadores,
		las cuales se componen principalmente de tablas de madera de distintas
		dimensiones. La empresa desea optimizar la generación de estas piezas 
		utilizando la menor cantidad de materia prima posible. La información
		de la materia prima y de los componentes a generar se detallan a 
		continuación:
		
		\begin{enumerate}
			\item
				Dimensión tabla ``Grande'': $100[cm] \times 100[cm]$
		
			\item	
				Demanda: 100 muebles
		
			\item
				Tablas de madera para construir 1 mueble:
				
				\begin{itemize}
				\item $a1 = 45[cm] \times 100[cm]$ (x 1)
				\item $a2 = 45[cm] \times 75[cm]$ (x 3)
				\item $a3 = 25[cm] \times 100[cm]$ (x 1)
				\item $a4 = 45[cm] \times 65[cm]$  (x 1)
				\item $a5 = 45[cm] \times 35[cm]$ (x 1)
				\end{itemize}
			
			\item	
				Patrones de corte:
		
				\begin{itemize}
					\item
						$x1 = a4$ ( x2), $a3$ ( x1), $a5$ ( x1)
					\item
						$x2 = a4$ ( x1), $a1$ ( x1), $a3$ ( x1)
					\item
						$x3 = a4$ ( x1), $a2$ ( x2)   
					\item
						$x4 = a1$ ( x1), $a3$ ( x1), $a4$ ( x1)  
					\item
						$x5 = a2$ ( x1), $a3$ ( x2), $a5$ ( x1)
					\item
						$x6 = a3$ ( x2), $a5$ ( x2)   
					\item
						$x7 = a5$ ( x2), $a2$ ( x2)  
					\item
						$x8 = a5$ ( x4), $a2$ ( x1) 
					\item
						$x9 = a5$ ( x3), $a3$ ( x2)
					\item
						$x10 = a1$ ( x1), $a2$ ( x1), $a5$ ( x1) 
				\end{itemize}
		\end{enumerate}
			Y para la extensión agregamos:
			\begin{enumerate}
			\item 
				Costos de ejecutar cada patrón:
				\begin{itemize}
					\item
						$x1 = 7$
					\item
						$x2 = 4$
					\item
						$x3 = 10$   
					\item
						$x4 = 2  $
					\item
						$x5 = 6$
					\item
						$x6 = 7 $  
					\item
						$x7 = 5$
					\item
						$x8 = 3$
					\item
						$x9 = 6$
					\item
						$x10 = 9$
				\end{itemize}
			\item 
				Límite de ejecución de un patrón (diarios):
				\begin{itemize}
					\item
						$x1 = 70$
					\item
						$x2 = 25$
					\item
						$x3 = 50$   
					\item
						$x4 = 83  $
					\item
						$x5 = 71$
					\item
						$x6 = 37 $  
					\item
						$x7 = 65$
					\item
						$x8 = 59$
					\item
						$x9 = 44$
					\item
						$x10 = 40$
				\end{itemize}
			\end{enumerate}
	\subsection{Resultados modelo estándar} 
		A continuación mostramos el resultado obtenido por el software para
		el problema propuesto:
		\begin{verbatim}
    Global optimal solution found.
    Objective value:  200.0000
    Total solver iterations: 2
    
    
    Variable  Value       Reduced Cost
    X1        0.000000    1.000000
    X2        0.000000    0.5000000
    X3        100.0000    0.000000
    X4        0.000000    0.5000000
    X5        0.000000    0.5000000
    X6        0.000000    1.000000
    X7        0.000000    0.000000
    X8        0.000000    0.5000000
    X9        0.000000    1.000000
    X10        100.0000   0.000000
		\end{verbatim}
		La resolución del tiempo en el software es en segundos, por
		lo tanto no pudimos obtener el tiempo exacto de ejecución.

	\subsection{Resultados modelo extendido}
		Para el modelo extendido el software nos da el siguiente resultado:
		\begin{verbatim}
    Global optimal solution found.
    Objective value: 1291.000
    Total solver iterations: 3
    
    
    
    Variable    Value    Reduced Cost
    X1          0.000000    7.000000
    X2          6.000000    0.000000
    X3          50.00000    0.000000
    X4          83.00000    0.000000
    X5          0.000000    1.000000
    X6          0.000000    7.000000
    X7          65.00000    0.000000
    X8          59.00000    0.000000
    X9          0.000000    6.000000
    x10         11.00000    0.000000
		\end{verbatim}
		Como mencionamos anteriormente, la resolución de tiempo del
		software no alcanza para registrar el tiempo de ejecución.
		
\section{Análisis de Resultados}
	En el modelo estándar vemos como el software escoge el mínimo
	número de patrones que generan las tablas solicitadas y de estas
	suplir la demanda, aparentemente muestra una preferencia por los
	patrones que abarcan o cubren más sub-elementos. 
	
	En el segundo caso, dada las restricciones agregadas, las preferencias
	cambian lógicamente acorde al beneficio que se espera y a las limitantes
	que acompañan a la ejecución de cada patrón.
	
	Si bien no son comparables los resultados de las funciones objetivo dado
	que apuntan a cosas distintas, estas no son de naturaleza tan lejana, dado
	que implícitamente apuntan a abaratar costos de producción.
	
	El cambio de un modelo a otro pasa simplemente por lo que se considere más
	relevante en la cadena de producción de sub-elementos, por lo que de ello 
	dependerá las preferencias que al final nos muestre el resultado de la
	resolución del problema.

\section{Conclusiones y trabajo futuro}
	Parte del trabajo futuro consiste en la generación de patrones. Es aquí
	donde radica la raíz del problema de corte o más bien gran parte de su 
	complejidad. Sin embargo, de omitir este paso y considerar que los patrones 
	de corte ya están establecidos, el problema de optimización 
	tomando en cuenta ciertas características (tipo de corte, coste asociado, 
	uso de materia prima, etc.) está prácticamente resuelto, quedando en función 
	del contexto las restricciones y los parámetros involucrados.
	
	En general, en la literatura del problema, y en particular algunos de los
	autores mencionados en la sección~\ref{sec:arte}, hacen referencia a
	algoritmos para la generación de patrones y la optimización de los mismos,
	sin embargo sólo hicimos mención de ellos debido a que tienen una gran 
	complejidad y su estudio es más interesante como un problema de 
	combinatoria que como un problema de investigación de operaciones. Aún así
	la escala de la mano-factura actual hace que la cantidad de piezas 
	solicitadas superen los cientos, en este contexto, los algoritmos que 
	generan buenos patrones en tiempos acotados son de vital importancia.
	
	Por otro lado, se puede solucionar el problema de corte con modelos que no 
	sean de programación lineal. La programación dinámica, las heurísticas y las
	meta-heurísticas son ampliamente estudiadas para los problemas de 
	optimización y tienen gran valor como instrumento de estudio para este
	tipo de problemas.

\section{Referencias}
\bibliographystyle{plain}
\renewcommand{\section}[2]{}% Para que no salga el título de ref.
\begin{thebibliography}{11}
	\bibitem{Tzu-Yi Yu:2014}
		Tzu-Yi Yu, Jiann-Cherng Yang, Yeong-Lin Lai, Chih-Cheng Chen and Han-Yu
		Chang.
		\newblock{\em Applying an Enhanced Heuristic Algorithm to a Constrained 
		Two-Dimensional Cutting Stock Problem.}
		\newblock{26 Mar. 2014.}\\
		\newblock{\url{http://www.naturalspublishing.com/files/published/7z5uf8oq7y668r.pdf}}
	
	\bibitem{Delgadillo:2007}
		Rosa Sumactika Delgadillo Avila.
		\newblock{\em Un estudio algorítmico del problema de corte y
		empaquetado 2D.}
		\newblock{Lima, Perú 2007.}\\
		\newblock{\url{http://cybertesis.unmsm.edu.pe/bitstream/cybertesis/1518/1/delgadillo_ar.pdf}}
		
	\bibitem{Karp:1972}
		Richard M. Karp.
		\newblock {\em Reducibility Among Combinatorial Problems}

	\bibitem{Gilmore:1961}
		P. Gilmore and R. Gomory.
		\newblock {\em A linear programming approach to the cutting stock 
		problem}.

	\bibitem{Gilmore:1963}
		P. Gilmore and R. Gomory.
		\newblock {\em A linear programming approach to the cutting stock 
		problem - Part II}.

	\bibitem{Gilmore:1965}
		P. Gilmore and R. Gomory.
		\newblock {\em Multistage Cutting Stock Problems of Two and More
		Dimensions}.

	\bibitem{Beasley:1985}
		J. E. Beasley.
		\newblock {\em Algorithms for unconstrained two-dimensional guillotine
		cutting}.
		\newblock Journal of the Operational Research Society, 1985.

	\bibitem{Chris:1977}
		N Christofides, C Whitlock.
		\newblock {\em An algorithm for two-dimensional cutting problems}.
		\newblock Operations Research, 1977.

	\bibitem{Viswa:1993}
		KV Viswanathan, A Bagchi.
		\newblock {\em Best-first search methods for constrained two-dimensional
		cutting stock problems}.
		\newblock Operations Research, 1993.

	\bibitem{Hifi:1997}
		M Hifi, R Ouafi.
		\newblock {\em Best-first search and dynamic programming methods for
		cutting problems: the cases of one or more stock plates}.
		\newblock Computers \& industrial engineering, 1997.

	\bibitem{Wang:1983}
		PY Wang
		\newblock {\em Two algorithms for constrained two-dimensional cutting
		stock problems}.
		\newblock Operations Research, 1983.

	\bibitem{Vasco:1989}
		FJ Vasko.
		\newblock {\em A computational improvement to Wang's two-dimensional
		cutting stock algorithm}.
		\newblock Computers \& industrial engineering, 1989.

	\bibitem{Olib:1990}
		J. F. Oliveira, J. S. Ferreira.
		\newblock {\em An improved version of Wang's algorithm for
		two-dimensional cutting problems}.
		\newblock European Journal of Operational Research, 1990.

	\bibitem{Valdes:2000}
		A. P. Guevara, J. M. T. Goerlich, R. A Valdés.
		\newblock {\em Algoritmos Grasp y Tabu Search para el problema de corte
		guillotina bidimendional}.
		\newblock XXV Congreso Nacional de Estadística e Investigación
		Operativa: Vigo, 2000.

	\bibitem{Golver:1997}
		F. Golver.
		\newblock {\em Tabu search and adaptive memory programming — advances,
		applications and challenges}.
		\newblock Interfaces in computer science and operations research, 1997.

	\bibitem{Feo:1989}
		T. A. Feo, M. G. C. Resende.
		\newblock {\em A probabilistic heuristic for computationally difficult
		set covering problems}.
		\newblock Operation Research Letters, 1989.

	\bibitem{Holland:1975}
		J. E. Holland.
		\newblock {\em Adaptation in natural and artificial systems: An 
		introductory analysis with applications to biology, control, and 
		artificial intelligence}.
		\newblock 1975.

	\bibitem{Kirk:1983}
		S. Kirkpatrick, C. D. Gelatt, M. P. Vecchi.
		\newblock {\em Optimization by simmulated annealing}.
		\newblock science, 1983.

	\bibitem{Cerny:1985}
		V. Černý.
		\newblock {\em Thermodynamical approach to the traveling
		salesman problem: An efficient simulation algorithm}.
		\newblock Journal of optimization theory and applications,
		1985.

	\bibitem{Lai:1997}
		K. K. Lai, J. W. M. Chan.
		\newblock {\em Developing a simulated annealing algorithm 
		for the cutting stock problem}.
		\newblock Computers \& industrial engineering, 1997.

	\bibitem{Faina:1999}
		L. Faina.
		\newblock {\em An application of simulated annealing to the 
		cutting stock problem}.
		\newblock European Journal of Operational Research, 1999.
		
	\bibitem{Parada:1996}
		V. Parada, M. Sepúlveda, M. Solar, A. Gómes.
		\newblock {\em Solution for the constrained guillotine cutting 
		problem by simulated annealing}.
		\newblock Computers \& Operations Research, 1998.

	\bibitem{Jackobs:1993}
		S. Jakobs.
		\newblock {\em On genetic algorithms for the packing of polygons}.
		\newblock European Journal of Operational Research, 1996. 
	
	\bibitem{Wu:2006}
		Wu Tai-Hsi, Wu Yi-Hua, Chang Chin-Ju.
		\newblock {\em Solving Two-Dimensional Packing Problems by 
		Using a Genetic Algorithm.}.
		\newblock Journal of Science and Engineering Technology, 2006.

	\bibitem{Gon:2007}
		JF Gonçalves.
		\newblock {\em A hybrid genetic algorithm-heuristic for a 
		two-dimensional orthogonal packing problem}.
		\newblock European Journal of Operational Research, 2007.
	
	\bibitem{Dyckhoff:1990}
		H. Dyckhoff.
		\newblock {\em A typology of cutting and packing problems}.
		\newblock European Journal of Operational Research, 1990.	
\end{thebibliography}

\end{document}
