Primero, normalizaremos el modelamiento:
\begin{align*}
\textbf{F.O.:} \quad \max z &= 25{x_1} + 15{x_2} + 16{x_3} + 0{s_1} + 0{s_3} + 0{s_4} - M{a_2}
\end{align*}
\vspace{-1cm}
\begin{align*}
	\textbf{S.T.:} \quad
		4{x_2} + 8{x_3} + s_1 &= 1600\\
		10{x_1} + 2{x_2} + a_2 &= 2100\\ 
		x_3 + s_3 &= 300\\ 
		x_2 + s_4 &= 250
\end{align*}

\begin{enumerate}
	\item
		Como $x_2 = 250$ y $x_3 = 75$, tenemos para la restricción 1:
		\begin{align*}
			\hspace{4cm}
				4\cdot250 + 8\cdot75 + s_1 &= 1600 \\ 
				1600 + s_1 &= 1600 \\
				s_1 &= 0
		\end{align*}
		Por lo tanto, $s_1$ no es basal, se ocupan todos los 
		recursos en la restricción 1.

	\item 
		Modificando el coeficiente de $x_2$ en el tableau:
		%%%Tableau
		\begin{table}[ht]
			\centering
			\begin{tabular}{ cc|ccccccc|c }
				& & $x_1$ & $x_2$ & $x_3$ & $s_1$ & $a_2$ & $s_3$ & $s_4$ &\\ 
				Base & $c_j$ & 25 & $15 + \delta$ & 16 & 0 & -M & 0 & 0 & $b_j$ \\
				\hline
				$x_3$ & 16 & 0 & 0 & 1 & 1/8 & 0 & 0 & -1/2 & 75 \\ 
				$x_1$ & 25 & 1 & 0 & 0 & 0 & 1/10 & 0 & -1/5 & 160 \\
				$s_3$ & 0  & 0 & 0 & 0 & -1/8 & 0 & 1 & 1/2 & 225 \\
				$x_2$ & $15 + \delta$ & 0 & 1 & 0 & 0 & 0 & 0 & 1 & 250 \\
				\hline
				& $z_j$ & 25 & $15 + \delta$ & 16 & 2 & 5/2 & 0 & $2 + \delta$ &  \\ 
				& $c_j - z_j$ & 0 & 0 & 0 & -2 & -M-5/2 & 0 & $-2-\delta$ & \\ 
			  \end{tabular}
		\end{table}
		\\Luego el rango de insignificancia para $x_2$ es:
		\begin{align*}
			\hspace{5.33cm}
				-2-\delta &\leq 0  	\quad\emph{(caso max en el que no seguimos iterando)}.\\
				-\delta &\leq 2 \\
				\delta &\geq -2 \\ 
				+\infty &\geq \delta \geq -2 	\quad\emph{(rango de insignificancia)}.
		\end{align*}
		Si el coeficiente de $x_2$ cambia a 10:
		\begin{align*}
			\hspace{5.33cm}
				\Delta{c_j} &= c_j' - c_j \\
			  	\Delta{c_j} &= 10 - 15 \\ 
			  	\Delta{c_j} &= -5 = \delta
		\end{align*}
	  	$-5$ está fuera del rango de insignificancia, por tanto: $-2 - \delta = -2 + 5 = 3 \geq 0$.
		\\Es decir, $s_4$ entra a la base, calculando los $b_j/a_{ij}:$
		\begin{table}[ht]
			\centering
			\begin{tabular}{ cc|cc }
				... & $s_4$ & & \\
				... & 0& $b_j$ & $b_j/a_{ij} $\\
				\hline
				... & -1/2 & 75 & -\\
				... & -1/5 & 160 & - \\
				... & 1/2 & 225 & \textcolor{red}{225/2} \\
				... & 1 & 250 & 250 \\
				\hline
				... & -3 & & \\
				... & 3 & & \\
			\end{tabular}
		\end{table}
		\\Por lo tanto, cambia nuestro óptimo (se debe volver a iterar) y la restricción 3 se vuelve limitante
		(se sale de la base, por lo tanto vale 0 lo que quiere decir que se usan todos los recursos en dicha
		restricción.)
	\item
		\begin{itemize}
			\item 
				\textbf{No activa:} Significa variable artificial $\neq 0$ o
				que está en la base: $s_3 = 225 \Rightarrow$ Restricción 3 es
				inactiva.
			\item 
				\textbf{Activa:} Caso contrario al anterior, restricciones 1,2 
				y 4 no están en base por ende sus variables artificiales son 0,
				por ello son restricciones activas.
		\end{itemize}
	\item Tenemos que:
		\begin{align*}
			%%matriz
			\left[{
				\begin{array}{c}
					Nueva\\
					\\
					soluci\acute{o}n\\			
				\end{array} 
			} \right] 
			=
			\left[{
				\begin{array}{c}
					Soluci\acute{o}n\\
					\\
					actual\\
				\end{array} 
			} \right] 
			+
			\Delta{b_i}a_{ij} \geq 0
		\end{align*}
		Reemplazando por los valores correspondientes a la solución actual y a
		los coeficientes de la variable artificial correspondiente a la
		restricción 1 y naturaleza de las variables, encontraremos el rango para
		$\Delta{b_i}$:.
		\begin{align*}
			\hspace{5.33cm}
				\left[{
					\begin{array}{c}
						75\\
						160\\
						225\\
						250	
					\end{array} 
				} \right] 
				+
				\Delta{b_1}
				\left[{
					\begin{array}{c}
						1/8\\
						0\\
						-1/8\\
						0	
					\end{array} 
				} \right]  
				\geq 0
		\end{align*}
		Luego, el rango para $b_1$:
		\begin{align*}
			\hspace{4cm}
				75 + \Delta{b_1}/8 &\geq 0 \quad \rightarrow \quad \Delta{b_1} \geq -600\\
				225 - \Delta{b_1}/8 &\geq 0 \quad \rightarrow \quad \Delta{b_1} \leq 1800\\
				-600 &\leq \Delta{b_1} \leq 1800 \quad\emph{(Rango dentro del cual no cambia la base)}
		\end{align*}
		Si la solución varía en +1000:
		\begin{align*}
			%%matriz
			\left[{
				\begin{array}{c}
					Nueva\\
					\\
					soluci\acute{o}n\\			
				\end{array} 
			} \right] 
			=
			\left[{
				\begin{array}{c}
					75\\
					160\\
					225\\
					250
				\end{array} 
			} \right] 
			+
			1000
			\left[{
				\begin{array}{c}
					1/8\\
					0\\
					-1/8\\
					0	
				\end{array} 
			} \right]
			=
			\left[{
				\begin{array}{c}
					75 + 1000/8\\
					160\\
					225 - 1000/8\\
					250	
				\end{array} 
			} \right]
			=
			\left[{
				\begin{array}{cc}
					200 & x_3*\\
					160 & x_1*\\
					100 & s_3*\\
					250 & x_2*	
				\end{array} 
			} \right]
			\quad \emph{Nueva solución}
		\end{align*}
		Evaluando en la \textbf{F.O.}: $25\cdot160 + 15\cdot250 + 16\cdot200 + 0\cdot100 = 10950$\\
		{\medskip}\\
		\textbf{Casos Extremos}
		\begin{description}
			\item[Caso 1] $\Delta{b_1} = -600$\\
				\begin{align*}
					%%sol nueva
					\left[{
						\begin{array}{c}
								Nueva\\
							\\
							soluci\acute{o}n\\			
						\end{array} 
					} \right] 
					=
					%%vieja solución
					\left[{
						\begin{array}{c}
							75\\
							160\\
							225\\
							250
						\end{array} 
					} \right] 
					-
					600
					\left[{
						\begin{array}{c}
							1/8\\
							0\\
							-1/8\\
							0	
						\end{array} 
					} \right]
					=
					%%sol final
					\left[{
						\begin{array}{c}
							75 - 600/8\\
							160\\
							225 + 600/8\\
							250	
						\end{array} 
					} \right]
					=
					\left[{
						\begin{array}{cc}
							0 & x_3*\\
							160 & x_1*\\
							300 & s_3*\\
							250 & x_2*	
						\end{array} 
					} \right]
					\quad \emph{Nueva solución}
				\end{align*}
				En este caso, $x_3$ sale de la base.
				\textbf{F.O.} $z = 25\cdot160 + 15\cdot250 + 16\cdot0 + 0\cdot300 = 7750$
			\item[Caso 2] $\Delta{b_1} = 1800$\\
				\begin{align*}
					%%sol nueva
					\left[{
						\begin{array}{c}
								Nueva\\
							\\
							soluci\acute{o}n\\			
						\end{array} 
					} \right] 
					=
					%%vieja solución
					\left[{
						\begin{array}{c}
							75\\
							160\\
							225\\
							250
						\end{array} 
					} \right] 
					+
					1800			
					\left[{
						\begin{array}{c}
							1/8\\
							0\\
							-1/8\\
							0	
						\end{array} 
					} \right]
					=
					%%sol final
					\left[{
						\begin{array}{c}
							75 + 1800/8\\
							160\\
							225 - 1800/8\\
							250	
						\end{array} 
					} \right]
					=
					\left[{
						\begin{array}{cc}
							300 & x_3*\\
							160 & x_1*\\
							0 & s_3*\\
							250 & x_2*	
						\end{array} 
					} \right]
					\quad \emph{Nueva solución}
				\end{align*}
				En este caso, $s_3$ sale de la base.
				\textbf{F.O.} $z = 25\cdot160 + 15\cdot250 + 16\cdot300 + 0\cdot0 = 12550$
		\end{description}
	\item
		Evaluemos las variables implicadas en la nueva restricción ($x_1 = 160, x_2=250, x_3=75$):
		\begin{align*}
			\hspace{4cm}
				6x_1 + 3x_2 + 4x_3 &\leq 1900\\
				6\cdot160 + 3\cdot250 + 4\cdot75 &= 2010 \geq 1900
		\end{align*}
		Luego, no se cumple la restricción. Para resolver el problema, debemos retroceder en el tableau y chequear la restricción
		hasta que se cumpla, luego volver a iterar hasta hallar la nueva solución.
\end{enumerate}
