\begin{itemize}
	\item
		\textbf{Parte 1:} Si bien en un principio el modelo aparenta tener varias
		variables, un análisis adecuado nos permite llegar a las que realmente 
		influyen en el proceso de optimización. Por otra parte, se hacen
		evidentes las limitaciones del método gráfico, si bien su carácteristica
		principal es ser bastante claro al momento de visualizar la región
		factible y las soluciones, cuando el modelo se vuelve más complejo, el
		método se vuelve inútil.
	\item
		\textbf{Parte 2:} Sabemos que hay muchas formas de modelar, por lo mismo
		hay que considerar varias opciones al momento de escoger un modelo. En
		un principio nuestro modelo de optimización era complejo, sin embargo,
		con las debidas modificaciones pudimos simplificar el problema y
		resolver el tableu en menos iteraciones. A esto, el software nos 
		permitió ''jugar'' de manera rápida con las variables y restricciones,
		de manera que pudimos comprobar nuestras sospechas y verificar que el
		modelo era correcto.
	\item
		\textbf{Parte 3:} El análisis de sensibilidad nos permite analizar de
		manera directa los cambios en varias zonas de un modelo, ya sean
		recursos, variables nuevas, restricciones nuevas, etc. Este proceso nos
		permite ahorrar bastante tiempo dado que no es necesario volver a
		resolver el modelo completo. Por otro lado, nos permite tener un control
		más claro, a la hora de tomar decisiones respecto a las variables que
		modifican más nuestras predicciones.
\end{itemize}
