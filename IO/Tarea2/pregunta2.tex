%tex_origin: Tarea2-IO1-SandovalVargas.tex
\section{Ranking Top, difusión nacional.}
\begin{enumerate}
	\item
		\textbf{SUPUESTO: Solo se puede poner una antena por región.} \\
		Las variables binarias serán la existencia de la antena y su respectiva 
		ampliación. Por simplicidad nos referiremos a la región metropolitana
		por el número 0. Además ordenaremos los datos suministrados de la misma
		manera. Tenemos entonces:
		\begin{align*}
			x_i \forall i \in [0,15] \subset \mathbb{N} \quad 
				&\text{Antena en la región $i$.} \\
			y_i \forall i \in [0,15] \subset \mathbb{N} \quad
				&\text{Ampliación de la antena $i$.}\\
			C_i \forall i \in [0,15] \subset \mathbb{N} \quad
				&\text{Costo de construcción en la región $i$.}\\
			D_i \forall i \in [0,15] \subset \mathbb{N} \quad
				&\text{Demanda de la región $i$.}
		\end{align*}
		La función objetivo será aquella que minimice los costos de 
		construcción, luego:
		$$ \text{\textbf{F.O.:}}\min Z=\sum_0^{15}(x_i\cdot C_i +2\cdot y_i) $$
		Ya que el costo de ampliación es constante lo dejamos explícitamente
		expresado.\\
		Por ultimo tenemos las restricciones de demanda y de cableado en
		Santiago. Para Santiago simplemente basta con establecer una restricción
		por región afectada (dos en total). Mientras que para la restricción de
		demanda debemos relacionar las antenas de cada región con las de las
		regiones adyacentes y sus respectivas demandas. Queda expresado como:
		$$  x_0 + x_5 \leq 1 \quad \land \quad x_0 + x_6 \leq 1 
			\quad \text{Restricciones para Santiago.} $$
		$$	D_i \leq C_i \cdot x_i + 400 \cdot y_i +
			C_{i+1} \cdot x_{i+1} + 400 \cdot y_{i+1} - D_{i+1} +
			C_{i-1} \cdot x_{i-1} + 400 \cdot y_{i-1} - D_{i-1} $$
		 $$	\forall i \in [0,15] \quad \text{Restricciones de demanda con:}
		 \quad x_{-1} = y_{-1} = x_{16} = y_{16} = 0 $$ 
		 Nuevamente como el aumento de abastecimiento por la ampliación de una 
		 antena es constante se escribe explícitamente.
	\item
		De la ejecución del código (ver anexo) obtenemos los siguientes 
		resultados:
		\verbinput{p2-IO-lab2-output.txt}
\end{enumerate}
