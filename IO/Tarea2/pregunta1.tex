%tex_origin: Tarea2-IO1-SandovalVargas.tex
\section{Guaripolo S.A.}
\begin{enumerate}
	\item \textbf{Malla del programa, ruta crítica y duración esperada.} \\
		La malla queda representada por la siguiente figura, donde se asignan
		números a los nodos y letras a los trabajos (las lineas punteadas son
		\texttt{dummies}):
		\begin{center}
			%tex_origin: Tarea2-IO1-SandovalVargas.tex
\begin{tikzpicture}[shorten >=1pt, node distance=3cm, on grid, >=stealth, 
		every state/.style={draw=navyblue!50, very thick,
		fill=navyblue!20}, bend angle=25]
	%%%%    NODOS    %%%%
	\node[state]			(x0)						{$0$};
	\node[state]			(x1)	[above right=of x0]	{$1$};
	\node[state]			(x2)	[below right=of x0]	{$2$};
	\node[state]			(x3)	[right=of x1]		{$3$};
	\node[state]			(x4)	[right=of x2]		{$4$};
	\node[state]			(x5)	[right=of x3]		{$5$};
	\node[state]			(x6)	[right=of x4]		{$6$};
	\node[state]			(x7)	[above right=of x6]	{$7$};
	\node[state]			(x8)	[above right=of x7]	{$8$};
	\node[state]			(x9)	[below right=of x7]	{$9$};
	\node[state]			(x10)	[right=of x9]		{$10$};
	\node[state, accepting]	(x11)	[right=of x8]		{$11$};
	%%%%    CONEXIONES    %%%%
	\path[->]
		(x0) edge 	node[above left]{$A$}	node[below right]{$(0,4)$}	(x1)
		(x0) edge 	node[above right]{$B$}	node[below left]{$(0,2)$}	(x2)
		(x1) edge 	node[above]{$C$}		node[below]{$(4,7)$}		(x3)
		(x2) edge 	node[above left]{$D$}	node[below right]{$(2,6)$}	(x3)
		(x3) edge 	node[left]{$F$}			node[right]{$(7,8)$}		(x4)
		(x4) edge 	node[above left]{$G$}	node[below right]{$(8,9)$}		(x5)
		(x3) edge 	node[above]{$H$}		node[below]{$(7,10)$}		(x5)
		(x5) edge 	node[below left]{$I$}	node[above right]{$(10,15)$}	(x7)
		(x6) edge 	node[above left]{$J$}	node[below right]{$(10,13)$}	(x7)
		(x7) edge 	node[below left]{$K$}	node[above right]{$(15,17)$}	(x9)
		(x7) edge 	node[above left]{$L$}	node[below right]{$(15,20)$}	(x8)
		(x9) edge 	node[above]{$M$}		node[below]{$(17,19)$}	(x10)
		(x8) edge 	node[above]{$N$}		node[below]{$(20,24)$}	(x11)
		(x10) edge	node[left]{$O$}			node[right]{$(20,25)$}	(x11)
		%%%   DUMMIES   %%%
		(x2) edge[dashed]	(x1)
		(x5) edge[dashed]	(x6)
		(x8) edge[dashed]	(x10)
		(x3) edge[dashed, bend left]	(x6);
\end{tikzpicture}

		\end{center}
		Podemos notar que la ruta crítica será:
		$$ A\rightarrow C\rightarrow H\rightarrow I\rightarrow L\rightarrow O $$
		Con una duración esperada de $25$ días.
	\item
		Como notamos del resultado anterior el programa no finaliza en los $22$
		días que espera nuestro gerente de operaciones, por lo que es necesario
		acelerar el proceso. Para ello analizamos la tabla presentada y notamos 
		que:
		\begin{itemize}
			\item
				De la ruta crítica el trabajo con menor costo de aceleración es
				$I$, por lo tanto lo aceleramos un día.
			\item
				Como la aceleración de $I$ no produjo un cambio de ruta crítica,
				y además no podemos seguir acelerando $I$ pasamos al siguiente
				menor que es $L$. Lo aceleramos un día.
			\item
				El último cambio afectó la ruta crítica: desde $I$ se
				puede llegar a $O$ ya sea por $L$ o por $K\rightarrow M$ sin
				cambiar su duración. Ahora un cambio en $L$ no afectará la ruta
				crítica a no ser que a su vez se cambie $K$ o $M$, lo que nos
				costará un mínimo de $\$ 15000 $.
			\item
				Vemos que en la ruta crítica tenemos un elemento con menor
				precio: $H$, lo aceleramos un día y obtenemos el resultado
				esperado.
			\item
				El costo de la aceleración fue de $5000 + 7000 + 8000 = 20000$
		\end{itemize}
		La nueva malla del programa será:
		\begin{center}
			\input{malla2.tex}
		\end{center}
		Las nueva ruta críticas serán:
		$$ A\rightarrow C\rightarrow H\rightarrow I\rightarrow L\rightarrow O $$
		$$ A\rightarrow C\rightarrow H\rightarrow I\rightarrow K\rightarrow 
			M\rightarrow O $$
		$$ A\rightarrow C\rightarrow F\rightarrow G\rightarrow I\rightarrow 
			L\rightarrow O $$
		$$ A\rightarrow C\rightarrow F\rightarrow G\rightarrow I\rightarrow 
			K\rightarrow M\rightarrow O $$
	\item
		De la resolución del problema usando \texttt{lp\_solve} (ver código 
		anexo) obtenemos el siguiente resultado:
		\verbinput{p1-IO-lab2-output.txt}
		El cual no varía del resultado obtenido sin el programa.
\end{enumerate}
