%tex_origin: Tarea2-IO1-SandovalVargas.tex
\section{Triviño a senador.}
\begin{enumerate}
\item
	Grafo de asignaciones con sus respectivas etapas de decisión.\\
	{\medskip}
	Etapa 1 \hfill Etapa 2 \hfill Etapa 3 \hfill Etapa 4
	\begin{center}
	%tex_origin: Tarea2-IO1-SandovalVargas.tex
\begin{tikzpicture}[shorten >=1pt, node distance=4cm, on grid, >=stealth, 
		every state/.style={draw=navyblue!50, very thick,
		fill=navyblue!20}, bend angle=25]
	%%%%    NODOS    %%%%
	\node[state]	(x0)				{$5$};
	%Primera etapa
	\node[state]	(x1)	[right =of x0]		{$3$};
	\node[state]	(x2)	[above =of x1]		{$4$};
	\node[state]	(x3)	[above =of x2]		{$5$};
	\node[state]	(x4)	[below =of x1]		{$2$};
	\node[state]	(x5)	[below =of x4]		{$1$};
	\node[state]	(x6)	[below =of x5]		{$0$};
	%segunda etapa
	\node[state]	(x7)	[right=of x1]		{$3$};
	\node[state]	(x8)	[right=of x2]	 	{$4$};
	\node[state]	(x9)	[right=of x3]		{$5$};
	\node[state]	(x10)	[right=of x4]		{$2$};
	\node[state]	(x11)	[right=of x5]		{$1$};
	\node[state]	(x12)	[right=of x6]		{$0$};
	%tercera etapa
	\node[state]	(x13)	[right=of x7]		{$3$};
	\node[state]	(x14)	[right=of x8]	 	{$4$};
	\node[state]	(x15)	[right=of x9]		{$5$};
	\node[state]	(x16)	[right=of x10]		{$2$};
	\node[state]	(x17)	[right=of x11]		{$1$};
	\node[state]	(x18)	[right=of x12]		{$0$};
	%cuarta etapa
	\node[state, accepting]	(x20)	[right=of x13]		{$0$};

	%%%%    CONEXIONES    %%%%
	\path[->]
		%primera etapa
		(x0) edge 	node[above left]{$0$}	(x3)
		(x0) edge 	node[above left]{$4$}	(x2)
		(x0) edge 	node[above]{$7$}	(x1)
		(x0) edge 	node[above right]{$9$}	(x4)
		(x0) edge 	node[above right]{$12$}	(x5)
		(x0) edge 	node[above left]{$15$}	(x6)
		%Segunda etapa
		(x3) edge 	node[blue, pos=0.2,above left]{$0$}	(x9)
		(x3) edge 	node[blue, pos=0.2,above left]{$6$}	(x8)
		(x3) edge 	node[blue, pos=0.2,above]{$8$}		(x7)
		(x3) edge 	node[blue, pos=0.2,above]{$10$}		(x10)
		(x3) edge 	node[blue, pos=0.2,above]{$11$}		(x11)
		(x3) edge 	node[blue, pos=0.13,above left]{$12$}	(x12)

		(x2) edge 	node[red, pos=0.2,above left]{$0$}	(x8)
		(x2) edge 	node[red, pos=0.2,above]{$6$}		(x7)
		(x2) edge 	node[red, pos=0.2,above]{$8$}		(x10)
		(x2) edge 	node[red, pos=0.2,above]{$10$}		(x11)
		(x2) edge 	node[red, pos=0.13,above left]{$11$}	(x12)

		(x1) edge 	node[green, pos=0.2,above]{$0$}		(x7)
		(x1) edge 	node[green, pos=0.2,above]{$6$}		(x10)
		(x1) edge 	node[green, pos=0.2,above]{$8$}		(x11)
		(x1) edge 	node[green, pos=0.13,above left]{$10$}	(x12)

		(x4) edge 	node[purple, pos=0.2,above]{$0$}	(x10)
		(x4) edge 	node[purple, pos=0.2,above]{$6$}	(x11)
		(x4) edge 	node[purple, pos=0.2,above left]{$8$}	(x12)
		
		(x5) edge 	node[orange, pos=0.2,above]{$0$}	(x11)
		(x5) edge 	node[orange, pos=0.2,above]{$6$}	(x12)

		(x6) edge 	node[above left]{$0$}	(x12)

		%Tercera etapa
		(x9) edge 	node[blue, pos=0.2,above left]{$0$}	(x15)
		(x9) edge 	node[blue, pos=0.2,above left]{$5$}	(x14)
		(x9) edge 	node[blue, pos=0.2,above]{$9$}		(x13)
		(x9) edge 	node[blue, pos=0.2,above]{$11$}		(x16)
		(x9) edge 	node[blue, pos=0.2,above]{$10$}		(x17)
		(x9) edge 	node[blue, pos=0.13,above left]{$9$}	(x18)

		(x8) edge 	node[red, pos=0.2,above left]{$0$}	(x14)
		(x8) edge 	node[red, pos=0.2,above]{$5$}		(x13)
		(x8) edge 	node[red, pos=0.2,above]{$9$}		(x16)
		(x8) edge 	node[red, pos=0.2,above]{$11$}		(x17)
		(x8) edge 	node[red, pos=0.13,above left]{$10$}	(x18)

		(x7) edge 	node[green, pos=0.2,above]{$0$}		(x13)
		(x7) edge 	node[green, pos=0.2,above]{$5$}		(x16)
		(x7) edge 	node[green, pos=0.2,above]{$9$}		(x17)
		(x7) edge 	node[green, pos=0.13,above left]{$11$}	(x18)

		(x10) edge 	node[purple, pos=0.2,above]{$0$}	(x16)
		(x10) edge 	node[purple, pos=0.2,above]{$5$}	(x17)
		(x10) edge 	node[purple, pos=0.2,above left]{$9$}	(x18)
		
		(x11) edge 	node[orange, pos=0.2,above]{$0$}	(x17)
		(x11) edge 	node[orange, pos=0.2,above]{$5$}	(x18)

		(x12) edge 	node[above left]{$0$}	(x18)
		%cuarta etapa
		(x15) edge 	node[blue, pos=0.2,above]{$16$}	(x20)
		(x14) edge 	node[red, pos=0.2,above]{$14$}	(x20)
		(x13) edge 	node[green, pos=0.2,above]{$12$}	(x20)
		(x16) edge 	node[purple, pos=0.2,above]{$7$}	(x20)
		(x17) edge 	node[orange, pos=0.2,above]{$3$}	(x20)
		(x18) edge 	node[black, pos=0.13,above]{$0$}	(x20)

		;
\end{tikzpicture}

	\end{center}
	\newpage
\item
	Iteraciones por etapas de decisión.\\
	\begin{table}[ht]
		\caption{\textbf{Cuarta etapa}, Titirileufú.}
		\centering
		\begin{tabular}{ |c|c|c| }                 
		\hline\hline
		S & $f^{\*}_4(S)$ & $X^{\*}_4$ \\
		\hline
		5 & 16 & 0\\
		4 & 14 & 0\\
		3 & 12 & \rojo{0}\\
		2 & 7  & 0\\
		1 & 3  & 0\\
		0 & 0  & 0\\
		\hline 
		\end{tabular}
	%\end{table}
	%\begin{table}[ht]
		\caption{\textbf{Tercera etapa}, Titirilahue.}
		\centering
		\begin{tabular}{ |c|cccccc|cc| }                 
		\hline
		 & \multicolumn{6}{c|}{$f_3(S,X_3) = C_{X_3} + f^{*}_4(S)$} & & \\
		\diagbox{S}{$X_3$} &5&4&3&2&1&0& $f^{*}_4(S)$ & $X^{*}_4$ \\
		\hline
		5 & 16 & 19 & 21 & 18 & 13 & 9 & 21 & 3\\
		4 & - & 14 & 17 & 16 & 14 & 10 & 17 & \rojo{3}\\
		3 & - & - & 12 & 12 & 12 & 11 & 12 & 3/2/1\\
		2 & - & - & - & 7 & 8 & 9 & 9 & 0\\
		1 & - & - & - & - & 3 & 5 & 5 & 0\\
		0 & - & - & - & - & - & 0 & 0 & 0\\
		\hline 
		\end{tabular}
	%\end{table}
	%\begin{table}[ht]
		\caption{\textbf{Segunda etapa}, Titiritalca.}
		\centering
		\begin{tabular}{ |c|cccccc|cc| }                 
		\hline
		 & \multicolumn{6}{c|}{$f_2(S,X_2) = C_{X_2} + f^{*}_3(S)$} & & \\
		\diagbox{S}{$X_2$} &5&4&3&2&1&0& $f^{*}_2(S)$ & $X^{*}_2$ \\
		\hline
		5 & 21 & 23 & 20 & 19 & 16 & 12 & 23 & \rojo{4}\\
		4 & - & 17 & 18 & 17 & 15 & 11 & 18 & 3\\
		3 & - & - & 12 & 15 & 13 & 10 & 15 & 2\\
		2 & - & - & - & 9 & 11 & 8 & 11 & 1\\
		1 & - & - & - & - & 5 & 6 & 6 & 0\\
		0 & - & - & - & - & - & 0 & 0 & 0\\
		\hline 
		\end{tabular}
	%\end{table}
	%\begin{table}[ht]
		\caption{\textbf{Primera etapa}, Titirilquen.}
		\centering
		\begin{tabular}{ |c|cccccc|cc| }                 
		\hline
		 & \multicolumn{6}{c|}{$f_1(S,X_1) = C_{X_1} + f^{*}_1(S)$} & & \\
		\diagbox{S}{$X_1$} &5&4&3&2&1&0& $f^{*}_1(S)$ & $X^{*}_1$ \\
		\hline
		5 & 23 & 22 & 22 & 20 & 18 & 15 & 23 & \rojo{5}\\
		\hline 
		\end{tabular}
	\end{table}
\item
	Por lo tanto, la solución encontrada es: \\
	$5 \rightarrow 5 \rightarrow 4 \rightarrow 3 \rightarrow 0$.\\
	Obteniendo así 2300 votos a favor del candidato a senador Triviño.\\
	\emph{Supuesto}: Se ocupan todos los comerciales disponibles dado que
	no existe restricción para ello y queremos maximizar los votantes.
\end{enumerate}	
