%tex_origin: ppt.tex
\subsection{Sockets.h}
\begin{frame}
  \frametitle{Sockets}
  \begin{itemize}
  \item	
	  Un socket es un mecanismo por el cual se traspasan mensajes entre
	  aplicaciones, generalmente en diferentes computadores.
  \item	
	  Está definido al menos por las direcciones IP, números de puerto y el 
	  protocolo de transporte.
  \item
	  Los protocolos más utilizados son TCP y UDP.
  \item
	  Implementan la arquitectura cliente-servidor, el programa que inicia la 
	  comunicación será el cliente, mientras que el que responde actúa como
	  servidor.
  \end{itemize}
\end{frame}

\begin{frame}
  \frametitle{socket.h}
  \begin{itemize}
  \item	
	  API para el trabajo con sockets en c para linux.
  \item
	  Soporta tanto IPv4 como IPv6.
  \item
	  Incorpora funciones para la creación de sockets, convención entre
	  formatos, y transmisión de datos.
  \item
	  Define estructuras para el manejo de direcciones.
  \end{itemize}
\end{frame}

\begin{frame}
	\frametitle{socket.h (continuación)}
	En general el procedimiento es el siguiente:
	\begin{itemize}
		\item
			Se crea la estructura para manejar la dirección.
			\texttt{sockaddr\_in} para IPv4 y \texttt{sockaddr\_in6} para IPv6.
		\item
			Se crea el socket y se le asigna un descriptor.
			\texttt{socket();}
		\item
			Se asocia el socket y la estructura de dirección. \texttt{bind();} 
		\item
			Se pone al socket como ``escucha'' esperando conexiones.
			\texttt{accept();}
		\item
			Cuando una conexión es detectada se crea un nuevo socket, al que se
			le llama ``socket conectado''.
		\item
			Se puede lanzar un subproceso para continuar ``escuchando'' y 
			trabajar con el ``socket conectado''.
	\end{itemize}
\end{frame}

%\begin{frame}
%	\frametitle{Ejemplo de servidor iterativo.}
%	
%\end{frame}
