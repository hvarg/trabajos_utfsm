\documentclass[spanish, fleqn]{article}
\usepackage[english]{babel}
\usepackage[utf8]{inputenc}
\usepackage{amsmath}
\usepackage{amsfonts}
%\usepackage{wasysym}
%\usepackage{mathrsfs}
\usepackage[colorlinks, urlcolor=blue]{hyperref}
\usepackage[top = 2.5cm, bottom = 2cm, left = 2.5cm, right = 2.5cm]{geometry}
\usepackage{fancyhdr, graphicx}
\usepackage{caption}

\usepackage{changepage}
\usepackage{wrapfig}

\renewcommand{\headrulewidth}{0pt}
\fancyhead[L]{\includegraphics[width=3cm]{Logo_INFO.png}}
\fancyhead[R]{\includegraphics[width=3.2cm]{Logo_UTFSM.png}}


\title{ TODO: Titulo}
\author{Hernán Vargas, 201073009-3 \\ hernan.vargas@alumnos.usm.cl}
\date{\today}

\begin{document}
	\maketitle
	\thispagestyle{empty}
%	\newpage
%	\null
%	\vskip 4em
%	\begin{center}
%		\textsc{\Large Universidad Técnica Federico Santa María.}\\[0.4cm]
%		\textsc{\Large Sistemas y Organizaciones.}\\[0.2cm]
%		\textsc{\LARGE Análisis de la Estructura Organizacional.}\\[1.2cm]
%	\end{center}

	\thispagestyle{fancy}
	\section{Introducción.} %Poner buen título.

	\section{Resumen.} %Titulo..
	ref\cite{r2}
	\newpage
	\newgeometry{left=2.5cm, right=2.5cm, top=2.5cm, bottom=2.5cm}
	\section{Aplicación.} %tit
	algo\footnote{nota al pie.}
	\begin{minipage}{0.75\textwidth}
		imagen:
	\end{minipage}
	\hfill
	\begin{minipage}{0.25\textwidth}
    	\begin{center}
    	   % \includegraphics[width=0.85\textwidth]{new_afiche.png}
	    	referencia\cite{r1}.
		\end{center}
	\end{minipage}

	\section{Consecuencias y conclusiones.}
	%Concluciones generales, terminar con una buena frase.

	\renewcommand{\refname}{\selectfont 5 \, Referencias} % Hack para nombre
	\begin{thebibliography}{x}
		\bibitem{r1}
			\textit{TODO: titulo1} - Descripción o link 1.
		\bibitem{r2}
			\textit{TODO: titulo2} - Descripción o link 2.
	\end{thebibliography}

	%Referencias.
	\begin{flushright}
		\textbf{Palabras clave:} \emph{TODO}
		\begin{flalign*}
			&&\text{\textbf{Tiempo SCT}: \emph{Análisis del artículo}} &= 0:00 \\
										 &&\emph{Creación del informe} &= 0:00 \\
											            &&\text{Total} &= 0:00
		\end{flalign*}
	\end{flushright}

\end{document}
