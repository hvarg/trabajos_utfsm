\documentclass[spanish, fleqn]{article}
\usepackage[english]{babel}
\usepackage[utf8]{inputenc}
\usepackage{amsmath}
\usepackage{amsfonts}
%\usepackage{wasysym}
%\usepackage{mathrsfs}
\usepackage[colorlinks, urlcolor=blue]{hyperref}
\usepackage[top = 2.5cm, bottom = 2cm, left = 2.5cm, right = 2.5cm]{geometry}
\usepackage{fancyhdr, graphicx}
\usepackage{caption}

\usepackage{changepage}
\usepackage{wrapfig}

\renewcommand{\headrulewidth}{0pt}
\fancyhead[L]{\includegraphics[width=3cm]{Logo_INFO.png}}
\fancyhead[R]{\includegraphics[width=3.2cm]{Logo_UTFSM.png}}

\title{ Analisís del documental ``La Corporación''}
\author{Hernán Vargas, 201073009-3 \\ hernan.vargas@alumnos.usm.cl}
\date{}

\begin{document}
	\maketitle
	\thispagestyle{empty}
%	\newpage
%	\null
%	\vskip 4em
%	\begin{center}
%		\textsc{\Large Universidad Técnica Federico Santa María.}\\[0.4cm]
%		\textsc{\Large Sistemas y Organizaciones.}\\[0.2cm]
%		\textsc{\LARGE Análisis de la Estructura Organizacional.}\\[1.2cm]
%	\end{center}

	\thispagestyle{fancy}
	\section{La Corporación como persona jurídica.}
	El documental comienza mostrándonos como las corporaciones pasaron de ser 
	meras instituciones dedicadas a los negocios a ser prácticamente personas
	independientes por medio de resquicios legales que aún hoy les brindan las 
	libertades que poseen. En éste ámbito las corporaciones pasan a ser entes
	prácticamente autónomos con la única voluntad de generar ingresos
	monetarios.\\
	La gestión, en este aspecto, se ve limitada a la hora de la toma de 
	decisiones, puesto que la corporación tiene ``vida propia''. En el 
	documental podemos ver como algunos \texttt{SEO} dicen sentirse obligados,
	a tomar las decisiones más rentables por sobre lo que es más responsable 
	socialmente. Muchas veces no pueden notar las consecuencias de sus
	acciones, puesto que estás toman lugar en terrenos muy alejados de su día a
	día.\\
	Aún así, creo que una buena gestión debe ser responsable con sus procesos 
	productivos e incentivar el uso de técnicas sustentables tanto social
	como naturalmente. Espero que algún día podamos concentrarnos más en no
	hacer daño al prójimo que en generar ganancias.

	\section{Media y la guerra de patentes.}
	En EEUU prácticamente es posible patentarlo todo. La guerra de patentes por
	parte de las corporaciones son un intento más de asegurar su propósito de
	vida: las ganancias. En el ámbito donde judicialmente les es imposible
	poner una patente, es decir en nuestras vidas, recurren a un medio
	aparentemente más inofensivo pero no por ello menos devastador: la
	publicidad y manipulación de la información.\\
	Las corporaciones, al no poder ``patentarnos'' legalmente optan por 
	amarrarnos a sus productos con publicidad desde pequeños, obligándonos a 
	pensar de la forma que ellos quieren que pensemos y ocultando lo que no les
	conviene que escuchemos. Es aquí donde la responsabilidad social nos dice 
	que existe un limite a la hora de patentar y publicitar. La manipulación de
	la información ni siquiera es necesaria, una empresa responsable socialmente
	no tiene la por qué recurrir a ella.

	\section{Corporación, democracia y conclusiones.}
	El hambre de capital por parte de las maquinas imperialistas llamadas
	corporaciones no conoce limite. No le importa la moral, la ética, la
	nacionalidad, la cultura ni la lealtad. La corporación intentará generar
	ingresos de todos ellos. Es en este punto donde podemos ver más claramente
	cual es el verdadero problema de las corporaciones: es que no son humanos.
	Por más que se intente dar el estatus de humano a un ente como éste, una 
	corporación es fundamentalmente diferente. En éste sentido podemos entender
	más a la corporación como una maquina, no es mala en sí, pero con objetivos
	equivocados actuará de malas maneras.\\
	La gran solución sería cambiar el objetivo fundamental de las corporaciones,
	la generación de ganancias está bien, pero existen cosas más importantes
	como el bienestar de la sociedad y el trato digno. Las corporaciones del
	futuro deben aprender a dirigir sus esfuerzos hacia un clima de bienestar 
	social generalizado y no esperar las ganancias por sobre todo. La gestión 
	en esté proceso será fundamental y por ello es fundamental que nosotros, las
	``nuevas generaciones'', seamos conciertes del clima actual y podamos hacer
	algo para arreglarlo.

%	\renewcommand{\refname}{\selectfont 5 \, Referencias} % Hack para nombre
%	\begin{thebibliography}{x}
%		\bibitem{r2} None
%	\end{thebibliography}

	%Referencias.
	\begin{minipage}{0.43\textwidth}
		\textbf{Palabras clave:} \emph{Corporación, Documental,
										Responsabilidad Social.}
	\end{minipage}
	\hfill
	\begin{minipage}{0.40\textwidth}
	\begin{flushright}
		\begin{flalign*}
			&&\text{\textbf{Tiempo SCT}: \emph{Documental}} &= 3:00\\
							  &&\emph{Creación del informe} &= 0:50\\
		\end{flalign*}
	\end{flushright}
\end{minipage}

\end{document}
