\documentclass[spanish, fleqn]{article}
\usepackage[english]{babel}
\usepackage[utf8]{inputenc}
\usepackage{amsmath}
\usepackage{amsfonts}
%\usepackage{wasysym}
%\usepackage{mathrsfs}
\usepackage[colorlinks, urlcolor=blue]{hyperref}
\usepackage[top = 2.5cm, bottom = 2cm, left = 2.5cm, right = 2.5cm]{geometry}
\usepackage{fancyhdr, graphicx}
\usepackage{caption}

\usepackage{changepage}
\usepackage{wrapfig}

\renewcommand{\headrulewidth}{0pt}
\fancyhead[L]{\includegraphics[width=3cm]{Logo_INFO.png}}
\fancyhead[R]{\includegraphics[width=3.2cm]{Logo_UTFSM.png}}


\title{ Sistemas y Organizaciones. \\ 
		Sobre indicadores y su aplicación en la Sociedad de Debate.}
\author{Hernán Vargas, 201073009-3 \\ hernan.vargas@alumnos.usm.cl}
\date{\today}

\begin{document}
	\maketitle
	\thispagestyle{empty}
%	\newpage
%	\null
%	\vskip 4em
%	\begin{center}
%		\textsc{\Large Universidad Técnica Federico Santa María.}\\[0.4cm]
%		\textsc{\Large Sistemas y Organizaciones.}\\[0.2cm]
%		\textsc{\LARGE Análisis de la Estructura Organizacional.}\\[1.2cm]
%	\end{center}

	\thispagestyle{fancy}
	\section{¿Qué debemos medir?}
	El control de la información para predecir el comportamiento futuro del ambiente
	es fundamental para toda organización, aún así, existe un gran desconocimiento a
	la hora de elegir qué medir y como hacerlo. Muchas empresas optan por ocupar
	sus recursos en hacer análisis financiero, lo cual es correcto, pero con el
	tiempo notan que es insuficiente. En esta situación se enfrentan a la
	problemática de elegir qué medir, llegando a veces al extremo de intentar
	medirlo todo, proceso que es costoso y en gran parte inútil.\\
	Para solucionar nuestra problemática tengamos en cuenta lo que el articulo
	leído define como ``control de gestión'' según Robert Anthony: 
	\begin{quote}
		Proceso mediante el cual los ejecutivos aseguran que los recursos estén
		disponibles y sean usados efectiva y eficientemente para cumplir los
		objetivos de la organización.\cite{merc}
	\end{quote}
	Con esto en mente, los indicadores que debemos buscar son los que apoyen 
	directamente a la realización de la estrategia, para ello, necesitamos que 
	ésta sea clara y ejecutable, luego, nos basta con dividirla en una serie 
	de no más de diez objetivos estratégicos cuantificables y crear sus respectivos
	indicadores.

	\section{Indicadores.}
	Un indicador es aquello medible de un objetivo estratégico, nos ayudarán a
	saber cuanto nos falta para llegar a la meta que la organización planteó,
	aún así, no todo indicador nos da información útil en la toma de decisiones.
	Una buena manera de saber si nuestros indicadores son buenos es verificando
	las siguientes características; Un indicador debe:
	\begin{enumerate}
		\item
			\textbf{Medir el objetivo estrategico:} Obviamente el indicador debe
			ser el adecuado para medir el objetivo del cual proviene. Por ejemplo,
			si nuestro objetivo estratégico es \emph{Ampliación de la
			infraestructura} un buen indicador será la cantidad de metros cuadrados
			construidos.
		\item
			\textbf{Ser medible:} Es de vital importancia que el indicador tenga
			un carácter cuasi-numérico, de esta manera podemos saber cuan cerca
			estamos de la meta en todo momento. Por ejemplo, si nuestro objetivo
			es la \emph{Satisfacción del cliente}, buenos indicadores serán las
			encuestas con preguntas de escala de satisfacción.
		\item
			\textbf{Medir diferentes ambitos:} Un indicador toma verdadera
			relevancia cuando es capaz de medir tanto la causa como la consecuencia
			de las acciones que afectan a su objetivo especifico. El ejemplo de
			\emph{Satisfacción al cliente} podemos agregar preguntas del tipo 
			``cuanto está dispuesto a pagar'' y obtenemos además información 
			financiara.
		\item
			\textbf{Permitir el aprendizaje:} La información obtenida por estos
			indicadores debe permitir aprender de la situación actual y así, si
			es necesario, cambiar la estrategia.
		\item
			\textbf{Estar bien documentado:} Todo indicador debe estar anotado en
			nuestro ``diccionario de indicadores'' donde pondremos con que objetivo
			estratégico se liga, quien es el responsable, los límites, las metas 
			entre otros.
	\end{enumerate}
	\newpage
	\newgeometry{left=2.5cm, right=2.5cm, top=2.5cm, bottom=2.5cm}
	\section{La Sociedad de Debate y sus indicadores.}
	Como he descrito en informes anteriores\footnote{Entregable 1, ``La estrategia
	de Jobs en la Sociedad de Debate''.}, la Sociedad de Debate USM se encuentra 
	en un periodo de\\
	\begin{minipage}{0.75\textwidth}
	transición entre capitanes. Uno de los principales
	intereses de la nueva dirigencia es fortalecer la imagen publica de la sociedad
	sumada al constante interés en la integración de nuevos estudiantes, de esto
	podemos suponer los siguientes indicadores:\\
	Para fortalecer la imagen publica de la sociedad nos planteamos como objetivos
	específicos el reconocimiento por parte de la prensa local y organizar  más
	eventos públicos. Por otro lado, para aumentar la cantidad de miembros 
	podemos publicitarnos con afiches por la universidad y hacer charlas
	introductorias. El ejercicio se resume a la siguiente tabla:
	\end{minipage}
	\hfill
	\begin{minipage}{0.25\textwidth}
    	\begin{center}
    	    \includegraphics[width=0.85\textwidth]{new_afiche.png}
	    	Afiche utilizado el año 2014\cite{img}.
		\end{center}
	\end{minipage}
	\begin{figure}[!htbp]
	\begin{tabular}{|c|c|c|c|}
		\hline
		\textbf{Objetivo} & \textbf{Indicador} & \textbf{Meta} & \textbf{Iniciativa} \\
		\hline
		Reconocimiento de la prensa local. & \# artículos de prensa. & 6 & 
			Organización de Torneos locales.\\
		\hline
		Organización de eventos públicos. &\# eventos organizados & 10 & 
			Alianza con otras organizaciones. \\
		\hline
		Aumentar ingreso por afiches. &\% alumnos ingresados & 30\% &
			Distribuir afiches en casa central.\\
		\hline
		Aumentar ingreso por charlas. &\% alumnos ingresados & 40\% &
			Charlas para todas las carreras.\\
		\hline
		Permanencia en la sociedad &\# miembros  & 8 & 
			Apoyo indirecto en otras áreas.\\
		\hline
	\end{tabular}
	\end{figure}\\
	Estos son solo algunos ejemplos de los indicadores que podemos obtener de la
	situación planteada\footnote{Los números y porcentajes son solo una referencia
	y no representan necesariamente los intereses de la Sociedad de Debate USM}.
	Para medir la razón de ingreso podemos hacer encuestas, y en ellas medir
	además otros factores importantes.

	\section{Consecuencias y conclusiones.}
	Los indicadores son parte inherente de nuestra vida. En toda tarea con objetivos
	específicos la existencia de indicadores nos da cierta seguridad y condiciona
	nuestra toma de decisiones. En un acto tan simple como caminar hacia la universidad
	podemos ver como los indicadores cambian completamente nuestra ruta, por
	ejemplo, conocemos el vecindario, cuanta distancia nos falta por recorrer y
	a que hora debemos llegar, con solo ésta información nuestro actuar puede
	cambiar drásticamente. Si voy atrasado puedo acelerar el paso. Si tengo
	tiempo quizás pueda pasear por el parque cercano, etcétera.\\
	Esta misma situación le ocurre a toda organización, la diferencia radica en
	cuan fácil nos resulta obtener la información. Los indicadores son una buena
	manera de organizar y analizar lo que es importante y dejar de lado lo que no
	lo es para modificar nuestra ruta en base a ello.

	\renewcommand{\refname}{\selectfont 5 \, Referencias} % Hack para nombre
	\begin{thebibliography}{x}
		\bibitem{merc}
			\textit{¿Cuáles y cuántos indicadores usar?} - Mercurio clase ejecutiva,
			19 de mayo del 2014.
		\bibitem{img}
			\textit{Afiche Taller de Debate 2014} - Diseñado por Nicolás Hermosilla
			en conjunto con la Sociedad de Debate USM.
	\end{thebibliography}

	%Referencias.
	\begin{flushright}
		\textbf{Palabras clave:} \emph{Indicadores, Objetivos específicos, Estrategia.}
		\begin{flalign*}
			&&\text{\textbf{Tiempo SCT}: \emph{Análisis del artículo}} &= 0:30 \\
			&&\emph{Creación del informe} &= 4:13 \\
			&&\text{Total} &= 4:43
		\end{flalign*}
	\end{flushright}

\end{document}
